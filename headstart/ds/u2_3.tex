\documentclass{article}
\usepackage[utf8]{inputenc}
\usepackage{enumerate}

\title{Übung 2}
\author{Laurenz Weixlbaumer, 11804751}
\date{Oktober 2018}

\renewcommand{\arraystretch}{1.3}
\renewcommand\thesubsection{(\alph{subsection})}

\usepackage{mathtools}

\begin{document}

\maketitle

\stepcounter{section}\stepcounter{section} % Start at 3
\section{Ternärsystem}

\begin{enumerate}[{(\alph{enumi})}]
    \item Wandle die Dezimalzahl 69 in das Ternärsystem um. Gib dazu den Lösungsweg an.
    
    Der Basisumwandlung der Variante 2 folgend, kann die Dezimalzahl wie folgend in das Ternärsystem umgewandelt werden.
    
    \begin{center}
    \begin{tabular}{c | c}
        Rechnung & $Rest_3$\\
        \hline
        $69 : 3 = 23$ & 0\\
        $23 : 3 = 7$ & 2\\
        $7 : 3 = 2$ & 1\\
        $2 : 3 = 0$ & 2\\
        $0 : 3 = 0$ & 0\\
    \end{tabular}
    \end{center}
    
    Die Ternärzahl kann nun von oben nach unten aus den jeweiligen Resten abgelesen werden, demnach
    
    $$
    69_{10} = 2120_3
    $$
\end{enumerate}

\end{document}
