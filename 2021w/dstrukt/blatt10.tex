\documentclass{article}
\usepackage[utf8]{inputenc}
\usepackage[ngerman]{babel}

% Convenience improvements
\usepackage{csquotes}
\usepackage{enumitem}
\setlist[enumerate,1]{label={\alph*)}}
\usepackage{amsmath}
\usepackage{amssymb}
\usepackage{mathtools}

% Proper tables and centering for overfull ones
\usepackage{booktabs}
\usepackage{adjustbox}

% Change page/text dimensions, the package defaults work fine
\usepackage{geometry}

\usepackage{parskip}

% Drawings
\usepackage{tikz}

% Adjust header and footer
\usepackage{fancyhdr}
\pagestyle{fancy}
\fancyhead[L]{Diskrete Strukturen --- \textbf{Übungsblatt 10}}
\fancyhead[R]{Laurenz Weixlbaumer (11804751)}
\fancyfoot[C]{}
\fancyfoot[R]{\thepage}
% Stop fancyhdr complaints
\setlength{\headheight}{12.5pt}

\newcommand{\R}{\mathbb{R}\ \\\ \{0\}}

\newcommand{\frectangle}{\tikz[scale=0.7, baseline]{\draw[fill] (0, 0) rectangle (1em, 1em)}}
\newcommand{\fcircle}   {\tikz[scale=0.7, baseline]{\draw[fill] (0.5em, 0.5em) circle [radius=0.5em]}}
\newcommand{\ftriangle} {\tikz[scale=0.7, baseline]{\draw[fill] (0, 0) -- (0.5em, 1em) -- (1em, 0) -- cycle}}

\newcommand{\cmod}{\text{mod}}

%\newcommand{\gcd}{\text{gcd}}

\begin{document}

\paragraph{Aufgabe 1.}

\begin{enumerate}
    % TODO: Verify
    \item Für $a = 868318803$ und $b = 1601135481$ gilt $\gcd (a, b) = 3531$, $u = -104598$ und $v = 56725$.
    
    Für $a = 911761172$ und $b = 573241334$ gilt $\gcd (a, b) = 958$, $u = -146941$ und $v = 233715$.
\end{enumerate}

\paragraph{Aufgabe 2.}

\begin{enumerate}
    % TODO: Use set builder notation, this is probably enough already but it can't hurt

    \item Alle $x \in [6]_{\equiv_{13}}$, bzw. alle $x \in \{ 13y + 6 : y \in \mathbb{Z} \}$.
    % $x \in \{ \ldots, -20, -7, 6, 19, 32, \ldots \}$.

    \item Alle $x \in [3]_{\equiv_{12}}$, bzw. alle $x \in \{ 12y + 3 : y \in \mathbb{Z} \}$
    % $x \in \{ \ldots, -21, -9, 3, 15, 27, \ldots \}$.
    
    \item Es gibt keine derartigen $x$.
    
    \item Alle $x \in [4]_{\equiv_{12}}$, bzw. alle $x \in \{ 12y + 4 : y \in \mathbb{Z} \}$
    % $x \in \{ \ldots, -20, -8, 4, 16, 28, \ldots \}$.
\end{enumerate}

\paragraph{Aufgabe 3.}

Behauptet ist
\begin{equation*}
    \exists\, [y]_{\equiv_{m}} \in \mathbb{Z}_m : [x]_{\equiv_{m}} \cdot [y]_{\equiv_{m}} = [1]_{\equiv_{m}}\Longleftrightarrow \gcd(x, m) = 1.
\end{equation*}

\enquote{$\Rightarrow$} Angenommen es gibt ein solches $[y]_{\equiv_{m}}$. Es gilt also $xy \equiv_{m} 1$ beziehungsweise $\cmod(xy, m) = \cmod(1, m)$, woraus folgt $\cmod(xy, m) = 1$ nachdem $\cmod(1, m) = 1 - m \left\lfloor{1 / m}\right\rfloor = 1 - m \cdot 0 = 1$. (Nachdem $m \geq 2$ gilt für alle $m$, dass $\left\lfloor{1 / m}\right\rfloor = 0$.) Weiters gilt $\gcd(xy, m) = 1$ nachdem
\begin{align*}
    \{ g, g' \} &= \{ xy, m \}\\
    \{ g, g' \} &= \{ m, \cmod(xy, m) \}\\
    \{ g, g' \} &= \{ 1, \cmod(m, 1) \}\\
    \{ g, g' \} &= \{ 1, 0 \},
\end{align*}
wobei der Algorithmus bei $g' = 0$ mit $\gcd(xy, m) = g = 1$ terminiert. (Nachdem $m \in \mathbb{N}$ gilt für alle $m$, dass $\left\lfloor{m / 1}\right\rfloor = m$ und ergo $\cmod(m, 1) = 0$.)

Die Primfaktorzerlegungen von $xy = \chi_1\chi_2 \ldots \chi_i$ und $m = m_1m_2 \ldots m_j$ sind nun also voneinander verschieden (haben eine leere Schnittmenge). Das Produkt $xy$ kann weiter in die Primfaktoren von $x$ und $y$, also $xy = x_1x_2 \ldots x_k y_1y_2 \ldots y_l$ zerlegt werden. Nachdem die Primfaktoren von $xy$ und $m$ voneinander verschieden sind, müssen auch die Primfaktoren von $x$ und $m$ (und $y$ und $m$) voneinander verschieden sein. Demzufolge gilt $\gcd(x, m) = 1$ (und $\gcd(y, m) = 1$), was zu zeigen war.

%(Siehe Skriptum, S. 58: Der euklidische Algorithmus $\gcd(a, b)$ berechnet eine Folge von Divisionsresten $r_1 = a$, $r_2 = b$, \ldots. Jeweils zwei aufeinanderfolgende Zahlen $r_i$ haben den gleichen größten gemeinsamen Teiler.)
%(Gedankenschritt: Es gelte $\gcd(a, c) = 1$. Warum sollte nun $\gcd(ab, c) \neq 1$ gelten?) % TODO

\enquote{$\Leftarrow$} Angenommen es gilt $\gcd(x, m) = 1$. Zu zeigen ist, dass es ein $[y]_m \in \mathbb{Z}$ gibt mit $xy \equiv_m 1$ beziehungsweise $\cmod(xy, m) = 1$. 

%Man wähle ein $y$ derart, dass $\gcd(y, m) = 1$. Dann gilt $\gcd(xy, m) = 1$, woraus folgt, dass es $u, v \in \mathbb{Z}$ gibt derart, dass $uxy + vm = 1$.

Nachdem $\gcd(x, m) = 1$ gibt es $y, v \in \mathbb{Z}$ mit $yx + vm = 1$. Dann gilt
\begin{equation*}
    [1]_{\equiv_{m}} = [y]_{\equiv_{m}}[x]_{\equiv_{m}} + \underbrace{[vm]_{\equiv_{m}}}_{= [0]_{\equiv_{m}}} = [y]_{\equiv_{m}}[x]_{\equiv_{m}},
\end{equation*}
woraus folgt, dass $[y]_{\equiv_{m}}$ das gesuchte multiplikative Inverse von $[x]_{\equiv_{m}}$ ist, wie zu zeigen war.

%Seien $x_1, x_2, \ldots x_i$ die Primfaktoren von $x$, $y_1, y_2, \ldots y_j$ jene von $y$ und $m_1, m_2, \ldots, m_k$ jene von $m$. Nachdem $\gcd(x, m) = 1$ und $\gcd(y, m) = 1$ sind alle diese Primzahlen voneinander verschieden. Das Produkt $xy$ hat nun die Primfaktoren $x_1, x_2, \ldots x_i, y_1, y_2, \ldots y_j$, wovon keine in $m_1, m_2, \ldots, m_k$ sind. Es gilt demzufolge $\gcd(xy, m) = 1$, was zu zeigen war.

%Analog zu Satz 11 sei nun $A = [a]_m \in \mathbb{Z}$ beliebig. Es gibt $u, v \in \mathbb{Z}$ derart, dass $ux + vm = \gcd(x, m)$.

%Sei $A = [a]_m \in \mathbb{Z}$ beliebig. Wegen $[a]_{\equiv_m} = [\cmod(a, m)]_{\equiv_m}$ und $0 \leq \cmod(a, m) < m$ können wir annehmen $0 \leq a < m$ (siehe Satz 11, Skriptum). Wegen $m \in \mathbb{N}\,\backslash\{0, 1\}$ 

%Angenommen, $\gcd(x, m) = 1$ gilt nicht. Es gibt also $u, v \in \mathbb{Z}\backslash\{0\}$ für die gilt, dass $x \cdot uv = m$. Es gilt also weiters $u \mod m = u \neq 0$ und $v \mod m = v \neq 0$, also $[u]_{\equiv_{m}} \neq [0]_{\equiv_{m}}$ und $[v]_{\equiv_{m}} \neq [0]_{\equiv_{m}}$.

\end{document}

mod(xy, m) = 1

gcd(xy, m)

{g, g'} = {xy, m}
{g, g'} = {m, mod(xy, m)}
{g, g'} = {1, mod(m, 1)}
{g, g'} = {1, 0}

--

gcd(x, m)

{g, g'} = {x, m}
{g, g'} = {m, mod(x, m)}
{g, g'} = {}