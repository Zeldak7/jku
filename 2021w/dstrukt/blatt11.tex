\documentclass{article}
\usepackage[utf8]{inputenc}
\usepackage[ngerman]{babel}

% Convenience improvements
\usepackage{csquotes}
\usepackage{enumitem}
\setlist[enumerate,1]{label={\alph*)}}
\usepackage{amsmath}
\usepackage{amssymb}
\usepackage{mathtools}

% Proper tables and centering for overfull ones
\usepackage{booktabs}
\usepackage{adjustbox}

% Change page/text dimensions, the package defaults work fine
\usepackage{geometry}

\usepackage{parskip}

% Drawings
\usepackage{tikz}

% Adjust header and footer
\usepackage{fancyhdr}
\pagestyle{fancy}
\fancyhead[L]{Diskrete Strukturen --- \textbf{Übungsblatt 11}}
\fancyhead[R]{Laurenz Weixlbaumer (11804751)}
\fancyfoot[C]{}
\fancyfoot[R]{\thepage}
% Stop fancyhdr complaints
\setlength{\headheight}{12.5pt}

\newcommand{\R}{\mathbb{R}\ \\\ \{0\}}

\newcommand{\cmod}{\text{mod}}

\usepackage{seqsplit}

\begin{document}

\paragraph{Aufgabe 1.}

\emph{Public keys} werden f\"ur Verschl\"usselung, \emph{private keys} f\"ur Entschl\"usselung verwendet (Skriptum, S. 62). Ich interpretiere die Aufgabe also so, dass in (a) ein \emph{private key} $(m, d)$ gegeben ist und (b) die Rekonstruktion eines \emph{public keys} zur Verschl\"usselung verlangt.

\begin{enumerate}
    \item Es gilt $m = 2471004485256198699723347$ und $d = 679256688868919115503281$. Dann gilt $\cmod(x^d, m) = 282033334039281634 = \texttt{merryxmas}$.
    
    \item Man zerlege $m$ in die Primfaktoren $p = 1417021450037$ und $q = 1743801750631$. Daraus ergibt sich weiters $\phi = (p - 1)(q - 1) = 2471004485253037876522680$ und, nachdem $de \equiv_{\phi} 1$ gelten muss, $e = 2094576675656119524619681$ als modulares Inverses von $d$. Die Nachricht $\texttt{happynewyear}$ wird zu $x = 231631314029203840201633$ codiert, was zu
    \begin{equation*}
        \cmod(x^e, m) = 156693474749568634695296
    \end{equation*}
    verschlüsselt werden kann.
\end{enumerate}

% Gegeben ist ein \emph{public key} und eine verschlüsselte Nachricht. Verlangt ist eine Entschlüsselung der Nachricht gefolgt von einer Ableitung des \emph{private keys}.

% Das macht für mich aus mehreren Gründen keinen Sinn. Erstens werden die Begriffe \emph{public} und \emph{private key} dubios verwendet. Im ersten Teil soll mithilfe eines \emph{public keys} eine Nachricht entschlüsselt werden, im zweiten Teil mithilfe eines \emph{private keys} eine Nachricht verschlüsselt werden. Jeweils das Umgekehrte wäre sinnvoll --- \emph{public key} verschlüsselt, \emph{private key} entschlüsselt.

% Zweitens ergibt sich aus dem gegebenen \emph{public key} kein valider \emph{private key}. Man zerlege $m$ in die Primfaktoren $p = 1417021450037$ und $q = 1743801750631$. Daraus ergibt sich weiters $\phi = 2471004485253037876522680$ und $d = 2094576675656119524619681$ (eine von mehreren gleichwertigen Lösungen für $d$). Dann gilt aber für $x = 272797218939662984624437$, dass $\cmod(x^d, m) = \seqsplit{2191972902183325951471984}$, was kein valides Ergebnis ist.

% Interpretiert man aber das als \emph{public key} gegebenen Paar als \emph{private key} $(m, d)$, nimmt man also $m = 2471004485256198699723347$ und $d = 679256688868919115503281$, funktioniert die Entschlüsselung wie erwartet mit
% \begin{equation*}
%     \cmod(x^d, m) = 282033334039281634 = \texttt{merryxmas}.
% \end{equation*} (In diesem Fall muss auch der einleitende Satz zu \enquote{Ihr \emph{private key} ist\ldots. Sie erhalten von Ihrem Übungsleiter folgende Nachricht, die mit Ihrem \emph{public key} verschlüsselt wurde.} umgedacht werden.)

% Interpretiert man nun weiter den zweiten Teil als Ableitung eines \emph{public keys} aus einem \emph{private key} machen auch die Hinweise, die sonst eher beim ersten Teil hilfreich gewesen wären, mehr Sinn. Abzuleiten ist nun also $e$ aus dem gegebenen $m$ und $d$. Gelten muss $de \equiv_{\phi} 1$, das $\phi$ ist schon von oben bekannt und unverändert, es ergibt sich $e = 2094576675656119524619681$ als modulares Inverses von $d$. Die Nachricht $\texttt{happynewyear}$ wird zu $x = 231631314029203840201633$ codiert, was zu
% \begin{equation*}
%     \cmod(x^e, m) = 156693474749568634695296
% \end{equation*}
% verschlüsselt werden kann.

\paragraph{Aufgabe 2}

Der gegebene Mechanismus erzeugt keine besonders guten Zufallsziffern nachdem Ziffern $< 6$ bedeutend wahrscheinlicher als solche $\geq 6$ sind. Das ist dadurch bedingt, dass der Ausdruck $b_0 + 2b_1 + 4b_2 + 8b_3$ uniform Werte im Bereich $[0, 15]$ erzeugt. Werte in $[10, 15]$ werden durch den Modulo allerdings zu $[0, 5]$ gewandelt, sie kommen also h\"aufiger vor.

\begin{table}[h]
    \centering
    \makebox[0pt]{\begin{tabular}{c c c c c c c c c c}\toprule
        0 & 1 & 2 & 3 & 4 & 5 & 6 & 7 & 8 & 9 \\\midrule
        0.124757 & 0.124656 & 0.124566 & 0.125098 & 0.124697 & 0.125103 & 0.062505 & 0.062703 & 0.062811 & 0.063104 \\\bottomrule
    \end{tabular}}
    \caption{Relative H\"aufigkeit der m\"oglichen Ziffern, $n = 1000000$.}
\end{table}

Sei $B(i)$ eine Funktion die das Zufallsbit $b_i$ retourniert. Dann w\"are ein uniformer Generator die Funktion
\begin{equation*}
    f(i) = \begin{cases}
        b_{i} + 2b_{i + 1} + 4b_{i + 2} + 8b_{i + 3} & \text{f\"ur}\ b_{i} + 2b_{i + 1} + 4b_{i + 2} + 8b_{i + 3} < 10 \\
        f(i + 4) & \text{andernfalls} \\
    \end{cases}
\end{equation*}
welche so lange neue Bits durchprobiert bis eine Zahl $< 10$ das Ergebnis ist. Somit wird die Verwendung von $\cmod$ und die damit einhergehenden Probleme umgangen. 

\paragraph{Aufgabe 3}

\begin{enumerate}
    \item $\{ \ldots, -58, -23, 12, 47, 82, \ldots \}$ bzw. $\{ 35y + 12 : y \in \mathbb{Z} \}$.
    % -16 -9 -2 5 12 19 26 33 40 47 54
    % -8 -3 2 7 12 17 22 27 32 37 42 47
    \item $\{ \ldots, -43, -19, 5, 29, 53, \ldots \}$ bzw. $\{ 24y + 5 : y \in \mathbb{Z} \}$.
    % -1 5 11 17 23 29 35 41 47 53 59 65 71
    % -27 -19 -11 -3 5 13 21 29 37 45 53 61
    \item $\{ \ldots, -25, -1, 23, 47, 71, \ldots \}$ bzw. $\{ 24y + 23 : y \in \mathbb{Z} \}$.
    % -1, 5, 11, 17, 23, 29 
    % -1, 7, 15, 23, 
\end{enumerate}

\end{document}
