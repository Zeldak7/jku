\documentclass{article}
\usepackage[utf8]{inputenc}
\usepackage[ngerman]{babel}

% Convenience improvements
\usepackage{csquotes}
\usepackage{enumitem}
\setlist[enumerate,1]{label={\alph*)}}
\usepackage{amsmath}
\usepackage{amssymb}
\usepackage{mathtools}
\usepackage{tabularx}

% Proper tables and centering for overfull ones
\usepackage{booktabs}
\usepackage{adjustbox}

% Change page/text dimensions, the package defaults work fine
\usepackage{geometry}

\usepackage{parskip}

% Drawings
\usepackage{tikz}

% Adjust header and footer
\usepackage{fancyhdr}
\pagestyle{fancy}
\fancyhead[L]{Diskrete Strukturen --- \textbf{Übungsblatt 13}}
\fancyhead[R]{Laurenz Weixlbaumer (11804751)}
\fancyfoot[C]{}
\fancyfoot[R]{\thepage}
% Stop fancyhdr complaints
\setlength{\headheight}{12.5pt}

\newcommand{\R}{\mathbb{R}\ \\\ \{0\}}

\newcommand{\cmod}{\text{mod}}

\newcommand{\bO}{\text{O}}

\begin{document}

\paragraph{Aufgabe 1.}

 Zu zeigen ist
\begin{align*}
    f(n) = \bO(g(n)) \land g(n) &= \bO(h(n)) \Rightarrow f(n) = \bO(h(n)).
\end{align*}
Angenommen die linke Seite der Implikation,
\begin{align}
    \label{eq:1-1} & \exists\, c > 0\ \exists\, n_0 \in \mathbb{N} : \forall\, n \geq n_0 : |f(n)| \leq c|g(n)| \\
    \label{eq:1-2} & \exists\, k > 0\ \exists\, n_1 \in \mathbb{N} : \forall\, n \geq n_1 : |g(n)| \leq k|h(n)|
\end{align}
gilt. Die Ungleichung aus \eqref{eq:1-1} kann zu $\frac{|f(n)|}{c} \leq |g(n)|$ umformuliert werden und in \eqref{eq:1-2} eingesetzt werden:
\begin{align*}
    & \exists\, k > 0\ \exists\, n_1 \in \mathbb{N} : \frac{|f(n)|}{c} \leq k|h(n)| \\
    & \exists\, k > 0\ \exists\, n_1 \in \mathbb{N} : |f(n)| \leq ck|h(n)|
\end{align*}
Dann gilt
\begin{align*}
    \exists\, j > 0\ \exists\, n_2 \in \mathbb{N} : \forall\, n \geq n_2 : |f(n)| \leq j|h(n)| 
\end{align*}
für $j = ck$ und $n_2 = \text{max}\{n_0, n_1\}$. Somit gilt $f(n) = O(h(n))$ wenn $f(n) = O(g(n))$ und $g(n) = O(h(n))$, die $\bO$-Notation ist transitiv.

\paragraph{Aufgabe 2.}

Es gilt $f(n) = \sqrt{n} = \Theta(\sqrt{n})$.

\begin{enumerate}
    \item Es gilt $a = 1$, $b = 2$ und $c = \log_2(1) = 0$. Es greift Fall 4 wegen $f(n) = \Omega(n^{0 + \frac{1}{2}}) = \Omega(\sqrt{n})$ ($\epsilon = \frac{1}{2}$). Somit gilt $T(n) = \Omega(\sqrt{n})$.

    \item Es gilt $a = 2$, $b = 2$ und $c = \log_2(2) = 1$. Es greift Fall 1 wegen $f(n) = \bO(n^{1 - \frac{1}{2}}) = \bO(\sqrt{n})$ ($\epsilon = \frac{1}{2}$). Somit gilt $T(n) = \Theta(\sqrt{n})$.

    \item Es gilt $a = 2$, $b = 4$ und $c = \log_4(2) = \frac{1}{2}$. Es greifen Fall 2 und 3 wegen $f(n) = \bO(n^{\frac{1}{2}} \log(n)^0) = \bO(\sqrt{n})$ und $f(n) = \Omega(n^{\frac{1}{2}} \log(n)^0) = \Omega(\sqrt{n})$ ($k = 0$). Somit gilt $T(n) = \Theta(\sqrt{n} \log(n))$.
\end{enumerate}

\paragraph{Aufgabe 3.}

% TODO: Tabelle mit Frequency | Ops, jeweils  >= maximum

% Die folgende Tabelle zeigt die maximale Anzahl an Ausführungen für eine bestimmte Zeile, in Abhängigkeit von $n$. Es wurde immer vom schlechtesten Fall ausgegangen, 

Die folgende Tabelle zeigt für jede Zeile die maximale Anzahl der Ausführungen (\enquote{Häufigkeit}) und die daraus resultierende maximale Anzahl an Operationen. Nachdem etwa Zeile 3 zu einem früheren Abbruch von Schleifeniterationen führen kann sind diese Werte als oberes Limit zu verstehen, nicht zwingend als tatsächlich erreichbares Maximum.

\begin{tabular}{c c c l}
    Zeile & Häufigkeit & Operationen &  \\

    1   & 1         & 0             & \\
    2   & 1         & $n$           & \\
    3   & $n$       & $3n$          & \\
    4   & $n$       & $2n^2$        & Größtmögl. $i = n$ also $2n \cdot n = 2n^2$ \\
    5   & $2n^2$    & $12n^2$       & \\
    6   & $2n^2$    & $6n^3 + 1$    & Größtmögl. $j = 2i = 2n$ also $2n + n = 3n$ und $2n^{2} 3n = 6n^3$ \\
    7   & $6n^3$    & $0$           & \\
    8   & $6n^3$    & $18n^3$       & \\
    9   & $6n^3$    & 0             & \\
    10  & $6n^3$    & $24n^3$       & \\
    11  & 1         & 0             & \\
\end{tabular}

Die größte vorkommende maximale Anzahl der Operationen ist $24n^3$, somit benötigt der Algorithmus $O(n^3)$ arithmetische Operationen. 

\end{document}
