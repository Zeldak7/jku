\documentclass{article}
\usepackage[utf8]{inputenc}
\usepackage[ngerman]{babel}

% Convenience improvements
\usepackage{csquotes}
\usepackage{enumitem}
\usepackage{amsmath}
\usepackage{amssymb}

% Proper tables and centering for overfull ones
\usepackage{booktabs}
\usepackage{adjustbox}

% Change page/text dimensions, the defaults work fine
\usepackage{geometry}

\usepackage{parskip}

% Drawings
\usepackage{tikz}

% Adjust header and footer
\usepackage{fancyhdr}
\pagestyle{fancy}
\fancyhead[L]{Diskrete Strukturen --- \textbf{Übungsblatt 6}}
\fancyhead[R]{Laurenz Weixlbaumer (11804751)}
\fancyfoot[C]{}
\fancyfoot[R]{\thepage}
% Stop fancyhdr complaints
\setlength{\headheight}{12.5pt}

\newcommand{\R}{\mathbb{R}\ \\\ \{0\}}

\newcommand{\frectangle}{\tikz[scale=0.7, baseline]{\draw[fill] (0, 0) rectangle (1em, 1em)}}
\newcommand{\fcircle}   {\tikz[scale=0.7, baseline]{\draw[fill] (0.5em, 0.5em) circle [radius=0.5em]}}
\newcommand{\ftriangle} {\tikz[scale=0.7, baseline]{\draw[fill] (0, 0) -- (0.5em, 1em) -- (1em, 0) -- cycle}}

\begin{document}

\paragraph{Aufgabe 1.}

\begin{enumerate}[label=\alph*)]
    \item $h$ ist ein Graphenhomomorphismus. Es gilt
    \begin{equation*}
        \forall\, u, v \in V_1 : (u, v) \in E_1 \Rightarrow (h(u), h(v)) \in E_2
    \end{equation*}
    unter Annahme von $G_1 = (V_1, E_1)$ und $G_2 = (V_2, E_2)$.
    
    \item $h$ ist ein Graphenhomomorphismus. Für alle $x \in V_1$ gilt auch $h(x) = 1$, somit gilt auch immer $h(x) \in \{ 1 \}$.
    
    \item $h$ ist kein Graphenhomomorphismus. Es gibt keinen direkten Weg von der Kreuzung Graben/Kollegiumgasse (1) zur Kreuzung Kollegiumgasse/Pfarrplatz (2), die entsprechende Straße ist eine Einbahn in die entgegengesetzte Richtung.
\end{enumerate}

\paragraph{Aufgabe 2.}

$G_1$ und $G_3$ können nicht zu $G_2$ isomorph sein. $G_1$ und $G_3$ beinhalten beide einen Knoten (\texttt{1} bzw. \texttt{7}) mit zwei nach außen und keinen nach innen gerichteten Kanten. $G_2$ beinhaltet keinen solchen Knoten.

$G_1$ und $G_3$ sind zueinander isomorph, ein Isomorphismus $h$ ist etwa
\begin{center}
    \begin{tabular}{c | ccccccc}
        $v$    &1&2&3&4&5&6&7\\\hline
        $h(v)$ &7&4&1&2&5&3&6\\
    \end{tabular}\ .
\end{center}

\paragraph{Aufgabe 3.}

Es gibt einen bijektiven Graphenhomomorphismus $h: V_1 \rightarrow V_2$ (und einen Graphenhomomorphismus $h^{-1}: V_2 \rightarrow V_1$). Der Graph $U = (V_u, E_u)$ sei ein beliebiger Untergraph von $G_1$, somit gibt es einen injektiven Graphenhomomorphismus $t: V_u \rightarrow V_1$. Zu zeigen ist, dass es nun auch einen einen injektiven Graphenhomomorphismus $f: V_u \rightarrow V_2$ gibt.

Es gelte $f = h \circ t$. Diese Funktion ist injektiv (Skriptum, Satz 1.1, S. 15) und ein Graphenhomomorphismus von $V_u$ nach $V_2$ nachdem gilt, dass
\begin{equation*}
    \forall\, u, v \in V_u : (u, v) \in E_u \Rightarrow (t(u), t(v)) \in E_1
\end{equation*}
und
\begin{equation*}
    \forall\, u, v \in V_1 : (u, v) \in E_1 \Rightarrow (h(u), h(v)) \in E_2
\end{equation*}
beziehungsweise also
\begin{equation*}
    \forall\, u, v \in V_u : (u, v) \in E_u \Rightarrow (h(t(u)), h(t(v))) \in E_2.
\end{equation*}

\end{document}
