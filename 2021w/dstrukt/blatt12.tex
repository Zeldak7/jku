\documentclass{article}
\usepackage[utf8]{inputenc}
\usepackage[ngerman]{babel}

% Convenience improvements
\usepackage{csquotes}
\usepackage{enumitem}
\setlist[enumerate,1]{label={\alph*)}}
\usepackage{amsmath}
\usepackage{amssymb}
\usepackage{mathtools}

% Proper tables and centering for overfull ones
\usepackage{booktabs}
\usepackage{adjustbox}

% Change page/text dimensions, the package defaults work fine
\usepackage{geometry}

\usepackage{parskip}

% Drawings
\usepackage{tikz}

% Adjust header and footer
\usepackage{fancyhdr}
\pagestyle{fancy}
\fancyhead[L]{Diskrete Strukturen --- \textbf{Übungsblatt 12}}
\fancyhead[R]{Laurenz Weixlbaumer (11804751)}
\fancyfoot[C]{}
\fancyfoot[R]{\thepage}
% Stop fancyhdr complaints
\setlength{\headheight}{12.5pt}

\newcommand{\R}{\mathbb{R}\ \\\ \{0\}}

\newcommand{\cmod}{\text{mod}}

\usepackage{seqsplit}

\begin{document}

\paragraph{Aufgabe 1.}

Es gilt eine surjektive Funktion von $S_n$ in die Menge $P$ aller Partitionen von $\{1, \ldots, n\}$ zu konstruieren.

Beobachtung: Zerlegt man ein $\pi \in S_n$ in disjunkte Zyklen $z_1, \ldots, z_m$ und interpretiert man die Zyklen $z = (k_1\ \ldots\ k_j)$ als $\overline{z} = \{ k_1, \ldots, k_j \}$ so ergibt $\{ \overline{z}_1, \ldots, \overline{z}_m \}$ eine Menge von Mengen die $\in P$ ist. Beweis: Zu zeigen ist, dass f\"ur $\{ \overline{z}_1, \ldots, \overline{z}_m \}$ die Bedingungen einer Partition gelten. Nach der Definition von Zyklen gilt $\overline{z}_i \in \mathcal{P}(\{1, \ldots, n\})$, $\overline{z}_i \neq \emptyset$ und $\overline{z}_1 \cup \cdots \cup \overline{z}_m = \{ 1, \ldots, n \}$. Weiters gilt $\overline{z}_i \cap \overline{z}_j = \emptyset$ f\"ur $i \neq j$ nachdem die Zyklen disjunkt sind.

Die gesuchte Funktion geht gem\"aß der obigen Beobachtung vor. Diese Funktion ist surjektiv, f\"ur alle $p \in P$ gibt es ein $\pi \in S_n$ mit $f(\pi) = p$, nachdem f\"ur den Inhalt einer beliebigen Partition $p = \{ u_1, \ldots, u_k \} \in P$ und f\"ur (gem\"aß der obigen Ausf\"uhrungen uminterpretierten) disjunkte Zyklen gleiche Bedingungen gelten. Die Funktion ist allerdings nicht injektiv, nachdem zwei verschiedene Zyklen --- etwa $(1\ 2\ 3)$ und $(1\ 3\ 2)$ --- zur gleichen Menge uminterpretiert werden.

\paragraph{Aufgabe 2.}

K\"onnte kein Topf leer sein, so w\"urde es $\begin{Bsmallmatrix}n \\k\end{Bsmallmatrix}$ M\"oglichkeiten zur Aufteilung geben. Nachdem dies nicht der Fall ist, muss $\begin{Bsmallmatrix}n \\k\end{Bsmallmatrix}$ (keine leeren T\"opfe), $\begin{Bsmallmatrix}n \\k - 1\end{Bsmallmatrix}$ (ein leerer Topf), $\begin{Bsmallmatrix}n \\k - 2\end{Bsmallmatrix}$ (zwei leeren T\"opfe), \ldots ,$\begin{Bsmallmatrix}n \\k - k\end{Bsmallmatrix}$ ($k$ leere T\"opfe) berechnet und summiert werden.

\paragraph{Aufgabe 3.}

Wenn die Reihenfolge der B\"ucher irrelevant w\"are, so w\"urde $\begin{psmallmatrix}n + k - 1 \\ k - 1\end{psmallmatrix}$ die Anzahl der M\"oglich\-kei\-ten modellieren, sie zu platzieren. Man denke die B\"oden zu \enquote{Separatoren} um. Dann gibt es f\"ur $k$ B\"oden immer $k - 1$ Separatoren, weil der erste Separator zwischen dem ersten und zweiten (und nicht unter dem ersten) \enquote{Stock} liegt. Im Kontext des Binomialkoeffizienten gibt es dann insgesamt $n + k - 1$ m\"ogliche Positionen zwischen denen $k - 1$ Separatoren verschoben werden k\"onnen.

\begin{center}
    \begin{tabular}{c c c}
        n & k & $\begin{psmallmatrix}n + k - 1 \\ k - 1\end{psmallmatrix}$ \\
        1 & 1 & 1 \\
        1 & 2 & 2 \\
        1 & 3 & 3 \\
        1 & 4 & 4 \\
        \multicolumn{3}{c}{$\cdots$}
    \end{tabular}
    \quad
    \begin{tabular}{c c c}
        n & k & $\begin{psmallmatrix}n + k - 1 \\ k - 1\end{psmallmatrix}$ \\
        2 & 1 & 1 \\
        2 & 2 & 3 \\
        2 & 3 & 6 \\
        2 & 4 & 10 \\
        \multicolumn{3}{c}{$\cdots$}
    \end{tabular}
    \quad
    $\cdots$
\end{center}

Nachdem die Reihenfolge der B\"ucher relevant ist, muss dieser Wert noch um die Anzahl der m\"oglichen Permutationen der Menge der B\"ucher, $|S_n| = n!$, skaliert werden.

\end{document}

n! = |S_n|

anzahl der teilmengen von 1 ... n + k - 1 mit k - 1 elementen

