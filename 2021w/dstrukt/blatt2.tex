\documentclass{article}
\usepackage[utf8]{inputenc}
\usepackage[ngerman]{babel}

% Convenience improvements
\usepackage{csquotes}
\usepackage{enumitem}
\usepackage{amsmath}
\usepackage{amssymb}

% Proper tables and centering for overfull ones
\usepackage{booktabs}
\usepackage{adjustbox}

% Change page/text dimensions, the defaults work fine
\usepackage{geometry}

% Adjust header and footer
\usepackage{fancyhdr}
\pagestyle{fancy}
\fancyhead[L]{Diskrete Strukturen --- \textbf{Übungsblatt 2}}
\fancyhead[R]{Laurenz Weixlbaumer (11804751)}
\fancyfoot[C]{}
\fancyfoot[R]{\thepage}

% Specific to this particular exercise
\renewcommand{\thetable}{\alph{table}}
\usepackage{caption}
\DeclareCaptionLabelFormat{custom}{Truth Table (#2):}
\captionsetup{labelformat={custom}, labelsep=space}

\begin{document}

\paragraph{Aufgabe 1.}

% \begin{enumerate}[label=\alph*)]
%     \item Wenn $f(x) = x$, $g(x) = x$ und $h(x) = g^{-1}(x)$ dann gilt $h \circ g \circ f = id_A$.

%     \item Wenn $f(x) = a$ und $g(x) = b$ für $a, b \in C$ dann gilt der gegebene Ausdruck.

%     \item Für zwei Funktionen $f : A \rightarrow B$ und $g : B \rightarrow A$ gilt dann $g \circ f = id_A$ wenn $g = f^{-1}$. Demnach müsste für $f \circ f = id_A$ gelten, dass $f = f^{-1}$. Es gibt keine solche Funktion.
% \end{enumerate}

\begin{enumerate}[label=\alph*)]
    \item
    \begin{minipage}{0.3\textwidth}
        \begin{center}
            $f\ :\ A \rightarrow B$
            \vspace{0.1cm}

            \begin{tabular}{c | c | c | c}
                3 &  &  & $\times$ \\\hline
                2 &  & $\times$ &  \\\hline
                1 & $\times$ & &  \\\hline
                  & $\square$ & $\bigcirc$ & $\triangle$
            \end{tabular}
        \end{center}
    \end{minipage}
    \hfill
    \begin{minipage}{0.3\textwidth}
        \begin{center}
            $g\ :\ B \rightarrow C$
            \vspace{0.1cm}

            \begin{tabular}{c | c | c | c}
                a & $\times$ &   &   \\\hline
                b &   & $\times$ &   \\\hline
                c &   &   & $\times$ \\\hline
                  & 1 & 2 & 3 \\
            \end{tabular}
        \end{center}
    \end{minipage}
    \hfill
    \begin{minipage}{0.3\textwidth}
        \begin{center}
            $h\ :\ C \rightarrow A$
            \vspace{0.1cm}

            \begin{tabular}{c | c | c | c}
                $\triangle$ &   &   & $\times$ \\\hline
                $\bigcirc$ &   & $\times$ &   \\\hline
                $\square$ & $\times$ &   &   \\\hline
                  & a & b & c
            \end{tabular}
        \end{center}
    \end{minipage}
    
    \begin{equation*}
        h \circ g \circ f = id_A
    \end{equation*}
    
    \item Wenn $f(a) = c_1$ und $g(b) = c_2$ für $c_1 \neq c_2$ und beliebige $a \in A, b \in B$ und $c_1, c_2 \in C$ dann ist der Schnitt $\emptyset$.
    
    \item Für zwei Funktionen $f : A \rightarrow A$ und $g : A \rightarrow A$ gilt dann $g \circ f = id_A$ wenn $g = f^{-1}$. Demnach müsste für $f \circ f = id_A$ gelten, dass $f = f^{-1}$. Es gibt keine solche Funktion.

\end{enumerate}

\paragraph{Aufgabe 2.}

\begin{enumerate}[label=\alph*)]
    \item Injektiv und nicht surjektiv.
    \item Injektiv und surjektiv.
    \item Injektiv und nicht surjektiv.
    \item Weder injektiv noch surjektiv.
    \item Injektiv und surjektiv. (Unter der Annahme, dass der Bildbereich dieser Funktion die Menge aller derzeit vergebenen Matrikelnummern und nicht $\mathbb{N}$ o. Ä. ist.)
\end{enumerate}

\paragraph{Aufgabe 3.}

Durch den gegebenen Ausdruck wird eine Funktion definiert. Jedes $x$ hat mindestens einen Ausgabewert und kein $x$ führt zu mehr als einem Ausgabewert. Konkret für Fallunterscheidungen gilt, dass jedes $x$ von genau einem Fall abgedeckt werden muss.

\end{document}