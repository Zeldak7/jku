\documentclass[twocolumn]{article}

% Font
\usepackage[T2A]{fontenc}
% Language
\usepackage[ngerman]{babel}
% Encoding
\usepackage[utf8]{inputenc}

% Custom header/footer
\usepackage{fancyhdr}

% Make LaTeX better
\usepackage{enumitem}
\usepackage{booktabs}
\usepackage{csquotes}

\usepackage{parskip}

% Sizing
\usepackage{geometry}
\geometry{
    a4paper,
    total={170mm,257mm},
    left=20mm,
    top=20mm,
}

% The \term command is used to introduce new terminology. It should only be used when the term is first introduced.
\newcommand{\term}[1]{\textbf{#1}}

% Visually separate content. Can be used to highlight a certain paragraph or to signify a context break.
\newcommand{\separator}{\vspace{0.5em}\noindent}

\begin{document}

\section{Grundlagen}

\term{Diversität} beschreibt die Vielfalt von Individuen. Die Merkmale hierfür sind sozial konstruiert und umfassen all das, worin Menschen sich unterscheiden können. Es wird zwischen 
\begin{itemize}
    \item objektiven (etwa Geschlecht, Alter, Behinderungen) und
    \item subjektiven (etwa Erziehung, Religion, Lebensstil)
\end{itemize} 
Unterschieden differenziert.

Im Kontext Hochschule sind demographische, kognitive, fachliche, funktionale und institionelle Merkmale relevant.

\term{Stereotype} sind neutrale Erwartunen und Vorstellungen wie sich Mitglieder von Gruppen verhalten, wie sie aussehen oder welche Fähigkeiten sie haben.

\term{Vorurteile} sind mit Emotionen behaftet und basieren auf Stereotypen, sie umfassen eine negative oder positive Bewertung.

\term{Diskriminierung} ist eine Verhaltensreaktion auf stereotype Bewertungen, also auf Vorurteile.

\section{Informationsverarbeitung im Gehirn}

Informationen werden in Form von elektrischen und chemischen Signalen von einem Neuron zum anderen übertragen.

Der \term{präfontale Cortex} ist für die Erinnerung von Inhalten zuständig. Somit ist er an Einspeicherungsprozessen beteiligt, organisiert zu lernende Inhalte und ist eng mit dem Arbeitsgedächtnis verbunen. 

\term{Limbische Teile} in der Großhirnringe sind für die Handlungs- und Impulskontrolle zuständig. Sie verursachen bewusste Emotionen und sind für bewusste kognitive Leistungen zuständig.

Der \term{Hippocampus} ist der Organisator des (deklarativen) Gedächtnisses (Fakten, Vertrautheiten).

Die \term{Amygdala} ist für emotionale Konditionierung (vermittlung primitiver negativer/positiver Gefühle) zuständig.

Hormone und Neurotransmitter sind für Motivation, Interesse, Aufmerksamkeit und Lernfähigkeit zuständig:

\begin{center}
    \begin{tabular}{r l}
        Östrogen & Sprachbegabung \\
        Testosteron & Gedächtnis \\
        Noradrelanin & Aufmerksamkeit \\
        Dopamin & Antrieb, Neugier \\
        Glutamat & Konzentration \\
        Acetylcholin & gezielte Aufmerksamkeit \\
        Serotonin & Beruhigung \\
        Oxytozin & soziale Bindung
    \end{tabular}
\end{center}



\end{document}