\documentclass{article}
\usepackage[utf8]{inputenc}
\usepackage[ngerman]{babel}

% Convenience improvements
\usepackage{csquotes}
\usepackage{enumitem}
\setlist[enumerate,1]{label={\alph*)}}
\usepackage{amsmath}
\usepackage{amssymb}
\usepackage{mathtools}
\usepackage{tabularx}

% Proper tables and centering for overfull ones
\usepackage{booktabs}
\usepackage{adjustbox}

% Change page/text dimensions, the package defaults work fine
\usepackage{geometry}

\usepackage{parskip}

% Drawings
\usepackage{tikz}

% Adjust header and footer
\usepackage{fancyhdr}
\pagestyle{fancy}
\fancyhead[L]{Algebra --- \textbf{Exercise Sheet 2}}
\fancyhead[R]{Laurenz Weixlbaumer (11804751)}
\fancyfoot[C]{}
\fancyfoot[R]{\thepage}
% Stop fancyhdr complaints
\setlength{\headheight}{12.5pt}

\newcommand{\Deltaop}{\, \Delta\, }
\newcommand{\xor}{\, \oplus\, }

\begin{document}

\paragraph{Exercise 1}

\begin{enumerate}
    \item We have
    \begin{center}
        \begin{tabular}{c | c c | c}
            $r$ & $x$ & $y$ & $q$ \\\midrule
            135 & 1 & 0 & \\
            54 & 0 & 1 & 2 \\
            27 & 1 & -2 & 2 \\ 
            0 & -2 & 5 & \\ 
        \end{tabular}
    \end{center}
    and thus $\gcd(135, 54) = 27$. Since $27 \mid 0$, there is a solution.

    From the last row we know that $125 \cdot -2 + 54 \cdot 5 = 0$, thus $(-2, 5) \in L$. We can now describe $L$ as $L = \{(-2k, 5k) \mid k \in \mathbb{Z}\}$.

    \item We have (note that $x$ and $y$ are reversed)
    \begin{center}
        \begin{tabular}{c | c c | c}
            $r$ & $y$ & $x$ & $q$ \\\midrule
            105 & 1 & 0 & \\
            99 & 0 & 1 & 1 \\
            6 & 1 & -1 & 16 \\ 
            3 & -16 & 17 & 2 \\ 
            0 & 33 & -35 & \\ 
        \end{tabular}
    \end{center}
    and thus $\gcd(105, 99) = 3$. Since $3 \mid 12$, there is a solution.
    
    From the second to last row we know
    \begin{align*}
        3  &= (17 \cdot 99) + (-16 \cdot 105) \quad \text{and, after multiplying by 4}\\
        12 &= (68 \cdot 99) + (-64 \cdot 105).
    \end{align*}

    Thus we have that $(68, -64) \in L$ and further $L = \{(68 - 35k, -64 + 33k) \mid k \in \mathbb{Z}\}$.

    \item We have
    \begin{center}
        \begin{tabular}{c | c c | c}
            $q$ & $x$ & $y$ & $r$ \\\midrule
            38 & 1 & 0 & \\
            19 & 0 & 1 & 2 \\
            0 & 1 & -2 &  \\
        \end{tabular}
    \end{center}
    and thus $\gcd(38, 19) = \gcd(19, -38) = 19$. Since $19 \nmid 5$ this equation does not have a solution.
\end{enumerate}

\pagebreak
\paragraph{Exercise 2}

We are looking for solutions to
\begin{align*}
    35x + 45y = 1000
\end{align*}
where $x$ is the number of linear Algebra books and $y$ is the number of Analysis books. We have (note that $x$ and $y$ are reversed)
\begin{center}
    \begin{tabular}{c | c c | c}
        $r$ & $y$ & $x$ & $q$ \\\midrule
        45 & 1 & 0 & \\
        35 & 0 & 1 & 1 \\
        10 & 1 & -1 & 3 \\ 
        5 & -3 & 4 & 2 \\ 
        0 & 7 & -9 & \\ 
    \end{tabular}
\end{center}
and thus $\gcd(45, 35) = 5$. Since $5 \mid 1000$, there is a solution.

From the second to last row we know
\begin{align*}
    5  &=  (4 \cdot 35) + (-3 \cdot 45) \quad \text{and, after multiplying by 200}\\
    1000 &=  (800 \cdot 35) + (-600 \cdot 45)
\end{align*}
and thus $(800, -600) \in L$, allowing us to state $L = \{(800 \cdot -9k, -600 \cdot 7k) \mid k \in \mathbb{Z}\}$. For $86 \leq k \leq 88$ neither of the values in the pairs $\in L$ are negative. Thus we can either buy 
\begin{center}
    \begin{tabular}{c c}
        Lineare Algebra & Analysis \\
        26 & 2 \\
        17 & 9 \\
        8 & 16 \\
    \end{tabular}
\end{center}
books.

If the total available money were 1001 then we would have $5 \nmid 1000$, thus we would not be able to spend all of our budget.

\pagebreak
\paragraph{Exercise 3}

Idk.

\pagebreak
\paragraph{Exercise 4}

Interpreting the polynomials as being in $\mathbb{Z}_5$.
\begin{center}
    \begin{tabular}{l l l l l l l l l l l l l l l}
        $x^5$&$x^4$&$x^3$&$x^2$&$x^1$&$x^0$&&$x^3$&$x^2$&$x^1$&$x^0$&&$x^2$&$x^1$&$x^0$ \\
        3&1&4&1&0&4 &:& 2&2&1&3 &=& 4&4&1 \\
        3&3&4&2&&&&&&&&&&& \\\cmidrule{1-4}
        &3&0&4&0&4&&&&&&&&& \\
        &3&3&4&2&&&&&&&&&& \\\cmidrule{2-5}
        &&2&0&3&4&&&&&&& \\
        &&2&2&1&3&&&&&&& \\\cmidrule{3-6}
        &&&3&2&1&&&&&&& \\
    \end{tabular}
\end{center}

Interpreting the polynomials as being in $\mathbb{Q}$.

\begin{center}
    \begin{tabular}{l l l l l l l l l l l l l l l}
        $x^5$&$x^4$&$x^3$&$x^2$&$x^1$&$x^0$&&$x^3$&$x^2$&$x^1$&$x^0$&&$x^2$&$x^1$&$x^0$ \\
        3&1&4&1&5&9 &:& 2&7&1&8 &=& 1.5&-4.75&17.875 \\
        3&10.5&1.5&12&&&&&&&&&&& \\\cmidrule{1-4}
        &-9.5&3.5&-11&5&9&&&&&&&&& \\
        &-9.5&-33.25&-4.75&-38&&&&&&&&&& \\\cmidrule{2-5}
        &&35.75&-6.25&43&9&&&&&&& \\
        &&35.75&125.12&17.975&143&&&&&&& \\\cmidrule{3-6}
        &&&-132.37&25.125&-134&&&&&&& \\
    \end{tabular}
\end{center}

\pagebreak
\paragraph{Exercise 5}

\begin{enumerate}
    \item
    Consider that a general table for the GCD is 
    \begin{center}
        \begin{tabular}{c | c c | c}
            $r$ & $u$ & $v$ & $q$ \\\midrule
            $P_1$ & 1 & 0 & \\
            $P_2$ & 0 & 1 & $q_1$ \\
            $r_1$ & 1 & $v_1$ & $q_2$ \\
            $r_2$ & $u_2$ & $v_2$ & $q_3$ \\
            $r_3$ & $u_3$ & $v_3$ & $q_4$ \\
        \end{tabular}
    \end{center}

    We begin by calculating $q_1$ and $r_1$.

    \begin{center}
        \begin{tabular}{l l l l l l l l l l l l l l l}
            $x^5$&$x^4$&$x^3$&$x^2$&$x^1$&$x^0$&&$x^4$&$x^3$&$x^2$&$x^1$&$x^0$&&$x^1$&$x^0$ \\
            1&6&9&-6&-22&-12& :& 1&1&-4&-2&4& =& 1&5 \\
            1&1&-4&-2&4&&&&&&&&& \\\cmidrule{1-5}
            &5&13&-4&-26&-12&&&&&&&&& \\
            &5&5&-20&-10&20&&&&&&&&& \\\cmidrule{2-6}
            &&8&16&-16&-32&&&&&&&&& \\
        \end{tabular}
    \end{center}

    Thus $q_1 = x + 5$, $r_1 = 8x^3 + 16x^2 - 16x - 32$ and $v_1 = 0 - q_1 = -x - 5$.

    We continue by calculating $q_2$ and $r_2$.

    \begin{center}
        \begin{tabular}{ l l l l l l l l l l l l l}
            $x^4$&$x^3$&$x^2$&$x^1$&$x^0$&&$x^3$&$x^2$&$x^1$&$x^0$&&$x^1$&$x^0$ \\
            1&1&-4&-2&4& :& 8&16&-16&-32& =& $\frac{1}{8}$&$-\frac{1}{8}$ \\
            1&2&-2&-4&&&&&&&& \\\cmidrule{1-4}
            &-1&-2&2&4&&&&&&& \\
            &-1&-2&2&4&&&&&&&& \\\cmidrule{2-5}
            &0&0&0&0&&&&&&&& \\
        \end{tabular}
    \end{center}

    Thus $q_2 = \frac{1}{8}x - \frac{1}{8}$ and $r_2 = 0$. We have $\gcd(P_1, P_2) = r_1 = 8x^3 + 16x^2 - 16x - 32$.

    \item If $\frac{a}{b}$ is a root of $\gcd(P_1, P_2)$ then $a$ must be a divisor of $32$ and $b$ must be a divisor of $8$. The candidates are thus 
    \begin{align*}
        \pm1&&\boxed{-2}&&\pm4&&\pm8&&\pm16&&\pm32&&\pm\frac{1}{2}&&\pm\frac{1}{4}&&\pm\frac{1}{8}
    \end{align*}
    where boxed numbers are actual roots. We can thus factor out $x + 2$ by division.

    \begin{center}
        \begin{tabular}{cccc c cc c ccc}
            $x^3$&$x^2$&$x^1$&$x^0$&&$x^1$&$x^0$&&$x^2$&$x^1$&$x^0$ \\
            8&16&-16&-32&:&1&2&=&8&0&-16\\
            8&16\\\cmidrule{1-2}
            &&-16&-32\\
            &&-16&-32\\\cmidrule{3-4}
        \end{tabular}
    \end{center}

    We can now solve $8x^2 - 16 = 0$ through
    \begin{align*}
        x_1,x_2 = \frac{-b \pm \sqrt{b^2 - 4ac}}{2a} =
        \pm\frac{\sqrt{-4 \cdot 8 \cdot -16}}{2 \cdot 8} =
        \pm\frac{\sqrt{512}}{16} =
        \pm\frac{16\sqrt{2}}{16} = \pm \sqrt{2}.
    \end{align*}
    The roots are thus $-2, -\sqrt{2}$ and $\sqrt{2}$.
\end{enumerate}

\pagebreak
\paragraph{Exercise 6}

We are looking for the rational roots of
\begin{align*}
    P(x) = 18 x^6 - 51 x^5 - 7 x^4 + 106 x^3 - 62 x^2 - 8 x + 8
\end{align*}

Using the fact that, if $\frac{a}{b}$ is a root of a polynom then $a \mid a_0$ and $b \mid a_n$, we get
\begin{align*}
    \pm1&&\boxed{2}&&\pm4&&\pm8&&\boxed{\frac{1}{2}}&&\boxed{-\frac{1}{3}}&&\pm\frac{1}{6}&&\pm\frac{1}{9}&&\pm\frac{1}{18}&&\boxed{\frac{2}{3}}&&\pm\frac{2}{9}&&\pm\frac{2}{18}&&\pm\frac{4}{3}&&\pm\frac{4}{9}&&\pm\frac{8}{3}&&\pm\frac{8}{9}
\end{align*}
as potential roots. Boxed numbers are actual roots.

We can thus factor out
\begin{align*}
    \left(x + \frac{1}{3}\right)\left(x - \frac{1}{2}\right)\left(x - \frac{2}{3}\right)(x - 2) = x^4-\frac{17}{6}x^3+\frac{29}{18}x^2+\frac{2}{9}x-\frac{2}{9}
\end{align*}
by division.

\begin{center}
    \begin{tabular}{ccccccc c ccccc c ccc}
        $x^6$&$x^5$&$x^4$&$x^3$&$x^2$&$x^1$&$x^0$&&$x^4$&$x^3$&$x^2$&$x^1$&$x^0$&&$x^2$&$x^1$&$x^0$ \\
        18&-51&-7&106&-62&-8&8&:&1&$-\frac{17}{6}$&$\frac{29}{18}$&$\frac{2}{9}$&$-\frac{2}{9}$&=&18&0&-36 \\
        18&-51&29&4&-4&&&&&&&&&&&& \\\cmidrule{1-5}
        &&-36&102&-58&-8&8&&&&&&&&& \\
        &&-36&102&-58&-8&8\\\cmidrule{3-7}
    \end{tabular}
\end{center}

We can now solve $18x^2 - 36 = 0$ through
\begin{align*}
    x_1,x_2 = \frac{-b \pm \sqrt{b^2 - 4ac}}{2a} =
    \pm\frac{\sqrt{-4 \cdot 18 \cdot -36}}{2 \cdot 18} =
    \pm\frac{\sqrt{2592}}{36} =
    \pm\frac{36\sqrt{2}}{36} = \pm \sqrt{2},
\end{align*}
yielding no rational roots. The rational roots are thus these obtained previously.

\pagebreak
\paragraph{Exercise 7}

We have
\begin{align*}
    p(x) &= x^7 - 6x^6 + 10x^5 - 6x^4 + 9x^3 \\
    p'(x) &= 7x^6 - 36x^5 + 50x^4 - 24x^3 + 27x^2
\end{align*}
and we are looking for a square-free factorisation of $p$. Calculating the GCD of $p$ and $p'$ we first divide $p$ by $p'$
\begin{center}
    \begin{tabular}{ccccccc c cccc c cc}
        $x^7$&$x^6$&$x^5$&$x^4$&$x^3$&$x^2$&&$x^6$&$x^5$&$x^4$&$x^3$&$x^2$&&$x^1$&$x^0$ \\
        1&-6&10&-6&9&0&:&7&-36&50&-24&27&=&$\frac{1}{7}$&$-\frac{6}{49}$\\
        1&$-\frac{36}{7}$&$\frac{50}{7}$&$-\frac{24}{7}$&$\frac{27}{7}$ \\\cmidrule{1-5}
        &$-\frac{6}{7}$&$\frac{20}{7}$&$-\frac{17}{7}$&$\frac{36}{7}$&0\\
        &$-\frac{6}{7}$&$\frac{216}{49}$&$-\frac{300}{49}$&$\frac{144}{49}$&$-\frac{162}{49}$ \\\cmidrule{2-6}
        &&$-\frac{76}{49}$&$\frac{174}{49}$&$\frac{108}{49}$&$\frac{162}{49}$ \\
    \end{tabular}
\end{center}
and get $r_1 = -\frac{76 x^5}{49}+\frac{174 x^4}{49}+\frac{108 x^3}{49}+\frac{162 x^2}{49}$ and $q_1 = \frac{x}{7}-\frac{6}{49}$. We can simplify $r_1 = -38 x^5+87 x^4+54 x^3+81 x^2$.

Now we divide $p'$ by $r_1$
\begin{center}
    \begin{tabular}{lllll l llll l ll}
        $x^6$&$x^5$&$x^4$&$x^3$&$x^2$ && $x^5$&$x^4$&$x^3$&$x^2$ && $x^1$&$x^0$ \\
        7&-36&50&-24&27 &:& $-38$&$87$&$54$&$81$ &=& $-\frac{7}{38}$&$\frac{759}{1444}$ \\
        7&$\frac{126}{19}$&$\frac{175}{19}$&$\frac{84}{19}$&$-\frac{189}{38}$\\\cmidrule{1-5}
        &$-\frac{810}{19}$&$\frac{775}{19}$&$-\frac{540}{19}$&$\frac{1215}{38}$\\
        &$-\frac{810}{19}$&$\frac{66033}{1444}$&$\frac{20493}{722}$&$\frac{61479}{1444}$\\\cmidrule{2-6}
        &&$\frac{20531 x^4}{1444}$&$-\frac{13524 x^3}{361}$&$-\frac{22491 x^2}{1444}$
    \end{tabular}
\end{center}
and get $r_2 = \frac{20531 x^4}{1444}-\frac{13524 x^3}{361}-\frac{22491 x^2}{1444}$. We then divide $r_1$ by $r_2$ and get $r_3 = \frac{11696400 x^2}{175561}-\frac{3898800 x^3}{175561}$. We then divide $r_2$ by $r_3$ and get $r_4 = 0$.

Thus $r_3$, which can be simplified to $x^3 - 3x^2$, is the GCD we are looking for.

We now divide $p$ by this result which yields $x^4-3 x^3+x^2-3 x$ with no remainder.

\pagebreak
\paragraph{Exercise 8}

To show that $p \mid \left(\begin{smallmatrix}
    p \\k
\end{smallmatrix}\right)$ note that
\begin{align*}
    \begin{pmatrix*}
        p \\k
    \end{pmatrix*} &= \frac{p!}{k!(p - k)!} \\
    p! &= \begin{pmatrix*}
        p \\k
    \end{pmatrix*}(k!(p - k)!).
\end{align*}
Since the left hand side of the equation is clearly divisible by $p$, the right hand side must also be divisible by it. The expression $k!(p - k)!$ is not divisible by $p$ since it is a product of numbers smaller than $p$ and $p$ is prime. Thus the binomial coefficient must be the part which is divisible by $p$.

Now consider that, by the binomial theorem
\begin{align*}
    (x + y)^p &= \sum^{p}_{k = 0} \begin{pmatrix*}
        p \\k
    \end{pmatrix*}x^{p - k}y^{k} \\
    (x + y)^p &= \begin{pmatrix*}
        p \\0
    \end{pmatrix*}x^p y^0 + 
    \begin{pmatrix*}
        p \\1
    \end{pmatrix*}x^{p-1} y^1 + \cdots +
    \begin{pmatrix*}
        p \\p - 1
    \end{pmatrix*}
    x^1 y^{p - 1} + 
    \begin{pmatrix*}
        p \\p
    \end{pmatrix*}
    x^0 y^p \\
    (x + y)^p &= x^p + \begin{pmatrix*}
        p \\1
    \end{pmatrix*}x^{p-1} y^1 + \cdots +
    \begin{pmatrix*}
        p \\p - 1
    \end{pmatrix*}
    x^1 y^{p - 1} + y^p \\
    (x + y)^p &\equiv_6 x^p + y^p
\end{align*}

\end{document}
