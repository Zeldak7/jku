\documentclass{article}
\usepackage[utf8]{inputenc}
\usepackage[ngerman]{babel}

% Convenience improvements
\usepackage{csquotes}
\usepackage{enumitem}
\setlist[enumerate,1]{label={\alph*)}}
\usepackage{amsmath}
\usepackage{amssymb}
\usepackage{mathtools}
\usepackage{tabularx}

% Proper tables and centering for overfull ones
\usepackage{booktabs}
\usepackage{adjustbox}

% Change page/text dimensions, the package defaults work fine
\usepackage{geometry}

\usepackage{parskip}

% Drawings
\usepackage{tikz}
\usepackage{pgfplots}

% Adjust header and footer
\usepackage{fancyhdr}
\pagestyle{fancy}
\fancyhead[L]{Algebra --- \textbf{Exercise Sheet 11}}
\fancyhead[R]{Laurenz Weixlbaumer (11804751)}
\fancyfoot[C]{}
\fancyfoot[R]{\thepage}
% Stop fancyhdr complaints
\setlength{\headheight}{12.5pt}

\newcommand{\Deltaop}{\, \Delta\, }
\newcommand{\xor}{\, \oplus\, }
\newcommand{\id}{\text{id}}
\newcommand{\proj}{\text{proj}}

\begin{document}

\paragraph{Exercise 1}

\begin{enumerate}
    \item Linear.
    \begin{align*}
        f(\lambda u + \mu v) &= \lambda f(u) + \mu f(v) \\
        f\left(\begin{pmatrix}
            \lambda u_1 + \mu v_1 \\
            \lambda u_2 + \mu v_2 \\
            \lambda u_3 + \mu v_3 \\
        \end{pmatrix}\right) &= \begin{pmatrix}
            \lambda(u_1 + 3u_2 + 4u_3) \\
            \lambda u_3
        \end{pmatrix}
        +
        \begin{pmatrix}
            \mu(v_1 + 3v_2 + 4v_3) \\
            \mu v_3
        \end{pmatrix} \\
        \begin{pmatrix}
            \lambda u_1 + \mu v_1 + 3(\lambda u_2 + \mu v_2) + 4(\lambda u_3 + \mu v_3) \\
            \lambda u_3 + \mu v_3 \\
        \end{pmatrix} &= \begin{pmatrix}
            \lambda(u_1 + 3u_2 + 4u_3) + \mu(v_1 + 3v_2 + 4v_3) \\
            \lambda u_3 + \mu v_3
        \end{pmatrix}
    \end{align*}

    \item Not linear. The first equation produces $u_1v_2$ and $v_1u_2$ in the first row, which has no chance of happening in the second equation; they are not equal.
    \begin{align*}
        f(u + v) &= f\left(\begin{pmatrix}
            u_1 + v_1 \\
            u_2 + v_2 \\
            u_3 + v_3 \\
        \end{pmatrix}\right) = \begin{pmatrix}
            u_1 + v_1 + 2(u_2 + v_2) + (u_1 + v_1)(u_2 + v_2) + u_3 + v_3 \\
            u_1 + v_1 + u_2 + v_2 + u_3 + v_3 \\
        \end{pmatrix} \\
    \end{align*}
    \begin{align*}
        f(u) + f(v) = \begin{pmatrix}
            u_1 + 2u_2 + u_1u_2 + u_3 \\
            u_1 + u_2 + u_3 \\
        \end{pmatrix} + \begin{pmatrix}
            v_1 + 2v_2 + v_1v_2 + v_3 \\
            v_1 + v_2 + v_3 \\
        \end{pmatrix} \\ = \begin{pmatrix}
            u_1 + 2u_2 + u_1u_2 + u_3 + v_1 + 2v_2 + v_1v_2 + v_3 \\
            u_1 + u_2 + u_3 + v_1 + v_2 + v_3
        \end{pmatrix}
    \end{align*}

    \item Not linear because $\lambda f(u) + \mu f(v)$ will necessarily have $\lambda + \mu$ in it. The statements are not equal. \begin{align*}
        f(\lambda u + \mu v) =
        f\left(\begin{pmatrix}
            \lambda u_1 + \mu v_1 \\
            \lambda u_2 + \mu v_2 \\
            \lambda u_3 + \mu v_3 \\
        \end{pmatrix}\right) =
        \lambda u_1 + \mu v_1 + \lambda u_2 + \mu v_2 + \lambda u_3 + \mu v_3 + 1
    \end{align*}

    \item Not linear. \begin{align*}
        f(\lambda u + \mu v) = f\left(\begin{pmatrix}
            \lambda u_1 + \mu v_1 \\
            \lambda u_2 + \mu v_2 \\
        \end{pmatrix}\right) = \begin{pmatrix}
            \lambda u_2 + \mu v_2 \\
            \lambda u_1 + \mu v_1 \\
        \end{pmatrix} = \lambda\begin{pmatrix}
            u_2 \\
            u_1 \\
        \end{pmatrix} + \mu\begin{pmatrix}
            v_2 \\
            v_1 \\
        \end{pmatrix} \neq \lambda\begin{pmatrix}
            u_1 \\
            u_2 \\
        \end{pmatrix} + \mu\begin{pmatrix}
            v_1 \\
            v_2 \\
        \end{pmatrix}
    \end{align*}
\end{enumerate}

\paragraph{Exercise 3}

\begin{enumerate}
    \item \begin{align*}
        \begin{pmatrix}
            1 \\ -1 \\
        \end{pmatrix}
    \end{align*}

    \item 
\end{enumerate}

\paragraph{Exercise 4}

\begin{enumerate}
    \item The vectors in both B and C are linearly independent and thus both form a base of $R^2$.
    \begin{center}
        \begin{tabular}{c c | c}
            1 & -1 & 0 \\
            1 & 1 & 0 \\
        \end{tabular}\quad
        \begin{tabular}{c c | c}
            1 & -1 & 0 \\
            0 & 2 & 0 \\
        \end{tabular}\qquad and \qquad \begin{tabular}{c c | c}
            2 & 3 & 0 \\
            1 & -2 & 0 \\
        \end{tabular}\quad
        \begin{tabular}{c c | c}
            2 & 3 & 0 \\
            0 & -3.5 & 0 \\
        \end{tabular}
    \end{center}
    
    \item Since
    \begin{center}
        \begin{tabular}{c c | c}
            2 & 3 & 1 \\
            1 & -2 & 1 \\
        \end{tabular}\quad
        \begin{tabular}{c c | c}
            2 & 3 & 1 \\
            0 & $-\frac{7}{2}$ & $\frac{1}{2}$ \\
        \end{tabular} \quad with $x_1 = \frac{5}{7}$ and $x_2 = -\frac{1}{7}$
    \end{center}
    \begin{center}
        \begin{tabular}{c c | c}
            2 & 3 & -1 \\
            1 & -2 & 1 \\
        \end{tabular}\quad
        \begin{tabular}{c c | c}
            2 & 3 & -1 \\
            0 & $-\frac{7}{2}$ & $\frac{3}{2}$ \\
        \end{tabular} \quad with $x_1 = \frac{1}{7}$ and $x_2 = -\frac{3}{7}$
    \end{center}
    we have
    \begin{align*}
        A_C^B = \begin{pmatrix}
            \frac{5}{7} & \frac{1}{7} \\
            -\frac{1}{7} & -\frac{3}{7} \\
        \end{pmatrix}.
    \end{align*}

    \item \begin{align*}\begin{pmatrix}
        \frac{3}{2} & \frac{1}{2} \\
        -\frac{1}{2} & -\frac{5}{2} \\
    \end{pmatrix}\end{align*}

    \item \phantom{} \begin{center}
        \begin{tabular}{c c | c}
            1 & -1 & 2 \\
            1 & 1 & 1 \\
        \end{tabular} \quad \text{and} \begin{tabular}{c c | c}
            1 & -1 & 3 \\
            1 & 1 & -2 \\
        \end{tabular}
    \end{center}

    \begin{align*}
        \begin{pmatrix}
            \frac{1}{7} & -\frac{3}{7} \\
            \frac{1}{2} & -\frac{5}{2} \\
        \end{pmatrix}
    \end{align*}
\end{enumerate}

\end{document}
