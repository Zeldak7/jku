\documentclass{article}
\usepackage[utf8]{inputenc}
\usepackage[ngerman]{babel}

\usepackage{multicol}

% Convenience improvements
\usepackage{csquotes}
\usepackage{enumitem}
\setlist[enumerate,1]{label={\alph*)}}
\usepackage{amsmath}
\usepackage{amssymb}
\usepackage{mathtools}
\usepackage{tabularx}

% Proper tables and centering for overfull ones
\usepackage{booktabs}
\usepackage{adjustbox}

% Change page/text dimensions, the package defaults work fine
\usepackage{geometry}

\usepackage{parskip}

% Adjust header and footer
\usepackage{fancyhdr}

\newcommand{\Deltaop}{\, \Delta\, }
\newcommand{\xor}{\, \oplus\, }
\newcommand{\id}{\text{id}}
\newcommand{\proj}{\text{proj}}

\begin{document}

\begin{multicols}{2}

\section*{Vektoren}

\begin{align*}
    \begin{pmatrix*}
        u_1 \\
        u_2 \\
        u_3
    \end{pmatrix*} \times \begin{pmatrix*}
        v_1 \\
        v_2 \\
        v_3
    \end{pmatrix*} = \begin{pmatrix*}
        (u_2v_3) - (u_3v_2) \\
        (u_3v_1) - (u_1v_3) \\
        (u_1v_2) - (v_2u_1) 
    \end{pmatrix*}
\end{align*}

Die Länge eines Vektors ist $\sqrt{v_1^2 + \ldots v_n^2}$.

Der Vektor $\vec{n} = \vec{u} \times \vec{v}$ ist normal auf die Vektoren $\vec{u}$ und $\vec{v}$.

Die Normalenform einer Ebene $E$ in $\mathbb{R}^3$ ist
\begin{align*}
    E = \{\ \vec{x} \in \mathbb{R}^3\ |\ (\vec{x} - \vec{a})\vec{n} = 0\ \}
\end{align*}
mit $\vec{n}$ normal auf $\vec{a}$.

Von Gleichungsform $ax + by + cz = d$ einer Ebene auf Normalenform: Normalvektor ist $(a\ b\ c)$, dann $\vec{a}$ finden mit $\vec{a}\vec{n} = d$.

Orthogonalisierung einer Basis $\{v_1, \ldots, v_n\}$:
\begin{enumerate}[label={\arabic*)}]
    \item $w_1 = v_1$, dann für $i = 2, \ldots, n$ 
    \item $w_i = v_i - (\text{proj}_{w_1}(v_i) + \ldots + \text{proj}_{w_{i - 1}}(v_i))$
\end{enumerate}
mit
\begin{align*}
    \text{proj}_u(v) = \frac{\langle u, v\rangle}{\langle u, u\rangle} \cdot u
\end{align*}

\section*{Matrizen}

Eine (quadr.) Matrix $A$ ist \emph{singulär} wenn $\text{det}(A) = 0$, also wenn sie nicht invertierbar ist, bzw. ihre Spalten linear abhängig sind. Anderfalls ist sie \emph{regulär}.

\begin{align*}
    \text{det}\left(\begin{pmatrix*}
        a_{1,1} & a_{1,2} \\
        a_{2,1} & a_{2,2}
    \end{pmatrix*}\right) &= a_{1,1}a_{2,2} - a_{1,2}a_{2,1}
    % \text{det}\left(\begin{pmatrix*}
    %     a_{1,1} & a_{1,2} & a_{1,3} \\
    %     a_{2,1} & a_{2,2} & a_{2,3} \\
    %     a_{3,1} & a_{3,2} & a_{3,3}
    % \end{pmatrix*}\right) &= a_{1,1}a_{2,2}a_{3,3} + a_{1,2}a_{2,3}a_{3,1} + a_{1,3}a_{2,1}a_{2,3} - a_{1,3}a_{2,2}a_{3,1} - a_{1,1}a_{2,3}a_{3,2} - a_{1,2}a_{2,1}a_{3,3}
\end{align*}

Eine Basistransformationsmatrix $A_C^B$ (\enquote{B ausgedr. durch C}) ist
\begin{align*}
    A_C^B = \left((v_1)_C\ \cdots\ (v_n)_C\right)
\end{align*}
mit $B = \{ v_1, \ldots, v_n \}$.

Eigenwerte einer Matrix $A$ sind Nullstellen von
\begin{align*}
    \chi_A(\lambda) = \text{det}(A - \lambda E_n).
\end{align*}
Algebraische Vielfachheit ist die Potenz der Nullstelle. Dimension des Eigenraums ist geometrische Vielfachheit. Es gilt alg. V. $\geq$ geo. V.

Der Eigenraum zu einem Eigenwert $\lambda$ ist
\begin{align*}
    E_{A,\lambda} = \text{Ker}(A - \lambda E_n) \\
    \{ \vec{v}\ |\ (A - \lambda E_n) = \vec{0} \}
\end{align*}

Zur Diagonalisierung ($\lambda$ sind Eigenwerte, $v$ sind Eigenräume):
\begin{align*}
    P^{-1} \cdot A \cdot P &= \text{diag}(\lambda_1, \ldots, \lambda_n) \quad \text{mit} \\ 
    P &= (v_1, \ldots v_n)
\end{align*}
Nur möglich wenn für alle Werte alg. V. = geo V.

Zeilen- und Spaltenraum ist die lineare Hülle der Zeilen- bzw. Spaltenvektoren. (Auf Dimension aufpassen!)

\section*{Strukturen}

$\mathbb{Z}_n$ ist dann ein Körper wenn $n$ prim ist.

\section*{Polynome}

Sei $p = a_nx^n + \cdots + a_1x + a_0$. Wenn $p$ eine Nullstelle $\frac{a}{b} \in \mathbb{Q}$ besitzt dann $a\,|\,a_0$ und $b\,|\,a_n$. (Bei Polynom in $\mathbb{Z}_n$ ist oft ausprobieren aller möglichen Nullstellen einfacher.) 

\end{multicols}

\end{document}
