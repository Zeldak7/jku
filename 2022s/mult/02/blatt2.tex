\documentclass{article}
\usepackage[utf8]{inputenc}
\usepackage[ngerman]{babel}

% Convenience improvements
\usepackage{csquotes}
\usepackage{enumitem}
\setlist[enumerate,1]{label={\alph*)}}
\usepackage{amsmath}
\usepackage{amssymb}
\usepackage{mathtools}
\usepackage{tabularx}

\usepackage{csquotes}

% Proper tables and centering for overfull ones
\usepackage{booktabs}
\usepackage{adjustbox}

% Change page/text dimensions, the package defaults work fine
\usepackage{geometry}

\usepackage{parskip}

% Drawings
\usepackage{tikz}
\usepackage{forest}

% Adjust header and footer
\usepackage{fancyhdr}
\pagestyle{fancy}
\fancyhead[L]{Multimedia --- \textbf{Blatt 2}}
\fancyhead[R]{Laurenz Weixlbaumer (11804751)}
\fancyfoot[C]{}
\fancyfoot[R]{\thepage}
% Stop fancyhdr complaints
\setlength{\headheight}{12.5pt}

\newcommand{\Deltaop}{\, \Delta\, }
\newcommand{\xor}{\, \oplus\, }

\begin{document}

\paragraph{Convolution Filter}

Ein Convolution Filter wird für die (künstliche bzw. nachträgliche) Reduktion oder Erhöhung der Schärfe von Bildern verwendet. Das Eingabebild wird als Matrix von Farb- und Helligkeitswerten interpretiert. Ein sog. Kernel, eine benutzerdefinierte kleine Matrix, wird schrittweise \enquote{über} das Bild geschoben. Jeder Pixel nimmt einen Wert in Abhängigkeit von benachbarten Pixeln an, skaliert durch die Werte des Kernels.

\end{document}
