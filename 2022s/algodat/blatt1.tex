\documentclass{article}
\usepackage[utf8]{inputenc}
\usepackage[ngerman]{babel}

% Convenience improvements
\usepackage{csquotes}
\usepackage{enumitem}
\setlist[enumerate,1]{label={\alph*)}}
\usepackage{amsmath}
\usepackage{amssymb}
\usepackage{mathtools}
\usepackage{tabularx}

% Proper tables and centering for overfull ones
\usepackage{booktabs}
\usepackage{adjustbox}

% Change page/text dimensions, the package defaults work fine
\usepackage{geometry}

\usepackage{parskip}

% Drawings
\usepackage{tikz}

% Adjust header and footer
\usepackage{fancyhdr}
\pagestyle{fancy}
\fancyhead[L]{Algorithmen und Datenstrukturen --- \textbf{Blatt 1}}
\fancyhead[R]{Laurenz Weixlbaumer (11804751)}
\fancyfoot[C]{}
\fancyfoot[R]{\thepage}
% Stop fancyhdr complaints
\setlength{\headheight}{12.5pt}

\newcommand{\Deltaop}{\, \Delta\, }
\newcommand{\xor}{\, \oplus\, }

\begin{document}

\paragraph{Aufgabe 1.}

\begin{enumerate}
    \item Wir haben \(T(n) = 3T(\frac{n}{3}) + n\) mit \(T(1) = 1\). Wiederholtes Einsetzen
    \begin{align*}
        T(n) &= 3 T\left(\frac{n}{3}\right) + n \\
        &= 3 \cdot \left(3T\left(\frac{n}{9}\right) + \frac{n}{3}\right) + n =  9T\left(\frac{n}{9}\right) + 2n \\
        &= 9 \cdot \left(3T\left(\frac{n}{27}\right) + \frac{n}{9}\right) + 2n = 27T\left(\frac{n}{27}\right) + 3n
    \end{align*}
    führt zum Schluss, dass
    \begin{align*}
        T(n) = 3^i \cdot T\left(\frac{n}{3^i}\right) + i \cdot n
    \end{align*}
    mit einer maximalen Rekursionstiefe von \(\log_3(n)\) nachdem \(n / 3^{\log_3(n)} = 1\).

    Nach Einsetzen dieses Werts bleibt
    \begin{align*}
        T(n) &= 3^{\log_3(n)} \cdot T\left(\frac{n}{3^{\log_3(n)}}\right) + \log_3(n) \cdot n \\
        &= n \cdot T(1) + \log_3(n) \cdot n = n + n \cdot \log_3(n).
    \end{align*}

    Die Annahme ist nun, dass \(T(n) = O(n \cdot \log_3(n))\). Zu zeigen: Es gibt \(c, n_0\) mit \(n + n \cdot \log_3(n) \leq c(n \cdot \log_3(n))\) für alle \(n \geq n_0\). Man wähle \(c = 2\). Gemäß
    \begin{align*}
        n + n \cdot \log_3(n) &\leq 2(n \cdot \log_3(n)) \\
        n &\leq n \cdot \log_3(n) \\
        1 &\leq \log_3(n) \\
        3 &\leq n
    \end{align*}
    gilt \(n_0 = 3\). Somit gilt \(T(n) = O(n \log_3(n))\) mit $c = 2$ und $n_0 = 3$.

    \item Die Induktionsannahme ist \(T(n) = n + n\log_3(n)\). Die Induktionsbasis \(T(1) = 1 + 1 \cdot \log_3(1) = 1\) gilt. Gemäß
    \begin{align*}
        T(n) &= 3T\left(\frac{n}{3}\right) + n \\
        &= 3\left(\frac{n}{3} + \frac{n}{3}\log_3\left(\frac{n}{3}\right)\right) + n \\
        &= n + n \log_3\left(\frac{n}{3}\right) + n \\
        &= n + n \log_3(n)
    \end{align*}
    gilt nun $T(n) = O(n + n \log_3(n))$.
\end{enumerate}

\pagebreak
\paragraph{Aufgabe 2.}

\begin{enumerate}
    \item Wir haben $a = 6, b = 3, n^{\log_{3}(6)} = n^{1.631\ldots}$ und $f(n) = n^2 \log(n)$. Es gilt $f(n) = \Omega(n^{\log_{3}(6) + \epsilon})$ nachdem es $c, n_0, \epsilon > 0$ gibt derart, dass
    \begin{align*}
        n^2 \log(n) &\geq c \cdot n^{\log_{3}(6) + \epsilon}
    \end{align*}
    für alle $n > n_0$. (Etwa f\"ur $c \leq \log(n)$ und $0 < \epsilon \leq (2 - \log_{3}(6))$ ab $n_0 = 0$.)
 
    Ebenso gibt es ein $c < 1$ derart, dass
    \begin{align*}
        a\left(f\left(\frac{n}{3}\right)\right) &\leq c \cdot f(n) \\
        6\left(\left(\frac{n}{3}\right)^2 \log(\frac{n}{3})\right) &\leq c \cdot n^2 \log(n) \\
        \frac{6}{9}n^2 \log(\frac{n}{3}) &\leq c \cdot n^2 \log(n)
    \end{align*}
    nämlich $c = \frac{6}{9}$, nachdem klarerweise für alle $n > 0$ gilt, dass $\log(\frac{n}{3}) \leq \log(n)$. Somit haben wir $T(n) = \Theta(n^2 \log(n))$.

    \item Wir haben $a = 4, b = 2, log_2(4) = 2$ und $f(n) = n^2$. F\"ur $k = 0$ gilt $f(n) = \Theta(n^2 \cdot (\log(n))^0) = \Theta(n^2)$ und somit $T(n) = \Theta(n^2 \log(n))$.
    
    \item Wir haben $a = \sqrt{2}$, $b = 2$, $\log_2(\sqrt{2}) = \frac{1}{2}$ und $f(n) = \log(n)$. Nachdem $f(n) = O(\sqrt{n})$ (es gibt $c$ mit $\log(n) \leq c\sqrt{n}$, etwa $c = 1$ f\"ur $n \geq 0$) und das auch f\"ur eine Funktion mit minimal kleinerer Wachstumsrate als $\sqrt{n}$ funktioniert, gilt $T(n) = \Theta(\sqrt{n})$.
\end{enumerate}

\end{document}
