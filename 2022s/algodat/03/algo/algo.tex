\documentclass{article}
\usepackage[utf8]{inputenc}
\usepackage[ngerman]{babel}

% Convenience improvements
\usepackage{csquotes}
\usepackage{enumitem}
\usepackage{amsmath}
\usepackage{amssymb}
\usepackage{mathtools}
\usepackage{tabularx}

% Proper tables and centering for overfull ones
\usepackage{booktabs}
\usepackage{adjustbox}

% Change page/text dimensions, the package defaults work fine
\usepackage{geometry}

\usepackage{parskip}

% Drawings
\usepackage{tikz}

% Adjust header and footer
\usepackage{fancyhdr}
\pagestyle{fancy}
\fancyhead[L]{Algodat --- \textbf{Assignment 3}}
\fancyhead[R]{Laurenz Weixlbaumer (11804751)}
\fancyfoot[C]{}
\fancyfoot[R]{\thepage}
% Stop fancyhdr complaints
\setlength{\headheight}{12.5pt}

\newcommand{\Deltaop}{\, \Delta\, }
\newcommand{\xor}{\, \oplus\, }
\newcommand{\id}{\text{id}}

\begin{document}

The algorithm is given a maze (formatted as outlined in the assignment) and a starting (current) position $p$.

\begin{enumerate}
    \item If $p$ is an exit, mark it.
    \item If $p$ is a wall or has already been visited, return.
    \item Get all positions neighboring $p$ and recursively call the algorithm with the obtained positions.
\end{enumerate}

The first eight steps are thus

\begin{minipage}{.25\linewidth}
    \centering
    \begin{verbatim}
  0123456789A
0 ###########
1 #  #  #   #
2 # #  #  # #
3 #   ## ## #
4 ###       #
5 #S  # # ###
6 # ##  #   #
7 #   # ### #
8 # # # # # #
9 #   #     #
A ###########
    \end{verbatim}
\end{minipage}
\hfill
\begin{minipage}{.74\linewidth}
    \centering
    \begin{tabular}{c c c c}\toprule
        Depth & $p$ & obstacle/visited & neighbours \\\midrule
        0 & (5,1) & no & (5,0) (5,2) (4,1) (6,1) \\
        1 & (5,0) & yes & \\
        1 & (5,2) & no & (5,1) (5,3) (4,2) (6,2) \\
        2 & (5,1) & yes & \\
        2 & (5,3) & no & (5,2) (5,4) (4,3) (6,3) \\
        3 & (5,2) & yes & \\
        3 & (5,4) & yes & \\
        3 & (4,3) & no & (4,2) (4,4) (3,3) (5,3) \\\bottomrule
    \end{tabular}
\end{minipage}

\end{document}
