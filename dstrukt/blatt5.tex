\documentclass{article}
\usepackage[utf8]{inputenc}
\usepackage[ngerman]{babel}

% Convenience improvements
\usepackage{csquotes}
\usepackage{enumitem}
\usepackage{amsmath}
\usepackage{amssymb}

% Proper tables and centering for overfull ones
\usepackage{booktabs}
\usepackage{adjustbox}

% Change page/text dimensions, the defaults work fine
\usepackage{geometry}

\usepackage{parskip}

% Drawings
\usepackage{tikz}

% Adjust header and footer
\usepackage{fancyhdr}
\pagestyle{fancy}
\fancyhead[L]{Diskrete Strukturen --- \textbf{Übungsblatt 5}}
\fancyhead[R]{Laurenz Weixlbaumer (11804751)}
\fancyfoot[C]{}
\fancyfoot[R]{\thepage}

\newcommand{\R}{\mathbb{R}\ \\\ \{0\}}

\newcommand{\frectangle}{\tikz[scale=0.7, baseline]{\draw[fill] (0, 0) rectangle (1em, 1em)}}
\newcommand{\fcircle}   {\tikz[scale=0.7, baseline]{\draw[fill] (0.5em, 0.5em) circle [radius=0.5em]}}
\newcommand{\ftriangle} {\tikz[scale=0.7, baseline]{\draw[fill] (0, 0) -- (0.5em, 1em) -- (1em, 0) -- cycle}}

\begin{document}

\paragraph{Aufgabe 1.}

\begin{enumerate}[label=\alph*)]
    \item Zulässig. Die Elemente einer Äquivalenzklasse sind alle gleich lang.
    
    \item Zulässig. Die Elemente einer Äquivalenzklasse haben alle die gleichen Buchstaben.
    
    \item Nicht zulässig. Die Elemente einer Äquivalenzklasse können sich in der Anordnung der Buchstaben unterscheiden.
    
    \item Zulässig. Zu zeigen ist $x \sim y \Rightarrow xx \sim yy$ ($\sim$ sei die gegebene Relation, $x, y \in \Omega^*$). Es gilt $\lvert x \rvert = \lvert y \rvert$ und somit auch $\lvert xx \rvert = \lvert yy \rvert$ (Skriptum, Definition 2.2, S. 8). Ist ein beliebiger Buchstabe $c$ nicht in einem beliebigen Wort $\omega$ enthalten so ist er auch nicht in $\omega\omega$ enthalten.
    
    Somit haben $xx$ und $yy$ die gleiche Länge und beinhalten die gleichen Buchstaben. Es gilt $xx \sim yy$.
\end{enumerate}

\paragraph{Aufgabe 2.}

\begin{enumerate}[label=\alph*)]
    \item Nicht widerspruchsfrei. Es gilt etwa $h([0]_\sim) = 0$ und $f([0]_\sim) = 3$.
    
    \item Widerspruchsfrei. Es gilt
    \begin{align*}
        [0]_\sim = [3]_\sim = [6]_\sim &= \{ 0, 3, 6 \} \\
        [1]_\sim = [4]_\sim &= \{ 1, 4 \} \\
        [2]_\sim = [5]_\sim = [8]_\sim &= \{ 2, 5, 8 \} \\
    \end{align*}
    und
    \begin{align*}
        [0]_{\equiv_3} = [3]_{\equiv_3} = [6]_{\equiv_3} \\
        [1]_{\equiv_3} = [4]_{\equiv_3} \\
        [2]_{\equiv_3} = [5]_{\equiv_3} = [8]_{\equiv_3}. \\
    \end{align*}

    \item Widerspruchsfrei. Es gilt
    \begin{align*}
        [0]_{\equiv_6} = [6]_{\equiv_6} = [12]_{\equiv_6} \\
        [2]_{\equiv_6} = [8]_{\equiv_6} \\
        [4]_{\equiv_6} = [10]_{\equiv_6} = [16]_{\equiv_6} \\
    \end{align*}

    \item Nicht widerspruchsfrei. Es gilt etwa $h([0]_\sim) = \{ \ldots, -6, 0, 6, 12, \ldots \}$ aber $h([3]_\sim) = \{ \ldots, -9, -3, 3, 9, \ldots \}$ bei $[0]_\sim = [3]_\sim$.
\end{enumerate}

\paragraph{Aufgabe 3.}

\begin{enumerate}[label=\alph*)]
    \item\hfill

    \begin{tikzpicture}[
        node distance={10mm},
        main/.style={draw, circle}
    ]
        \node[main] (1) {$1$};
        \node[main] (2) [right of=1] {$2$};
        \node[main] (3) [right of=2] {$3$};
        \node[main] (4) [right of=3] {$4$};

        \draw (1) to (2);
        \draw (2) to (3);
        \draw (3) to (4);
        \draw (4) to[out=90, in=90, looseness=0.5] (1);
        \draw (2) to[out=90, in=90, looseness=0.5] (4);
        \draw (3) to[out=-90, in=-90, looseness=0.5] (1);
    \end{tikzpicture}
    
    \item\hfill
    
    \begin{tikzpicture}[
        node distance={12mm},
        main/.style={draw, circle}
    ]
        \node[main] (aaa) {\footnotesize\texttt{aaa}};
        \node[main] (aba) [below of=aaa] {\footnotesize\texttt{aba}};
        \node[main] (baa) [left of=aba] {\footnotesize\texttt{baa}};
        \node[main] (aab) [right of=aba] {\footnotesize\texttt{aab}};
        \node[main] (bbb) [below of=aba] {\footnotesize\texttt{bbb}};
        \node[main] (bba) [left of=bbb] {\footnotesize\texttt{bba}};
        \node[main] (abb) [right of=bbb] {\footnotesize\texttt{abb}};
        \node[main] (bab) [below of=bbb] {\footnotesize\texttt{bab}};

        \draw (aaa) to (aba);
        \draw (aaa) to (baa);
        \draw (aaa) to (aab);
        
        \draw (baa) to (bba);
        \draw (baa) to[out=225, in=-180, looseness=1.1] (bab);
        \draw (aab) to (abb);
        \draw (aab) to[out=-45, in=-0, looseness=1.1] (bab);
        
        \draw (aba) to (bba);
        \draw (aba) to (abb);
        
        \draw (bbb) to (bab);
        \draw (bbb) to (bba);
        \draw (bbb) to (abb);
    \end{tikzpicture}
    
    \item\hfill
    
    \begin{tikzpicture}[
        node distance={12mm},
        main/.style={draw, circle}
    ]
        \node[main] (a) {\footnotesize\texttt{a}};
        \node[main] (c) [right of=a] {\footnotesize\texttt{c}};
        \node[main] (e) [right of=c] {\footnotesize\texttt{e}};
        \node[main] (d) [right of=e] {\footnotesize\texttt{d}};
        \node[main] (b) [right of=d] {\footnotesize\texttt{b}};
        
        \draw (a) to (c);
        \draw (c) to (e);
        \draw (e) to (d);
        \draw (d) to (b);
        \draw (b) to[out=90, in=90, looseness=0.5] (a);
    \end{tikzpicture}
\end{enumerate}

\end{document}

aaa aab aba abb baa bab bba bbb

2 aaa : {aab, aba, baa}
3 aab : {aaa, abb, bab}
2 aba : {aaa, abb, bba}
3 abb : {aab, aba, bbb}
2 baa : {aaa, bab, bba}
3 bab : {aab, baa, bbb}
  bba : {aba, baa, bbb}
2 bbb : {abb, bab, bba}
