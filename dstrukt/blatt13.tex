\documentclass{article}
\usepackage[utf8]{inputenc}
\usepackage[ngerman]{babel}

% Convenience improvements
\usepackage{csquotes}
\usepackage{enumitem}
\setlist[enumerate,1]{label={\alph*)}}
\usepackage{amsmath}
\usepackage{amssymb}
\usepackage{mathtools}

% Proper tables and centering for overfull ones
\usepackage{booktabs}
\usepackage{adjustbox}

% Change page/text dimensions, the package defaults work fine
\usepackage{geometry}

\usepackage{parskip}

% Drawings
\usepackage{tikz}

% Adjust header and footer
\usepackage{fancyhdr}
\pagestyle{fancy}
\fancyhead[L]{Diskrete Strukturen --- \textbf{Übungsblatt 13}}
\fancyhead[R]{Laurenz Weixlbaumer (11804751)}
\fancyfoot[C]{}
\fancyfoot[R]{\thepage}
% Stop fancyhdr complaints
\setlength{\headheight}{12.5pt}

\newcommand{\R}{\mathbb{R}\ \\\ \{0\}}

\newcommand{\cmod}{\text{mod}}

\newcommand{\inline}{}

\begin{document}

\paragraph{Aufgabe 1.}

Induktionsanfang: Für $n = 0$ gilt
\begin{align*}
    \sum_{k = 0}^{0} (-1)^k {{m + 1}\choose{k}}
    &=
    (-1)^0 {{m}\choose{0}}
    \\
    (-1)^0 {{m + 1}\choose{0}}
    &=
    (-1)^0 {{m}\choose{0}}
    \\
    1 &= 1
\end{align*}
nachdem für beliebiges $x \in \mathbb{N}$ gilt, dass $x^0 = 1$ und ${{x}\choose{0}} = 1$.

Induktionsvorraussetzung: Angenommen $n \in \mathbb{N}$ ist so, dass
\begin{equation}\label{eq:induction-pre}
    \sum_{k = 0}^{n} (-1)^k {{m + 1}\choose{k}} = (-1)^n {{m}\choose{n}}
\end{equation}
gilt.

Induktionsschritt: Zu zeigen ist, dass dann auch
\begin{align}
    \sum_{k = 0}^{n + 1} (-1)^k {{m + 1}\choose{k}} &= (-1)^{n + 1} {{m}\choose{n + 1}} \label{eq:induction-goal}\\
    \left( \sum_{k = 0}^{n} (-1)^k {{m + 1}\choose{k}} \right) + (-1)^{n + 1} {{m + 1}\choose{n + 1}}
    &=
    (-1)^{n + 1} {{m}\choose{n + 1}} \nonumber \\
    (-1)^n {{m}\choose{n}} + (-1)^{n + 1} {{m + 1}\choose{n + 1}} &= (-1)^{n + 1} {{m}\choose{n + 1}} \label{eq:induction-step-2}
\end{align}
gilt. In \eqref{eq:induction-step-2} wurde \eqref{eq:induction-pre} angewendet. Wenn $n$ gerade dann $(-1)^n = 1$, andernfalls $(-1)^n = -1$. Angenommen $n$ gerade (also $n + 1$ ungerade), dann gilt für \eqref{eq:induction-step-2}, dass
\begin{align*}
    {{m}\choose{n}} - {{m + 1}\choose{n + 1}} &= -{{m}\choose{n + 1}} \\
    -{{m + 1}\choose{n + 1}} &= -{{m}\choose{n + 1}} - {{m}\choose{n}} \\
    {{m + 1}\choose{n + 1}} &= {{m}\choose{n}} + {{m}\choose{n + 1}}.
\end{align*}
Angenommen $n$ ungerade (also $n + 1$ gerade), dann gilt für \eqref{eq:induction-step-2}, dass
\begin{align*}
    -{{m}\choose{n}} + {{m + 1}\choose{n + 1}} &= {{m}\choose{n + 1}} \\
    {{m + 1}\choose{n + 1}} &= {{m}\choose{n}} + {{m}\choose{n + 1}}.
\end{align*}
Gemäß der Rekurrenz des Pascal-Dreiecks ${{m + 1}\choose{n + 1}} = {{m}\choose{n}} + {{m}\choose{n + 1}}$ ist nun \eqref{eq:induction-goal} bewiesen.

\paragraph{Aufgabe 2}

Wir wissen $F_0 = 0$ und $F_1 = 1$.

Induktionsanfang: Für $n = 0$ gilt
\begin{align*}
    F_0 &= (2F_1 - F_0)F_0     & F_1 &= F^{2}_{1} + F^{2}_{0} \\
    0 &= (2 - 0)0              & 1 &= 1 + 0 \\
    0 &= 0                     & 1 &= 1
\end{align*}

Induktionsvorraussetzung: Angenommen $n \in \mathbb{N}$ ist so, dass
\begin{align*}
    F_{2n} &= (2F_{n + 1} - F_n)F_n & F_{2n + 1} &= F^{2}_{n + 1} + F^{2}_{n}
\end{align*}
gilt.

Induktionsschritt: Zu zeigen ist, dass dann auch
\begin{align*}
    F_{2n + 2} &= (2F_{n + 2} - F_{n + 1})F_{n + 1} & F_{2n + 3} &= F^{2}_{n + 2} + F^{2}_{n + 1}
\end{align*}
gilt. Wir wissen $F_{2n + 2} = F_{2n + 1} + F_{2n}$ und somit
\begin{align*}
    F_{2n + 1} + F_{2n} &= (2F_{n + 2} - F_{n + 1})F_{n + 1} \\
    F^{2}_{n + 1} + F^{2}_{n} + (2F_{n + 1} - F_n)F_n &= (2F_{n + 2} - F_{n + 1})F_{n + 1}
\end{align*}

\paragraph{Aufgabe 3}

\begin{enumerate}
    \item 
\end{enumerate}

\end{document}

x(y - z)

xy - xz
