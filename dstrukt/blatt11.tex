\documentclass{article}
\usepackage[utf8]{inputenc}
\usepackage[ngerman]{babel}

% Convenience improvements
\usepackage{csquotes}
\usepackage{enumitem}
\setlist[enumerate,1]{label={\alph*)}}
\usepackage{amsmath}
\usepackage{amssymb}
\usepackage{mathtools}

% Proper tables and centering for overfull ones
\usepackage{booktabs}
\usepackage{adjustbox}

% Change page/text dimensions, the package defaults work fine
\usepackage{geometry}

\usepackage{parskip}

% Drawings
\usepackage{tikz}

% Adjust header and footer
\usepackage{fancyhdr}
\pagestyle{fancy}
\fancyhead[L]{Diskrete Strukturen --- \textbf{Übungsblatt 11}}
\fancyhead[R]{Laurenz Weixlbaumer (11804751)}
\fancyfoot[C]{}
\fancyfoot[R]{\thepage}
% Stop fancyhdr complaints
\setlength{\headheight}{12.5pt}

\newcommand{\R}{\mathbb{R}\ \\\ \{0\}}

\newcommand{\cmod}{\text{mod}}

\usepackage{seqsplit}

\begin{document}

\paragraph{Aufgabe 1.}

Gegeben ist ein \emph{public key} und eine verschlüsselte Nachricht. Verlangt ist eine Entschlüsselung der Nachricht gefolgt von einer Ableitung des \emph{private keys}.

Das macht für mich aus mehreren Gründen keinen Sinn. Erstens werden die Begriffe \emph{public} und \emph{private key} dubios verwendet. Im ersten Teil soll mithilfe eines \emph{public keys} eine Nachricht entschlüsselt werden, im zweiten Teil mithilfe eines \emph{private keys} eine Nachricht verschlüsselt werden. Jeweils das Umgekehrte wäre sinnvoll --- \emph{public key} verschlüsselt, \emph{private key} entschlüsselt.

Zweitens ergibt sich aus dem angeblichen \emph{public key} kein valider \emph{private key}. Man zerlege $m$ in die Primfaktoren $p = 1417021450037$ und $q = 1743801750631$. Daraus ergibt sich weiters $\phi = 2471004485253037876522680$ und $d = 2094576675656119524619681$ (eine von mehreren gleichwertigen Lösungen für $d$). Dann gilt aber für $x = 272797218939662984624437$, dass $\cmod(x^d, m) = \seqsplit{2191972902183325951471984}$, was kein valides Ergebnis ist.

Interpretiert man aber das als \emph{public key} gegebenen Paar als \emph{private key} $(m, d)$, nimmt man also $m = 2471004485256198699723347$ und $d = 679256688868919115503281$, funktioniert die Entschlüsselung wie erwartet mit
\begin{equation*}
    \cmod(x^d, m) = 282033334039281634 = \texttt{merryxmas}.
\end{equation*} (In diesem Fall muss auch der einleitende Satz zu \enquote{Ihr \emph{private key} ist\ldots. Sie erhalten von Ihrem Übungsleiter folgende Nachricht, die mit Ihrem \emph{public key} verschlüsselt wurde.} umgedacht werden.)

Interpretiert man nun weiter den zweiten Teil als Ableitung eines \emph{public keys} aus einem \emph{private key} machen auch die Hinweise, die sonst eher beim ersten Teil hilfreich gewesen wären, mehr Sinn. Abzuleiten ist nun also $e$ aus dem gegebenen $m$ und $d$. Gelten muss $de \equiv_{\phi} 1$, das $\phi$ ist schon von oben bekannt und unverändert, es ergibt sich $e = 2094576675656119524619681$ als modulares Inverses von $d$. Die Nachricht $\texttt{happynewyear}$ wird zu $x = 231631314029203840201633$ codiert, was zu
\begin{equation*}
    \cmod(x^e, m) = 156693474749568634695296
\end{equation*}
verschlüsselt werden kann.

% Der Teil $m$ des gegebenen Keys kann in die Primfaktoren
% \begin{equation*}
%     p = 1417021450037 \quad \text{und} \quad q = 1743801750631
% \end{equation*}
% zerlegt werden. Damit kann
% \begin{equation*}
%     \phi = (1417021450037 - 1)(1743801750631 - 1) = 2471004485253037876522680
% \end{equation*}
% berechnet werden. (Es gilt nun $\gcd(e, \phi) = 1$.) Weiters kann nun ein modulares Inverses von $e$ mit dem Modulus $\phi$,
% \begin{equation*}
%     d e \equiv_{\phi} 1 \quad \text{mit} \quad d = 2094576675656119524619681
% \end{equation*}
% berechnet werden. Die Funktionen zu Ver- und Entschlüsselung (jeweils $\mathbb{Z}_m \rightarrow \mathbb{Z}_m$) sind nun
% \begin{align*}
%     f(X) &= X^e = X^{679256688868919115503281} \\
%     f^{-1}(X) &= X^d = X^{2094576675656119524619681}.
% \end{align*}

% \begin{enumerate}
%     \item Es gilt $f^{-1}(272797218939662984624437) = 2191972902183325951471984$, das ist kein valides Ergebnis nachdem etwa 84 (ersten beiden Ziffern) schon größer als der größtmögliche Wert (41) ist.
    
%     \item 
% \end{enumerate}

\end{document}
