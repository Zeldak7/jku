\documentclass{article}
\usepackage[utf8]{inputenc}
\usepackage[ngerman]{babel}

% Convenience improvements
\usepackage{csquotes}
\usepackage{enumitem}
\setlist[enumerate,1]{label={\alph*)}}
\usepackage{amsmath}
\usepackage{amssymb}
\usepackage{mathtools}

% Proper tables and centering for overfull ones
\usepackage{booktabs}
\usepackage{adjustbox}

% Change page/text dimensions, the package defaults work fine
\usepackage{geometry}

\usepackage{parskip}

% Drawings
\usepackage{tikz}

% Adjust header and footer
\usepackage{fancyhdr}
\pagestyle{fancy}
\fancyhead[L]{Diskrete Strukturen --- \textbf{Übungsblatt 13}}
\fancyhead[R]{Laurenz Weixlbaumer (11804751)}
\fancyfoot[C]{}
\fancyfoot[R]{\thepage}
% Stop fancyhdr complaints
\setlength{\headheight}{12.5pt}

\newcommand{\R}{\mathbb{R}\ \\\ \{0\}}

\newcommand{\cmod}{\text{mod}}

\newcommand{\bO}{\text{O}}

\begin{document}

\paragraph{Aufgabe 1.}

 Zu zeigen ist
\begin{align*}
    f(n) = \bO(g(n)) \land g(n) &= \bO(h(n)) \Rightarrow f(n) = \bO(h(n)).
\end{align*}
(Es seien $c, k, j, n_*$ in $\mathbb{N}$.) Angenommen die linke Seite der Implikation,
\begin{align}
    \label{eq:1-1} & \exists\, c, n_0 > 0 : \forall\, n \geq n_0 : |f(n)| \leq c|g(n)|,\ \text{und} \\
    \label{eq:1-2} & \exists\, k, n_1 > 0 : \forall\, n \geq n_1 : |g(n)| \leq k|h(n)|,
\end{align}
gilt. Die Ungleichung aus \eqref{eq:1-1} kann zu $\frac{|f(n)|}{c} \leq |g(n)|$ umformuliert werden und in \eqref{eq:1-2} eingesetzt werden:
\begin{align*}
    & \exists\, k, n_1 > 0 : \forall\, n \geq n_1 : \frac{|f(n)|}{c} \leq k|h(n)| \\
    & \exists\, k, n_1 > 0 : \forall\, n \geq n_1 : |f(n)| \leq ck|h(n)|
\end{align*}
Dann gilt
\begin{align*}
    \exists\, j, n_2 > 0 : \forall\, n \geq n_2 : |f(n)| \leq j|h(n)| 
\end{align*}
für $j = ck$ und $n_2 = \text{max}\{n_0, n_1\}$. Somit gilt $f(n) = O(h(n))$ wenn $f(n) = O(g(n))$ und $g(n) = O(h(n))$.

\paragraph{Aufgabe 2.}

Es gilt $f(n) = \sqrt{n}$.

\begin{enumerate}
    \item Es gilt $a = 1$, $b = 2$ und $c = \log_2(1) = 0$. 
    %Wir sind in Fall 2,  $\bO(\sqrt{n})$
    \item Es gilt $a = 2$, $b = 2$ und $c = \log_2(2) = 1$. 
    % Somit gilt $T(n) = \Theta(n)$.
    \item Es gilt $a = 2$, $b = 4$ und $c = \log_4(2) = \frac{1}{2}$. Es greift Fall 2 wegen $f(n) = \bO(\sqrt{n})$ mit $k = 0$. Somit gilt $T(n) = \bO(\sqrt{n} \log(n))$. (Fall 3 bzw. $f(n) = \Omega(\sqrt{n})$ gilt nicht weil es f\"ur kein $n$ ein $c > 0$ gibt mit $\sqrt{n}\geq c\sqrt{n}$.)
\end{enumerate}

\end{document}
