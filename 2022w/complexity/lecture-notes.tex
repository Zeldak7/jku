\documentclass{article}
\usepackage[utf8]{inputenc}
\usepackage[ngerman]{babel}

% Convenience improvements
\usepackage{csquotes}
\usepackage{enumitem}
\setlist[enumerate,1]{label={\alph*)}}
\usepackage{amsmath}
\usepackage{amssymb}
\usepackage{mathtools}
\usepackage{tabularx}

% Proper tables and centering for overfull ones
\usepackage{booktabs}
\usepackage{adjustbox}

% Change page/text dimensions, the package defaults work fine
\usepackage{geometry}

\usepackage{parskip}

% Drawings
\usepackage{tikz}
\usepackage{pgfplots}

% Adjust header and footer
\usepackage{fancyhdr}
\pagestyle{fancy}
\fancyhead[L]{\textbf{Computational Complexity} --- Lecture Notes}
\fancyhead[R]{Laurenz Weixlbaumer (11804751)}
\fancyfoot[C]{}
\fancyfoot[R]{\thepage}
% Stop fancyhdr complaints
\setlength{\headheight}{12.5pt}

\newcommand{\Deltaop}{\, \Delta\, }
\newcommand{\xor}{\, \oplus\, }
\newcommand{\id}{\text{id}}

\begin{document}

\paragraph{Exercise 1.2} The minimal number of parameters (pairwise distances) that is required to completely specify a TSP instance is $\begin{psmallmatrix}
    n \\ 2
\end{psmallmatrix}$, the binomial coefficient $n$ over 2. Given a set with $n$ elements it describes the number of subsets with exactly $k$ (in our case 2) elements.

\paragraph{Exercise 1.4} The number of possible routes as a function of the number of cities is $f(n) = \begin{bsmallmatrix}n \\ 1\end{bsmallmatrix}$, the $n$th stirling number of the first kind with $k = 1$. There are $n!$ different ways of arranging the $n$ cities into a route, but many of them are identical from our perspective. See for example all $4! = 26$ permutations of $\{1, 2, 3, 4\}$,
\begin{align*}
    (1, 2, 3, 4) &= (2, 3, 4, 1) = (3, 4, 1, 2) = (4, 1, 2, 3) \\
    (2, 1, 3, 4) &= (3, 4, 2, 1) = (4, 2, 1, 3) = (1, 3, 4, 2) \\
    (3, 1, 2, 4) &= (1, 2, 4, 3) = (2, 4, 3, 1) = (4, 3, 1, 2) \\
    (1, 3, 2, 4) &= (4, 1, 3, 2) = (2, 4, 1, 3) = (3, 2, 4, 1) \\
    (2, 3, 1, 4) &= (4, 2, 3, 1) = (3, 1, 4, 2) = (1, 4, 2, 3) \\
    (3, 2, 1, 4) &= (4, 3, 2, 1) = (1, 4, 3, 2) = (2, 1, 4, 3)
\end{align*}
of which only six are unique. The number of permutations on $n$ elements with one cycle describes this nicely.

\paragraph{Problem 1.9} See Exercises 1.2 and 1.4.

\paragraph{Numericals 1.10} Not implemented, getting the pairwise distances would be a hassle.

\paragraph{Problem 1.11} We want to show that: \enquote{Every symbol from a given alphabet can be represented as a bitstring. This representation is one-to-one and only scales logarithmically in alphabet size.} Consider an alphabet $A$ with $n$ symbols where each symbol $a_0, a_1, \ldots, a_{n - 1}$ can be represented by an unique integer. Because we can use the index of an element as its integer counterpart, at least one such representation exists

We can uniquely represent an integer $c$ as a bitstring $b$ through the following algorithm: Zero all bits in $b$. Find the largest $x$ such that $2^x$ divides $c$. Set $b_x$ to one and set $c$ to $c - 2^x$. Continue until $c$ is zero. The length of such a bitstring is determined by the $x$ of the first iteration, so the first $x$ such that $2^x\ |\ c$. Given only $c$, the length of the resulting bitstring will thus be $\lfloor log_2(c) \rfloor$.

\end{document}
