\documentclass{article}
\usepackage[utf8]{inputenc}
\usepackage[ngerman]{babel}

% Convenience improvements
\usepackage{csquotes}
\usepackage{enumitem}
\setlist[enumerate,1]{label={\alph*)}}
\usepackage{amsmath}
\usepackage{amssymb}
\usepackage{mathtools}
\usepackage{tabularx}

% Proper tables and centering for overfull ones
\usepackage{booktabs}
\usepackage{adjustbox}

% Change page/text dimensions, the package defaults work fine
\usepackage{geometry}

\usepackage{parskip}

% Drawings
\usepackage{tikz}
\usepackage{pgfplots}
\usetikzlibrary{automata,positioning} 

% Adjust header and footer
\usepackage{fancyhdr}
\pagestyle{fancy}
\fancyhead[L]{\textbf{Computational Complexity} --- Special Topic 3}
\fancyhead[R]{Laurenz Weixlbaumer (11804751)}
\fancyfoot[C]{}
\fancyfoot[R]{\thepage}
% Stop fancyhdr complaints
\setlength{\headheight}{12.5pt}

\usepackage{hyperref}

\newcommand{\Deltaop}{\, \Delta\, }
\newcommand{\xor}{\, \oplus\, }
\newcommand{\id}{\text{id}}

\begin{document}

\section*{Time Hierarchy Theorem}

Informally, the theorem states that given more time, a Turing machine can solve more problems. (In other words, more languages can be decided with more time.) One way to state it formally is that if $f$ and $g$ are time-honest functions with $f(n) \log f(n) = O(g(n))$, then
\begin{align*}
    \mathsf{DTIME}(f(n)) \subsetneq \mathsf{DTIME}\left(g(n)\right).
\end{align*}
As suggested, I have chosen to present a proof of a weaker version, showing that $\mathsf{DTIME}(n)$ is a strict subset of $\mathsf{DTIME}(n^{1.5})$. The following are notes based on the proof presented in \emph{Computational Complexity: A Modern Approach} by S. Arora and B. Barak, available on p. 66 of the public draft.

Let $M_x$ be the TM represented by a string $x$ with outputs $\{0, 1\}$ and let $U$ be an Universal TM (i.e. an \enquote{interpreter} of TMs, which is itself a TM). Now consider a TM $D$ which, given as input $x$, runs $U$ for $|x|^{1.4}$ steps, simulating the execution of $M_x$ on $x$. If an output $M_x(x)$ is obtained in time it outputs $1 - M_x(x)$, the opposite. Otherwise it outputs zero.

Clearly, $D$ finishes execution within $n^{1.4}$ steps since it only lets $U$ run for as many steps and doesn't do anything else that scales with $n$. Thus the language decided by $D$ is in $\mathsf{DTIME}(n^{1.5})$. The following will show that this language is not in $\mathsf{DTIME}(n)$, thus providing a proof of the initial statement.

For the sake of contradiction assume that there exists a TM $M$ which decides the same language as $D$ but runs in time $kn$ for input sizes $n$ where $k$ is a constant factor independent of $n$. Then for all possible inputs $x$ we have $M(x) = D(x)$: The machines are functionally equivalent but $M$ runs in linear time.

The time needed to simulate $M$ on $U$ is at most $ck|x|\log|x|$ for some constant $c$ which is independent of $x$. (This makes use of the general statement that the runtime of an Universal TM is bounded by $cT \log T$ where the simulated machine halts within $T$ steps.) Naturally, at some point $n^{1.4}$ will grow larger than $ckn \log n$. Let $n_0$ be the smallest $n$ such that this is the case and let $x$ (representing the TM $M_x$) have a length of $n_0$. Then $D(x)$ will obtain the output of $M(x)$ within $|x|^{1.4}$ steps (since it stops at that point) but by definition we have
\begin{align*}
    D(x) = 1 - M(x) \neq M(x),
\end{align*}
and have arrived at a contradiction.

\end{document}
