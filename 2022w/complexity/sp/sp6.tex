\documentclass{article}
\usepackage[utf8]{inputenc}
\usepackage[ngerman]{babel}

% Convenience improvements
\usepackage{csquotes}
\usepackage{enumitem}
\setlist[enumerate,1]{label={\alph*)}}
\usepackage{amsmath}
\usepackage{amssymb}
\usepackage{mathtools}
\usepackage{tabularx}

% Proper tables and centering for overfull ones
\usepackage{booktabs}
\usepackage{adjustbox}

% Change page/text dimensions, the package defaults work fine
\usepackage{geometry}

\usepackage{parskip}

% Drawings
\usepackage{tikz}
\usepackage{pgfplots}
\usetikzlibrary{automata,positioning} 

% Adjust header and footer
\usepackage{fancyhdr}
\pagestyle{fancy}
\fancyhead[L]{\textbf{Computational Complexity} --- Special Topic 6}
\fancyhead[R]{Laurenz Weixlbaumer (11804751)}
\fancyfoot[C]{}
\fancyfoot[R]{\thepage}
% Stop fancyhdr complaints
\setlength{\headheight}{12.5pt}

\usepackage{hyperref}

\newcommand{\Deltaop}{\, \Delta\, }
\newcommand{\xor}{\, \oplus\, }
\newcommand{\id}{\text{id}}

\begin{document}

\section*{The Deutsch-Josza Algorithm}

\paragraph{Essentials} We will describe quantum computation in terms of circuits as opposed to TMs. These circuits are time-reversible. Wires contain qubits, which are either $|0\rangle$, $|1\rangle$ or both (then we call the configuration a \emph{superposition}). Superposition collapse when they are inspected.

\paragraph{Minimal Circuit} A prototypical quantum circuit has four phases:

\begin{enumerate}[label=\arabic*.]
    \item Encode input bitstring (0, 1) into qubits ($|0\rangle$, $|1\rangle$)
    \item Enter superpositions
    \item Apply some Boolean circuit
    \item Revert superpositions
    \item Measure output wires (once again 0, 1)
\end{enumerate}

\emph{I didn't write down the full example, but suffice to say that we were able to construct a quantum circuit which determines whether a 1-bit function is constant or balanced using a single query. A conventional approach would naturally need two.}

\paragraph{Deutsch-Josza} The above generalizes to $n$-bit functions $\{0, 1\}^n \to \{0, 1\}$, meaning that there exists a quantum circuit to determine whether such a function is balanced or constant \emph{in a single query}. A conventional approach would require $O(2^n)$ function queries (worst case).

Although it's somewhat contrived, the algorithm demonstrates an exponential quantum advantage. Other problems that can be approached with similar ideas include factoring (Shor), solving linear systems, machine learning, \ldots

\end{document}
