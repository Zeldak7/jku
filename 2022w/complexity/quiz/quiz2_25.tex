\documentclass{article}
\usepackage[utf8]{inputenc}
\usepackage[ngerman]{babel}

% Convenience improvements
\usepackage{csquotes}
\usepackage{enumitem}
\setlist[enumerate,1]{label={\alph*)}}
\usepackage{amsmath}
\usepackage{amssymb}
\usepackage{mathtools}
\usepackage{tabularx}

% Proper tables and centering for overfull ones
\usepackage{booktabs}
\usepackage{adjustbox}

% Change page/text dimensions, the package defaults work fine
\usepackage{geometry}

\usepackage{parskip}

% Drawings
\usepackage{tikz}
\usepackage{pgfplots}
\usetikzlibrary{automata,positioning,shapes,shapes.geometric} 

% Adjust header and footer
\usepackage{fancyhdr}
\pagestyle{fancy}
\fancyhead[L]{\textbf{Computational Complexity} --- Quiz 2/25}
\fancyhead[R]{Laurenz Weixlbaumer (11804751)}
\fancyfoot[C]{}
\fancyfoot[R]{\thepage}
% Stop fancyhdr complaints
\setlength{\headheight}{12.5pt}

\newcommand{\Deltaop}{\, \Delta\, }
\newcommand{\xor}{\, \oplus\, }
\newcommand{\id}{\text{id}}

\begin{document}

It is assumed that the tape is initially set to zero. Using another symbol for emptyness wouldn't really make sense in this context in my opinion. (Because we want the alphabet to be printable, how do you print emptyness?) I use \enquote{\#} to mark the starting position, this is also printed as a \enquote{divider} between the binary numbers. The print state is labeled $p$.

A transition (*, *, L) should be interpreted as: Read any $c$ of the alphabet $c$, write $c$ to the tape again and move left; this is done to declutter the diagram, such edges can trivially be replaced with usual ones without changing behaviour.

\begin{center}
    \begin{tikzpicture}[shorten >=1pt,node distance=3cm,on grid,auto,initial text=]
        \node[state, initial] (initial) {$i$};
        \node[state, right of=initial] (return) {$p$};
        \node[state, right of=return] (write) {$w$};
        \node[state, below of=write] (write') {$w'$};

        \path[->] (initial) edge [] node {(0, \#, L)} (return);

        \path[->] (return) edge [loop above] node {(0, 0, R)} (return);
        \path[->] (return) edge [loop below] node {(1, 1, R)} (return);

        \path[->] (return) edge [] node {(\#, \#, L)} (write);

        \path[->] (write) edge [] node {(0, 1, R)} (write');
        \path[->] (write) edge [loop above] node {(1, 0, L)} (write);

        \path[->] (write') edge [] node  [sloped] {(*, *, L)} (return);
    \end{tikzpicture}
\end{center}

Outputs 0\#1\#10\#11\#100\#\ldots

\end{document}
