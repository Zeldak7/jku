\documentclass{article}
\usepackage[utf8]{inputenc}
\usepackage[ngerman]{babel}

% Convenience improvements
\usepackage{csquotes}
\usepackage{enumitem}
\setlist[enumerate,1]{label={\alph*)}}
\usepackage{amsmath}
\usepackage{amssymb}
\usepackage{mathtools}
\usepackage{tabularx}

% Proper tables and centering for overfull ones
\usepackage{booktabs}
\usepackage{adjustbox}

% Change page/text dimensions, the package defaults work fine
\usepackage{geometry}

\usepackage{parskip}

% Drawings
\usepackage{tikz}
\usepackage{pgfplots}
\usetikzlibrary{automata,positioning,shapes,shapes.geometric} 

% Adjust header and footer
\usepackage{fancyhdr}
\pagestyle{fancy}
\fancyhead[L]{\textbf{Computational Complexity} --- Quiz 2/8}
\fancyhead[R]{Laurenz Weixlbaumer (11804751)}
\fancyfoot[C]{}
\fancyfoot[R]{\thepage}
% Stop fancyhdr complaints
\setlength{\headheight}{12.5pt}

\newcommand{\Deltaop}{\, \Delta\, }
\newcommand{\xor}{\, \oplus\, }
\newcommand{\id}{\text{id}}

\begin{document}

Assume for the sake of contradiction that \textsc{Palindrome} is regular. Then the pumping lemma must hold for a certain pumping length $l$. Let $x = 0^l10^l \in \text{\textsc{Palindrome}}$, then $|x| = 2l + 1 \leq l$. By the pumping lemma we can now decompose this string into substrings $u, v, w$ with $x = uvw$ such that
\begin{align*}
    v &\neq \epsilon, & |uv| &\leq l, & uv^kw \in \text{\textsc{Palindrome} for all $k \geq 0$}.
\end{align*}
Because $uvw = 0^l10^l$, the substring $uv$ can not include any 1s since it can't be long enough to reach the single 1, which is at $l + 1$. So we conclude $v = 0^i$ for some $i \geq 1$ (not $\geq 0$ because $i \neq \epsilon$). By the above conditions we can \enquote{pump up} $uvw$ by repeating $v$ an arbitrary amount of times $k$. Let $k = 2$, then
\begin{align*}
    uv^2w = 0^{l + i}10^l
\end{align*}
must also be in \textsc{Palindrome}. It clearly isn't since $0^{l + k} \neq 0^l$ or rather $l + k \neq l$ because $k = 2$.

\end{document}
