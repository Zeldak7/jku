\documentclass{article}
\usepackage[utf8]{inputenc}
\usepackage[ngerman]{babel}

% Convenience improvements
\usepackage{csquotes}
\usepackage{enumitem}
\setlist[enumerate,1]{label={\alph*)}}
\usepackage{amsmath}
\usepackage{amssymb}
\usepackage{mathtools}
\usepackage{tabularx}

% Proper tables and centering for overfull ones
\usepackage{booktabs}
\usepackage{adjustbox}

% Change page/text dimensions, the package defaults work fine
\usepackage{geometry}

\usepackage{parskip}

% Drawings
\usepackage{tikz}
\usepackage{pgfplots}
\usetikzlibrary{automata,positioning} 

% Adjust header and footer
\usepackage{fancyhdr}
\pagestyle{fancy}
\fancyhead[L]{\textbf{Computational Complexity}}
\fancyhead[R]{Laurenz Weixlbaumer (11804751)}
\fancyfoot[C]{}
\fancyfoot[R]{\thepage}
% Stop fancyhdr complaints
\setlength{\headheight}{12.5pt}

\newcommand{\Deltaop}{\, \Delta\, }
\newcommand{\xor}{\, \oplus\, }
\newcommand{\id}{\text{id}}

\begin{document}

We want to show that: \enquote{Every symbol from a given alphabet can be represented as a bitstring. This representation is one-to-one and only scales logarithmically in alphabet size.} Consider an alphabet $A$ with $n$ symbols where each symbol $a_0, a_1, \ldots, a_{n - 1}$ can be represented by an unique integer. Because we can use the index of an element as its integer counterpart, at least one such representation exists

We can uniquely represent an integer $c$ as a bitstring $b$ through the following algorithm: Zero all bits in $b$. Find the largest $x$ such that $2^x$ divides $c$. Set $b_x$ to one and set $c$ to $c - 2^x$. Continue until $c$ is zero. The length of such a bitstring is determined by the $x$ of the first iteration, so the first $x$ such that $2^x\ |\ c$. The length of the resulting bitstring will thus be $\lfloor log_2(c) \rfloor$, hence the logarithmic scaling in the size of the alphabet, 2.

\end{document}
