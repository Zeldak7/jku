\documentclass{article}
\usepackage[utf8]{inputenc}
\usepackage[ngerman]{babel}

% Convenience improvements
\usepackage{csquotes}
\usepackage{enumitem}
\setlist[enumerate,1]{label={\alph*)}}
\usepackage{amsmath}
\usepackage{amssymb}
\usepackage{mathtools}
\usepackage{tabularx}
\usepackage{listings}

% Proper tables and centering for overfull ones
\usepackage{booktabs}
\usepackage{adjustbox}

% Change page/text dimensions, the package defaults work fine
\usepackage{geometry}

\usepackage{parskip}

% Drawings
\usepackage{tikz}
\usepackage{pgfplots}

% Adjust header and footer
\usepackage{fancyhdr}
\pagestyle{fancy}
\fancyhead[L]{Computernetzwerke --- \textbf{Übung 2}}
\fancyhead[R]{Laurenz Weixlbaumer (11804751)}
\fancyfoot[C]{}
\fancyfoot[R]{\thepage}
% Stop fancyhdr complaints
\setlength{\headheight}{12.5pt}

\newcommand{\Deltaop}{\, \Delta\, }
\newcommand{\xor}{\, \oplus\, }
\newcommand{\id}{\text{id}}

\begin{document}

\paragraph{Aufgabe 1.}

\begin{enumerate}
    \item Wenn keine Verbindung zum Internet notwendig ist; interne Kommunikationen und Abläufe.

    \item Die folgenden drei Blöcke sind reserviert
    \begin{lstlisting}[gobble=8]
        10.0.0.0        -   10.255.255.255  (10/8 prefix)
        172.16.0.0      -   172.31.255.255  (172.16/12 prefix)
        192.168.0.0     -   192.168.255.255 (192.168/16 prefix)
    \end{lstlisting}

    \item Die privaten Adressenbereiche der fusionierenden Firmen könnten Überschneidungen aufweisen, private Adressen müssen klarerweise nicht global eindeutig sein. 
\end{enumerate}

\paragraph{Aufgabe 2.} 

\begin{enumerate}
    \item \lstinline{18.37.100.203}

    \item \lstinline{00110100.00000000.10110100.11011100}

    \item \begin{enumerate}
        \item Nein, die jeweiligen Netzwerkteile sind ungleich (\lstinline{210.210.210} $\neq$ \lstinline{210.210.211}).
        \item Ja, \lstinline{15.1} = \lstinline{15.1}.
        \item Nein. Konvertierung in Bitmuster
        \begin{lstlisting}[gobble=12]
            16.17.18.19
            00010000.00010001.00010010.00010011

            16.17.64.1
            00010000.00010001.01000000.00000001
        \end{lstlisting}
        zeigt, dass die ersten 18 bits nicht gleich sind. Ergo nicht im selben Subnetz.
    \end{enumerate}
\end{enumerate}

\paragraph{Aufgabe 3.}

\begin{enumerate}
    \item \begin{lstlisting}[gobble=8]
        92.168.1.0/26       (Alpha)
        92.168.1.64/26      (Bravo)
        92.168.1.128/26     (Charlie)
        92.168.1.192/26     (Delta)
    \end{lstlisting}

    \item \begin{lstlisting}[gobble=8]
        92.168.1.0/25       (Alpha)     [128]
        92.168.1.128/26     (Delta)     [64]
        92.168.1.192/27     (Bravo)     [32]
        92.168.1.224/27     (Charlie)   [32]
    \end{lstlisting}
\end{enumerate}

\end{document}
