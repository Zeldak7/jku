\documentclass{article}
\usepackage[utf8]{inputenc}
\usepackage[ngerman]{babel}

% Convenience improvements
\usepackage{csquotes}
\usepackage{enumitem}
\setlist[enumerate,1]{label={\alph*)}}
\usepackage{amsmath}
\usepackage{amssymb}
\usepackage{mathtools}
\usepackage{tabularx}
\usepackage{listings}

% Proper tables and centering for overfull ones
\usepackage{booktabs}
\usepackage{adjustbox}

% Change page/text dimensions, the package defaults work fine
\usepackage{geometry}

\usepackage{parskip}

% Drawings
\usepackage{tikz}
\usepackage{pgfplots}

% Adjust header and footer
\usepackage{fancyhdr}
\pagestyle{fancy}
\fancyhead[L]{Computernetzwerke --- \textbf{Übung 4}}
\fancyhead[R]{Laurenz Weixlbaumer (11804751)}
\fancyfoot[C]{}
\fancyfoot[R]{\thepage}
% Stop fancyhdr complaints
\setlength{\headheight}{12.5pt}

\newcommand{\Deltaop}{\, \Delta\, }
\newcommand{\xor}{\, \oplus\, }
\newcommand{\id}{\text{id}}

\begin{document}

\paragraph{Aufgabe 1.}

\begin{enumerate}[label=\roman*.]
    \item Application Layer. Die Basis f\"ur Kommunikation im Internet, verwendet um \enquote{Hypertext} (\"ubl. HTML) zu transportieren. Bis HTTP/2 TCP, HTTP/3 QUIC aufbauend auf UDP. Port 80. 
    
    \item Application Layer. Verwendet um Daten zu \"ubertragen, \"ubl. mithilfe eines FTP-Clients. TCP/20 f\"ur transfer, TCP/21 f\"ur control.
    
    \item Network Layer. Versendet Pakete u\"ber Netzwerkgrenzen basierend auf IP-Adressen.
    
    \item Physical Layer und Data Link Layer. Erm\"oglicht kabelgebundene Computernetzwerke (etwa LANs).
    
    \item Application Layer. Siehe i., aber versendet mit TLS um Sicherheit, Integrit\"at und Authentizität zu gew\"ahrleisten. Port 443.
    
    \item Data Link Layer. Verwendet um in einem Netzwerk automatisch IP Adressen (etc.) zu verteilen. Ein DHCP-Server spricht mit Clients die den DHCP protocal stack verwenden. UDP/67 Server, UDP/68 Client.
    
    \item Network Layer. Verwendet um low-level Fehler- bzw. Statusmeldungen zu zu versenden, insb. von Routern.
    
    \item Application Layer. Verwendet um Netzwerke zu Verwalten (Router, Server, etc. \"uberwachen und kontrollieren). Agent empfangt auf UDP/161, Manager empfangt auf UDP/162.
    
    \item Application Layer. Assoziiert Domain Namen mit IP Adressen. 53/UDP bzw. TCP.
    
    \item Application Layer. Erweiterung von SSH um (sicheren) Datentransfer zu erm\"oglichen. TCP/22.
    
    \item Application Layer. Verwendet von Mail-Clients um Mails von einem Mailserver abzufragen. \"Ublicherweise werden alle Mails auf dem Mailserver gespeichert und sind so abfragbar. TCP/143.
    
    \item Application Layer. Verwendet von Mail-Clients um Mails von einem Mailserver abzufragen. Mails werden \"ublicherweise nicht auf dem Server gespeichert, sondern gel\"oscht sobald sie vom Client abgefragt wurden. (Mittlerweile de facto von IMAP abgel\"ost.) TCP/110.
    
    \item Application Layer. Verwendet von Mailservern um Mails zu versenden/erhalten. TCP/25 f\"ur server-server TCP/587 f\"ur client-server.
    
    \item Session Layer, Application Layer (?, passt nicht wirklich in das OSI Modell). Verwendet um Netzwerkkommunikation abh\"or und f\"alschungssicher zu machen. Wei ein \enquote{wrapper} um regul\"are Verbindung vorstellbar.
    
    \item Application Layer. Verwendet f\"ur remote login und command line execution auf (\"ubl.) Servern, abgesichert durch asymmetrische Kryptographie. TCP/22.
    
    \item Application Layer. Obsoleter Vorg\"anger von SSH mit \"ahnlicher Funktionalit\"at aber keiner Verschl\"usselung. TCP/23.
\end{enumerate}

\paragraph{Aufgabe 2.}

\begin{enumerate}
    \item Erfasst wurde
    \begin{enumerate}
        \item DHCP Discover, broadcast vom Client ohne requested IP address. Teilt dem Server mit, dass er eine IP Adresse will.
        \item DHCP Offer. Server bietet Client die Addresse 192.168.0.10 an.
        \item DHCP Request. Client broadcastet eine Anfrage, diese Addresse zu bekommen.
        \item DHCP ACK. Server best\"atigt die Anfrage, die Addresse ist jetzt dem Client zugewiesen.
    \end{enumerate}

    \item Auf L2 k\"onnte man Frames sehen. (Frame 1: 314 bytes on wire (2512 bits), 314 bytes captured (2512 bits) \ldots) Auf L3 k\"onnte man Pakete mit assoziierten IP Addressen sehen. (Internet Protocol Version 4, Src: 192.168.0.1, Dst: 192.168.0.10 \ldots) Auf L4 lebt UDP, hier werden Ports relevant. (User Datagram Protocol, Src Port: 67, Dst Port: 68 \ldots) Auf L7 sieht man dann die ganze DHCP Nachricht.
    
    \item 192.168.0.10
\end{enumerate}

\end{document}
