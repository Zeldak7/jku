\documentclass{article}
\usepackage[utf8]{inputenc}
\usepackage[ngerman]{babel}

\usepackage{multicol}

% Convenience improvements
\usepackage{csquotes}
\usepackage{enumitem}
\usepackage{amsmath}
\usepackage{amssymb}
\usepackage{mathtools}
\usepackage{tabularx}

% Proper tables and centering for overfull ones
\usepackage{booktabs}
\usepackage{adjustbox}

% Change page/text dimensions, the package defaults work fine
\usepackage{geometry}

\usepackage{parskip}

% Adjust header and footer
\usepackage{fancyhdr}

\newcommand{\Deltaop}{\, \Delta\, }
\newcommand{\xor}{\, \oplus\, }
\newcommand{\id}{\text{id}}
\newcommand{\proj}{\text{proj}}

\newcommand{\code}[1]{\texttt{#1}}

\begin{document}

\begin{multicols}{1}

\section*{Referenzmodelle}

\ldots beschreiben Netzarchitekturen, also Protokolle und Schichten, lassen dabei die Implementierung offen.

\paragraph{ISO/OSI} besteht aus

\begin{enumerate}
    \item Physical Layer. (Bitübertragung; Repeater und Hubs)
    \item Data Link Layer. (Data Frames, Checksums; MAC, ARP, Bridges, Switches -- LAN-Bereich)
    \item Network Layer. (Routing; IP, Router -- Weltweit)
    \item Transport Layer. (end-to-end Zuver\-lässig\-keit; TCP, UDP)
    \item Session Layer. (Recovery points, Duplex; RPC)
    \item Presentation Layer.
    \item Application Layer. (DNS, HTTP, SMTP)
\end{enumerate}

Kritik: Presentation und Session Layer könnten im Application Layer sein. Data Link und Network Layer sind sehr voll. Error control wiederholt sich auf mehreren Schichten.

\paragraph{TCP/IP} ist ein alternatives Modell das nur aus Application (OSI 7, 6, 5), Transport (4), Network (3) und Host-to-Net (2, 1) besteht. Nur für TCP/IP Netzarchitektur geeignet, keine klare Unterscheidung zwischen Dienst, Protokoll und Schnittstelle.

\paragraph{Hybrides Modell} nach Tanenbaum vereint Application Presentation und Session Layer im Application Layer und ist sonst äquivalent zu OSI.

\section*{Klassifikation}

\begin{description}
    \item[LAN] Örtlich begrenzter Bereich (Raum, Ge\-bäu\-de, \ldots) im Besitz einer Organisation. Übl. hohe Übertragungsleistung mit geringer Fehlerrate, diverse Topologien.
    \item[MAN, WAN] Motropolitan u. Wide Area Networks, große und unkontrollierte Wege (z.B. öff. Straßen).
    \item[Internet] Netzwerk-Verbund der obigen Typen.
\end{description}

Private Netze befinden sich im \enquote*{Besitz} einer Organisation und werden ausschließlich durch diese genutzt. Investitionskosten voll vom Besitzer getragen, keine nennenswerten nutzungsabhängigen Kosten (aber: Strom, Wartung, Support, \ldots). Keine Sicherheitsprobleme von außen, Protokolle müssen nicht beachtet werden.

Öffentliche Netze befinden sich im \enquote*{Besitz} eines Netzbetreibers und werden von dessen Kunden genutzt. Er trägt die Investitionskosten, nutzungsabhängige Kosten fallen für den Kunden an (ggf. als flatrate). Erhebliche Sicherheitsprobleme (interessantes Ziel), (müssen) internationalen Normen/Standards folgen.

VPN verhält sich wie privates Netz, baut aber auf der Infrastruktur eines öffentlichen Netzes auf. Private Protokolle über öffentl. Netze, oft billiger als ein Kabel von Standort A zu Standort B zu legen. In diesem Kontext ist ein Tunnel eine virtuelle p2p Verbindung von zwei Knoten.

\paragraph{???cast} Unicast an Nachbarn, Broadcast an alle, Multicast an manche, Anycast an einen beliebigen.

\paragraph{???plex} Simplex ist unidirektional, Duplex ist bidirektional. Bei Full-Duplex kann gleichzeitg gesendet und empfangen werden, bei Half-Duplex nicht. Multiplex siehe folgenden Abschnitt.

\paragraph{Topologie} ist logische Struktur von Netzen (Stern, Ring, Bus, \ldots). Relevant: gibt es Single Point of Failure (SPoF) bzw. wie beeinträchtigt Ausfall einer Komponente den Rest, wie läuft Uni/Broadcast?

\section*{Multiplex}

Ziel: Aufzeilung einer bestehenden Verbung (Leitung) in mehrere unabhängige Kanäle.

\paragraph{Statisches Aufteilen} \emph{Frequency Division Multiplexing} teilt das Frequenzband in sparierte gentrennt nutzbare Kanäle, jeder Sender/Empfänger hat einen eigenen Frequenzbereich.

\emph{Time Division Multiplexing} 

\end{multicols}

\end{document}
