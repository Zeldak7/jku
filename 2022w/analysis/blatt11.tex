\documentclass{article}
\usepackage[utf8]{inputenc}
\usepackage[ngerman]{babel}

% Convenience improvements
\usepackage{csquotes}
\usepackage{enumitem}
\setlist[enumerate,1]{label={\alph*)}}
\usepackage{amsmath}
\usepackage{amssymb}
\usepackage{mathtools}
\usepackage{tabularx}
\usepackage{hyperref}

% Proper tables and centering for overfull ones
\usepackage{booktabs}
\usepackage{adjustbox}

% Change page/text dimensions, the package defaults work fine
\usepackage{geometry}

\usepackage{parskip}

% Drawings
\usepackage{tikz}
\usepackage{pgfplots}
\pgfplotsset{compat=1.18}

% Adjust header and footer
\usepackage{fancyhdr}
\pagestyle{fancy}
\fancyhead[L]{Analysis --- \textbf{Übungsblatt 11}}
\fancyhead[R]{Laurenz Weixlbaumer (11804751)}
\fancyfoot[C]{}
\fancyfoot[R]{\thepage}
% Stop fancyhdr complaints
\setlength{\headheight}{12.5pt}

\newcommand{\Deltaop}{\, \Delta\, }
\newcommand{\xor}{\, \oplus\, }
\newcommand{\id}{\text{id}}

\begin{document}

\paragraph{Aufgabe 1}

Sei $t = \frac{1}{x}$, nun ist $x \to \infty$ äquivalent zu $t \to 0^+$. Dann gilt
\begin{align*}
    \lim_{x \to \infty} \frac{f(x)}{g(x)} = \lim_{t \to 0^+}\frac{f(1/t)}{g(1/t)} = \lim_{t \to 0^+}\frac{f'(1/t)}{g'(1/t)} = \lim_{t \to 0^+}\frac{-f'(1/t)/t^2}{-g'(1/t)/t^2} = \lim_{t \to 0^+} \frac{f'(1/t)}{g'(1/t)} = \lim_{x \to \infty} \frac{f'(x)}{g'(x)}.
\end{align*}

\paragraph{Aufgabe 2}

\begin{align*}
    \lim_{x \to \infty}\frac{x^2}{e^x} = \lim_{x \to \infty}\frac{2x}{e^x} = \lim_{x \to \infty}\frac{2}{e^x} = \lim_{x \to \infty}\frac{0}{e^x} = 0
\end{align*}

Das Ergebnis ist unabhängig von der Potenz von $x$, $e^x$ wächst schneller als jedes Polynom.

\paragraph{Aufgabe 3}

\begin{align*}
    \lim_{x \to \infty}\frac{\sqrt[7]{x}}{\ln(x)} = \lim_{x \to \infty}\frac{\frac{1}{7x^{6/7}}}{\frac{1}{x}} = \lim_{x \to \infty}\frac{x}{7x^{6/7}} = \infty
\end{align*}

Welche Wurzel genommen wird ist irrelevant.

\paragraph{Aufgabe 4}

\begin{enumerate}
    \item \begin{align*}
        \int(4x^3 - 6x^2 + 3)dx = \int 4x^3 dx - \int 6x^2 dx + \int 3 dx = x^4 - 2x^3 + 3x + c
    \end{align*}

    \item \begin{align*}
        \int(x^5 + 9x^4 - 7x^3 + x^2 - x - 8)dx = \frac{1}{6}x^6 + \frac{9}{5}x^5 - \frac{7}{4}x^4 + \frac{1}{3}x^3 - \frac{1}{2}x^{2} - 8x + c
    \end{align*}

    \item Wir wissen $\int \cos(x)^2 dx = \frac{1}{2} \cos(x)\sin(x) + \frac{1}{2}x + c$ und $\sin(x)^2 = 1 - \cos(x)^2$, somit
    \begin{align*}
        \int \sin(x)^2dx = \int 1dx - \int \cos(x)^2dx = x + c_1 - \frac{1}{2} \cos(x)\sin(x) + \frac{1}{2}x + c_2
    \end{align*}

    \item
    \begin{align*}
        \int \sec(x)^2dx = \tan(x)
    \end{align*}
\end{enumerate}

\paragraph{Aufgabe 5}

\begin{enumerate}
    \item Sei $f(x) = \ln(x)$ und $g(x) = \frac{1}{2}x^2$ mit $f'(x) = \frac{1}{x}$ und $g'(x) = x$, dann 
    \begin{align*}
        \int x\ln(x)dx = \int f(x)g'(x)dx = \frac{1}{2}x^2\ln(x) - \int \frac{1}{x} \frac{1}{2}x^2 dx = \frac{1}{2}x^2\ln(x) - \frac{1}{4}x^2
    \end{align*}

    \item \begin{align*}
        \int (-7x^{-1}+13x^{-2})dx = \int -\frac{7}{x}xd + \int \frac{13}{x^2}dx = 7|\ln(x)| - \frac{13}{x} + c
    \end{align*}

    \item \begin{align*}
        \int (5x^{-5} - 8x^{-4} + 2x^{-1})dx = \int \frac{5}{x^5}dx - \int\frac{8}{x^4}dx + \int\frac{2}{x}dx = -\frac{5}{4x^4} + \frac{8}{3x^3} + 2|\ln(x)| + c
    \end{align*}
\end{enumerate}

\paragraph{Aufgabe 6}

Nachdem wir nur Gravitationskräfte betrachten ist anzunehmen, dass der Ball fallen gelassen (und nicht geworfen) wird.

\begin{enumerate}
    \item Es gilt $a(t) = v'(t)$ bzw. $v(t) = \int a(t) = gx + c$. Das $c$ repräsentiert die Initialgeschwindigkeit die unbekannt ist, im gegebenen Beispiel ist $c = 0$ anzunehmen.
    
    \item Es gilt $v(t) = gx + c_1 = h'(t)$ bzw. $h(t) = \int(gx + c_1)dx = \frac{1}{2}gx^2 + c_1x + c_2$.
    
    \item Gemäß der obigen Funktion der Höhe des Balls erwarten wir eine Parabel.
\end{enumerate}

\end{document}
