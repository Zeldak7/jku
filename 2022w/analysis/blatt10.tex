\documentclass{article}
\usepackage[utf8]{inputenc}
\usepackage[ngerman]{babel}

% Convenience improvements
\usepackage{csquotes}
\usepackage{enumitem}
\setlist[enumerate,1]{label={\alph*)}}
\usepackage{amsmath}
\usepackage{amssymb}
\usepackage{mathtools}
\usepackage{tabularx}
\usepackage{hyperref}

% Proper tables and centering for overfull ones
\usepackage{booktabs}
\usepackage{adjustbox}

% Change page/text dimensions, the package defaults work fine
\usepackage{geometry}

\usepackage{parskip}

% Drawings
\usepackage{tikz}
\usepackage{pgfplots}
\pgfplotsset{compat=1.18}

% Adjust header and footer
\usepackage{fancyhdr}
\pagestyle{fancy}
\fancyhead[L]{Analysis --- \textbf{Übungsblatt 10}}
\fancyhead[R]{Laurenz Weixlbaumer (11804751)}
\fancyfoot[C]{}
\fancyfoot[R]{\thepage}
% Stop fancyhdr complaints
\setlength{\headheight}{12.5pt}

\newcommand{\Deltaop}{\, \Delta\, }
\newcommand{\xor}{\, \oplus\, }
\newcommand{\id}{\text{id}}

\begin{document}

\paragraph{Aufgabe 1.}

\begin{align*}
    f(x) = a(0) + a(1)x + a(2)x^2 + a(3)x^3 + \cdots
\end{align*}
\begin{align*}
    g(x) &= 3\left(a(0) + a(1)x + a(2)x^2 + a(3)x^3 + \cdots\right) + \left(a(0) + a(1)x + a(2)x^2 + a(3)x^3 + \cdots\right)^2 \\
    &= 3a(0) + 3a(1)x + 3a(2)x^2 + 3a(3)x^3 + \cdots + a(0)^2 + a(1)^2x^2 + a(2)^2x^4 + a(3)^2x^6 + (\cdots)^2 \\
    &= \underbrace{(3a(0) + a(0)^2)}_{\text{1st}}x^0 + \underbrace{3a(1)}_{\text{2nd}}x + \underbrace{(3a(2) + a(1)^2)}_{\text{3rd}}x^2 + \underbrace{3a(3)}_{\text{4th}}x^3 + \cdots
\end{align*}

\paragraph{Aufgabe 2.}

% \begin{enumerate}
%     \item Wenn $|x| < 5$.

%     \begin{align*}
%         \sum _{n=0}^{\infty } (-1)^n \frac{x^n}{5^n} &= 1-\frac{x}{5}+\frac{x^2}{25}-\frac{x^3}{125}+\frac{x^4}{625}-\frac{x^5}{3125}+\cdots = \frac{5}{x+5}\\
%         % \sum _{n=0}^{\infty } \frac{(-1)^n 5^n}{x^n} &= \frac{x^5}{3125}-\frac{x^4}{625}+\frac{x^3}{125}-\frac{x^2}{25}+\frac{x}{5} - \cdots \\
%     \end{align*}
% \end{enumerate}

\paragraph{Aufgabe 3.} Satz 7.14. kann nicht verwendet werden weil $\sin' = \cos$ und es gibt $x \in \mathbb{R}$ mit $\cos(x) = 0$.

\paragraph{Aufgabe 4.}

Sei $x_0 \in (a, b)$. Sei $(h_n)$ eine Nullfolge mit $h_n \in (a - x_0, b - x_0) \backslash \{0\}$.

\begin{align*}
    (f + g)'(x_0) &= \lim_{n \to \infty} \frac{f(x_0 + h_n) + g(x_0 + h_n) - f(x_0) - g(x_0)}{h_n} \\
    &= \lim_{n \to \infty} \frac{f(x_0 + h_n) - f(x_0)}{h_n} + \lim_{n \to \infty} \frac{g(x_0 + h_n) - g(x_0)}{h_n} \\
    &= f'(x_0) + g'(x_0)
\end{align*}

\paragraph{Aufgabe 5.}

\begin{align*}
    \left(\frac{1}{g}\right)'(x_0) &= \lim_{n \to \infty} \frac{\left(\frac{1}{g}\right)(x_0 + h_n) - \left(\frac{1}{g}\right)(x_0)}{h_n} = \lim_{n \to \infty} \frac{\frac{1}{g(x_0 + h_n)} - \frac{1}{g(x_0)}}{h_n} = \lim_{n \to \infty} \frac{\frac{g(x_0) - g(x_0 + h_n)}{g(x_0 + h_n)g(x_0)}}{h_n} \\
    &= \lim_{n \to \infty} \frac{g(x_0) - g(x_0 + h_n)}{g(x_0 + h_n)g(x_0)h_n} = -\lim_{n \to \infty} \frac{g(x_0 + h_n) - g(x_0)}{h_n} \cdot \lim_{n \to \infty} \frac{1}{g(x_0 + h_n)g(x_0)} \\
    &= -\lim_{n \to \infty} \frac{g(x_0 + h_n) - g(x_0)}{h_n} \cdot \frac{1}{\lim_{n \to \infty}(g(x_0 + h_n))g(x_0)} \\
    &= -g'(x_0) \frac{1}{g(x_0)^2} = -\frac{g'(x_0)}{g(x_0)^2}
\end{align*}

\begin{align*}
    \left(\frac{f}{g}\right)'(x) = \left(\frac{f(x)}{g(x)}\right)' = \left(f(x)\frac{1}{g(x)}\right)' = f'(x)\frac{1}{g(x)}+f(x)\left(-\frac{g'(x)}{g(x)^2}\right) = \frac{f'(x)g(x) - f(x)g'(x)}{g(x)^2}
\end{align*}

\end{document}
