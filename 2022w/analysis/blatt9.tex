\documentclass{article}
\usepackage[utf8]{inputenc}
\usepackage[ngerman]{babel}

% Convenience improvements
\usepackage{csquotes}
\usepackage{enumitem}
\setlist[enumerate,1]{label={\alph*)}}
\usepackage{amsmath}
\usepackage{amssymb}
\usepackage{mathtools}
\usepackage{tabularx}
\usepackage{hyperref}

% Proper tables and centering for overfull ones
\usepackage{booktabs}
\usepackage{adjustbox}

% Change page/text dimensions, the package defaults work fine
\usepackage{geometry}

\usepackage{parskip}

% Drawings
\usepackage{tikz}
\usepackage{pgfplots}
\pgfplotsset{compat=1.18}

% Adjust header and footer
\usepackage{fancyhdr}
\pagestyle{fancy}
\fancyhead[L]{Analysis --- \textbf{Übungsblatt 9}}
\fancyhead[R]{Laurenz Weixlbaumer (11804751)}
\fancyfoot[C]{}
\fancyfoot[R]{\thepage}
% Stop fancyhdr complaints
\setlength{\headheight}{12.5pt}

\newcommand{\Deltaop}{\, \Delta\, }
\newcommand{\xor}{\, \oplus\, }
\newcommand{\id}{\text{id}}

\begin{document}

\paragraph{Aufgabe 1.}

Wegen $0 \leq a_k$ ist $\sum_{k = 1}^{\infty} a_k$ monoton wachsend. Nach Annahme divergiert $\sum_{k = 1}^{\infty} a_k$, somit ist sie unbeschr\"ankt. Wegen $a_k \leq b_k$ gilt $\sum_{k = 1}^{\infty} a_k \leq \sum_{k = 1}^{\infty} b_k$, damit muss auch $b_k$ unbeschr\"ankt und somit divergent sein.

\paragraph{Aufgabe 2.}

Nach Annahme ist $\sum_{k = 1}^{\infty} |a_k|$ konvergent. Demnach muss auch $\sum_{k = 1}^{\infty} 2|a_k|$ konvergent sein. Wegen $0 \leq a_k + |a_k| \leq 2|a_k|$ ist gem\"a\ss\ des Majorantenkriteriums $\sum_{k = 1}^{\infty} (a_k + |a_k|)$ konvergent. Dann ist auch $\sum_{k = 1}^{\infty} a_k = \sum_{k = 1}^{\infty} (a_k + |a_k|) - \sum_{k = 1}^{\infty} |a_k|$ konvergent, weil es die Differenz zweier konvergenter Reihen ist.

\paragraph{Aufgabe 3.}

\begin{enumerate}
    \item Wegen $\lim_{k \to \infty} \frac{2k^2 + 1}{3k(k + 1)} = \frac{2}{3} \neq 0$ ist die Reihe divergent.
    
    \item
    %Es gilt
    %\begin{align*}
    %    \lim \left(\frac{\frac{\log (n + 1)}{n + 1}}{\frac{\log(n)}{n}}\right) = 1
    %\end{align*}
    %und somit ist die Reihe gem\"a\ss/ des Quotientenkriteriums divergent.

    \item 
    
    \item (Harmonische Reihe\footnote{\url{https://web.williams.edu/Mathematics/lg5/harmonic.pdf}}.) Angenommen die Reihe konvergiert mit
    \begin{align*}
        H = 1 + \frac{1}{2} + \frac{1}{3} + \frac{1}{4} + \frac{1}{5} + \frac{1}{6} + \frac{1}{7} + \frac{1}{8} + \cdots
    \end{align*}
    Dann ist mit 
    \begin{align*}
        H &\geq 1 + \frac{1}{2} + \frac{1}{4} + \frac{1}{4} + \frac{1}{6} + \frac{1}{6} + \frac{1}{8} + \frac{1}{8} + \cdots \\
        &\geq 1 + \frac{1}{2} + \frac{1}{2} + \frac{1}{3} + \frac{1}{4} + \cdots \\
        &\geq \frac{1}{2} + H
    \end{align*}
    ein Widerspruch gegeben.
\end{enumerate}

\paragraph{Aufgabe 4.}

\begin{align*}
    \sum \frac{4}{4k^2 - 1} = 4 \sum\frac{1}{4k^2 - 1} \\
    \frac{1}{4k^2 - 1} = \frac{1}{(2k - 1)(2k + 1)} = \frac{1}{4k - 2} - \frac{1}{4k + 2} \\
    \sum \left(\frac{1}{4k - 2} - \frac{1}{4k + 2}\right) \\
    = \left(\frac{1}{2} - \frac{1}{6}\right) + \left(\frac{1}{6} - \frac{1}{10}\right) + \left(\frac{1}{10} - \frac{1}{14}\right) + \cdots + \left(\frac{1}{4n - 6} - \frac{1}{4n - 2}\right) + \left(\frac{1}{4n - 2} - \frac{1}{4n + 2}\right) \\
    = \frac{1}{2} - \frac{1}{4n + 2}\\
    \lim \left(\frac{1}{n} - \frac{1}{4n + 2}\right) = \frac{1}{2} \\
    \sum \frac{4}{4k^2 - 1} = 4 \cdot \frac{1}{2} = 2
\end{align*}

\paragraph{Aufgabe 5.}

Seien $\sum a_k$ und $\sum b_k$ zwei Reihen mit den Partialsummen $(s_n)$ und $(t_n)$ derart, dass es ein $N$ mit $a_k = b_k$ f\"ur alle $k \geq N$ gibt. Dann gilt
\begin{align*}
    s_n = \sum_{k = 1} a_k = a_1 + a_2 + \cdots + a_{N - 1} + \sum_{k = N} a_k
\end{align*}
und
\begin{align*}
    t_n &= \sum_{k = 1} b_k = b_1 + b_2 + \cdots + b_{N - 1} + \sum_{k = N} a_k \\
    &= s_n - (a_1 + a_2 + \cdots + a_{N - 1}) + (b_1 + b_2 + \cdots + b_{N - 1}).
\end{align*}
Angenommen $\sum b_k$ konvergiert, es gibt also $\lim t_n = t$. Dann sei
\begin{align*}
    c = (a_1 + a_2 + \cdots + a_{N - 1}) + (b_1 + b_2 + \cdots + b_{N - 1})
\end{align*}
eine von $n$ unabh\"angige Konstante. Nun gilt
\begin{align*}
    t_n &= s_n - C \\
    \lim t_n &= \lim s_n - C \\
    \lim s_n &= t + C,
\end{align*}
also konvergiert auch $(s_n)$ bzw. $\sum a_k$.

\paragraph{Aufgabe 6.}

\begin{enumerate}
    \item Wegen $\lim \frac{1}{\sqrt[n]{n}} = 1 \neq 0$ ist die Reihe divergent.
    
    \item Wegen $\lim \left(n-\frac{\left(n^2+n+1\right)}{n}\right) = -1 \neq 0$ ist die Reihe divergent.
    
    \item Wegen $\lim |\frac{n!}{n^{2n}}|^{\frac{1}{n}} = \lim \frac{\sqrt[n]{n!}}{n^2} = 0 < 1$ ist die Reihe konvergent.
\end{enumerate}

\paragraph{Aufgabe 7.}

\begin{enumerate}
    \item \begin{align*}
        \lim \left|\frac{(n+1)!x^{n+1}}{n!x^n}\right| = \lim \left|\frac{(n+1)!x}{n!}\right| = |x| \lim \left|\frac{(n+1)!}{n!}\right| = |x| \infty
    \end{align*}
    Konvergenzradius ist 0.

    \item \begin{align*}
        \lim \left|\frac{(n + 2)(n + 3)\left(\frac{x}{2}\right)^{n + 1}}{(n + 1)(n + 2)\left(\frac{x}{2}\right)^{n}}\right| = \lim \left|\frac{(n + 3)\left(\frac{x}{2}\right)}{n + 1}\right| = \left|\frac{x}{2}\right|\lim\left|\frac{n + 3}{n + 1}\right| = \left|\frac{x}{2}\right|
    \end{align*}
    Konvergenzradius ist 2.

    \item \begin{align*}
        \lim \left|\frac{x^n}{(\ln(n))^n}\right|^{\frac{1}{n}} = \lim \left|\frac{x}{\ln(n)}\right| = |x|\lim \left|\frac{1}{\ln(n)}\right| = 0
    \end{align*}
    Konvergenzradius ist $\infty$.
\end{enumerate}

\paragraph{Aufgabe 8.}

\end{document}
