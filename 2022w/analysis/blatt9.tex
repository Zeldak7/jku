\documentclass{article}
\usepackage[utf8]{inputenc}
\usepackage[ngerman]{babel}

% Convenience improvements
\usepackage{csquotes}
\usepackage{enumitem}
\setlist[enumerate,1]{label={\alph*)}}
\usepackage{amsmath}
\usepackage{amssymb}
\usepackage{mathtools}
\usepackage{tabularx}
\usepackage{hyperref}

% Proper tables and centering for overfull ones
\usepackage{booktabs}
\usepackage{adjustbox}

% Change page/text dimensions, the package defaults work fine
\usepackage{geometry}

\usepackage{parskip}

% Drawings
\usepackage{tikz}
\usepackage{pgfplots}

% Adjust header and footer
\usepackage{fancyhdr}
\pagestyle{fancy}
\fancyhead[L]{Analysis --- \textbf{Übungsblatt 9}}
\fancyhead[R]{Laurenz Weixlbaumer (11804751)}
\fancyfoot[C]{}
\fancyfoot[R]{\thepage}
% Stop fancyhdr complaints
\setlength{\headheight}{12.5pt}

\newcommand{\Deltaop}{\, \Delta\, }
\newcommand{\xor}{\, \oplus\, }
\newcommand{\id}{\text{id}}

\begin{document}

\paragraph{Aufgabe 1.}

Wegen $0 \leq a_k$ ist $\sum_{k = 1}^{\infty} a_k$ monoton wachsend. Nach Annahme divergiert $\sum_{k = 1}^{\infty} a_k$, somit ist sie unbeschr\"ankt. Wegen $a_k \leq b_k$ gilt $\sum_{k = 1}^{\infty} a_k \leq \sum_{k = 1}^{\infty} b_k$, damit muss auch $b_k$ unbeschr\"ankt und somit divergent sein.

\paragraph{Aufgabe 2.}

Nach Annahme ist $\sum_{k = 1}^{\infty} |a_k|$ konvergent. Demnach muss auch $\sum_{k = 1}^{\infty} 2|a_k|$ konvergent sein. Wegen $0 \leq a_k + |a_k| \leq 2|a_k|$ ist gem\"a\ss\ des Majorantenkriteriums $\sum_{k = 1}^{\infty} (a_k + |a_k|)$ konvergent. Dann ist auch $\sum_{k = 1}^{\infty} a_k = \sum_{k = 1}^{\infty} (a_k + |a_k|) - \sum_{k = 1}^{\infty} |a_k|$ konvergent, weil es die Differenz zweier konvergenter Reihen ist.

\paragraph{Aufgabe 3.}

\begin{enumerate}
    \item Wegen $\lim_{k \to \infty} \frac{2k^2 + 1}{3k(k + 1)} = \frac{2}{3} \neq 0$ ist die Reihe divergent.
    
    \item
    %Es gilt
    %\begin{align*}
    %    \lim \left(\frac{\frac{\log (n + 1)}{n + 1}}{\frac{\log(n)}{n}}\right) = 1
    %\end{align*}
    %und somit ist die Reihe gem\"a\ss/ des Quotientenkriteriums divergent.

    \item 
    
    \item (Harmonische Reihe\footnote{\url{https://web.williams.edu/Mathematics/lg5/harmonic.pdf}}.) Angenommen die Reihe konvergiert mit
    \begin{align*}
        H = 1 + \frac{1}{2} + \frac{1}{3} + \frac{1}{4} + \frac{1}{5} + \frac{1}{6} + \frac{1}{7} + \frac{1}{8} + \cdots
    \end{align*}
    Dann ist mit 
    \begin{align*}
        H &\geq 1 + \frac{1}{2} + \frac{1}{4} + \frac{1}{4} + \frac{1}{6} + \frac{1}{6} + \frac{1}{8} + \frac{1}{8} + \cdots \\
        &\geq 1 + \frac{1}{2} + \frac{1}{2} + \frac{1}{3} + \frac{1}{4} + \cdots \\
        &\geq \frac{1}{2} + H
    \end{align*}
    ein Widerspruch gegeben.
\end{enumerate}

\end{document}
