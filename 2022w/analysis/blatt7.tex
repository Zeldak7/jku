\documentclass{article}
\usepackage[utf8]{inputenc}
\usepackage[ngerman]{babel}

% Convenience improvements
\usepackage{csquotes}
\usepackage{enumitem}
\setlist[enumerate,1]{label={\alph*)}}
\usepackage{amsmath}
\usepackage{amssymb}
\usepackage{mathtools}
\usepackage{tabularx}
\usepackage{hyperref}

% Proper tables and centering for overfull ones
\usepackage{booktabs}
\usepackage{adjustbox}

% Change page/text dimensions, the package defaults work fine
\usepackage{geometry}

\usepackage{parskip}

% Drawings
\usepackage{tikz}
\usepackage{pgfplots}

% Adjust header and footer
\usepackage{fancyhdr}
\pagestyle{fancy}
\fancyhead[L]{Analysis --- \textbf{Übungsblatt 7}}
\fancyhead[R]{Laurenz Weixlbaumer (11804751)}
\fancyfoot[C]{}
\fancyfoot[R]{\thepage}
% Stop fancyhdr complaints
\setlength{\headheight}{12.5pt}

\newcommand{\Deltaop}{\, \Delta\, }
\newcommand{\xor}{\, \oplus\, }
\newcommand{\id}{\text{id}}

\begin{document}

\paragraph{Aufgabe 1.} (Sei $\tilde{x} \in \mathbb{R}$ und $(h_n)$ eine Nullfolge.)
\begin{align*}
    \lim f(\tilde{x} + h_n) = \lim |\tilde{x} + h_n| = |\lim(\tilde{x} + h_n)| = |\tilde{x}| = f(\tilde{x})
\end{align*}

\paragraph{Aufgabe 2.}

\paragraph{Aufgabe 3.}
Es gilt
\begin{align*}
    \lim_{n \to x} f(n) &= f(x)\ &\text{weil $f$ stetig in $x$ ist und} \\
    \lim_{n \to f(x)} g(n) &= g(f(x))\ &\text{weil $g$ stetig in $f(x)$ ist.}
\end{align*}
Wir wollen zeigen, dass $\lim_{n \to x} g(f(n)) = g(f(x))$.
\begin{align*}
    \lim_{n \to x} g(f(n)) &= \lim_{n \to f(x)} g(n) \\
    &= g\left(\lim_{n \to f(x)} n\right) & \text{weil $g$ stetig in $f(x)$}\\
    &= g\left(f(x)\right)
\end{align*}

\paragraph{Aufgabe 4.} $f(x) = 0$, überall in $\mathbb{R}$ stetig.

\paragraph{Aufgabe 5.}

Ist eine Funktion $f$ in $x_0$ stetig dann gilt: Für alle $\epsilon > 0$ gibt es $\delta > 0$ derart, dass für alle $x$
\begin{align*}
    |x - x_0| < \delta \quad\Longrightarrow\quad |f(x) - f(x_0)| < \epsilon
\end{align*}
(Definition nach Weierstrass und Jordan.)

Wähle ein $\epsilon \leq 1$. Zwischen zwei rationalen Zahlen gibt es immer eine reelle Zahl, also finden wir für beliebig kleine $\delta$-Umgebungen mindestens eine reelle Zahl die einen Sprung mit Distanz 1 (also $\geq \epsilon$) verursacht. Zwischen zwei reellen Zahlen gibt es immer eine eine rationale Zahl \ldots\ selbes Argument. (Rationale Zahlen liegen dicht in $\mathbb{R}$.)

\paragraph{Aufgabe 6.}
\begin{enumerate}
    \item Die Vorzeichenfunktion ist in $x = 0$ unstetig, sonst stetig. (Siehe Skript.)
    \item \begin{align*}
        f: \mathbb{R} \rightarrow \mathbb{R}, f(x) = \begin{cases}
            x, & \text{wenn $x \in \mathbb{Q}$} \\
            0 & \text{sonst}
        \end{cases}
    \end{align*}
    Klarerweise gilt $\lim_{x \to 0} f(x) = f(0) = 0$ weil selbiges sowohl für $f_1(x) = x$ und $f_2(x) = 0$ gilt. Also ist $f$ stetig in 0.

    Weil aber sowohl die rationalen als auch die irrationalen Zahlen dichte Teilmengen von $\mathbb{R}$ sind, kann es keinen weiteren Punkt geben an dem die Funktion stetig ist. (Siehe Aufgabe 5.)
\end{enumerate}
% https://planetmath.org/functioncontinuousatonlyonepoint

\paragraph{Aufgabe 7.}
Es gilt
\begin{align*}
    f(xy) &= x\, f(y) & f(yx) &= y\, f(x)
\end{align*}
und somit
\begin{align*}
    x\, f(y) &= y\, f(x) \\
    \frac{f(y)}{y} &= \frac{f(x)}{x}.
\end{align*}
Unsere Funktionen sind also notwendigerweise lineare Funktionen $f(x) = cx + d$ für $c, d \in \mathbb{R}$. Klarerweise sind dieses Funktionen wegen
\begin{align*}
    \lim f(x + h_n) = \lim (cx + ch_n + d) = cx + d + \lim ch_n = cx = f(x)
\end{align*}
global stetig.

\paragraph{Aufgabe 8.} Nachdem $f$ stetig in einem Punkt $x'$ ist, gilt 
\begin{align*}
    \lim_{x \to x'} f(x) = f(x').
\end{align*}
Wir wollen zeigen, dass für beliebige $\tilde{x}$ auch $\lim_{x \to \tilde{x}}f(x) = f(\tilde{x})$ gilt.
\begin{align*}
    \lim_{x \to \tilde{x}} f(x) &= \lim_{x \to x'} f(x - x' + \tilde{x}) \\
    &= \lim_{x \to x'} (f(x) - f(x') + f(\tilde{x})) \\
    &= (\lim_{x \to x'}f(x)) - f(x') + f(\tilde{x}) \\
    &= f(x') - f(x') + f(\tilde{x}) \\
    &= f(\tilde{x})
\end{align*}
Ist eine additive Funktion stetig in einem beliebigen Punkt, ist sie auch global stetig.

\end{document}
