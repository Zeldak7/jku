\documentclass{article}
\usepackage[utf8]{inputenc}
\usepackage[ngerman]{babel}

% Convenience improvements
\usepackage{csquotes}
\usepackage{enumitem}
\setlist[enumerate,1]{label={\alph*)}}
\usepackage{amsmath}
\usepackage{amssymb}
\usepackage{mathtools}
\usepackage{tabularx}
\usepackage{hyperref}

% Proper tables and centering for overfull ones
\usepackage{booktabs}
\usepackage{adjustbox}

% Change page/text dimensions, the package defaults work fine
\usepackage{geometry}

\usepackage{parskip}

% Drawings
\usepackage{tikz}
\usepackage{pgfplots}

% Adjust header and footer
\usepackage{fancyhdr}
\pagestyle{fancy}
\fancyhead[L]{Analysis --- \textbf{Übungsblatt 4}}
\fancyhead[R]{Laurenz Weixlbaumer (11804751)}
\fancyfoot[C]{}
\fancyfoot[R]{\thepage}
% Stop fancyhdr complaints
\setlength{\headheight}{12.5pt}

\newcommand{\Deltaop}{\, \Delta\, }
\newcommand{\xor}{\, \oplus\, }
\newcommand{\id}{\text{id}}

\begin{document}

\paragraph{Aufgabe 1.} Zu zeigen ist
\begin{align*}
    |a + b| \leq |a| + |b|.
\end{align*}
Es gilt $|x| = \max(x, -x)$. Dann ist $|a + b| = \max(a + b, -(a + b)) = \max(a + b, -a - b)$. Weiters, nachdem $\pm x \leq |x|$, 
\begin{align*}
    a + b &\leq |a| + b \leq |a| + |b|, \quad \text{und} \\
    -a - b &\leq |a| - b \leq |a| + |b|.
\end{align*}

\paragraph{Aufgabe 2.}
\begin{align*}
    0 &\leq (x - y)^2 &\text{Quadrat einer reellen Zahl ist $\geq 0$} \\
    0 &\leq x^2 - 2xy + y^2 \\
    4xy &\leq x^2 + 2xy + y^2 \\
    4xy &\leq (x + y)^2 \\
    \sqrt{xy} &\leq \frac{x + y}{2} &\text{Quadratwurzel, dann Division durch 2} \\
\end{align*}

\paragraph{Aufgabe 3.}

\begin{enumerate}[]
    \item $\frac{2}{\sqrt{5}}$
    \item $\frac{\sqrt{5}}{3}$
    \item $\frac{1}{\sqrt{2}}$
\end{enumerate}

\paragraph{Aufgabe 4.}

Wir haben $\lim_{n \to \infty} |a_n| = 0$. Für beliebiges $\epsilon > 0$ gibt es ein $n_\epsilon$ mit $|0 - |a_n|| < \epsilon$ für $n > n_\epsilon$. Weil $|0 - |a_n|| = ||a_n|| = |a_n| = |0 - a_n|$ gilt auch $|0 - a_n| < \epsilon$ und somit $\lim_{n \to \infty} a_n = 0$.

\paragraph{Aufgabe 5.}

Sei $(a_n)$ eine Folge die gegen $a$ konvergiert. Für ein $\epsilon > 0$ gibt es also $n_\epsilon$ derart, dass
\begin{align*}
    |a_n - a| < \epsilon& \quad \Longrightarrow \quad |a_n| < |a| + \epsilon &\text{(für alle $n \geq n_\epsilon$)}
\end{align*}
Für alle $n \geq n_\epsilon$ haben wir nun also $|a| + \epsilon$ als obere Schranke. Es gibt endlich viele $1 \leq n < n_\epsilon$ also können wir durch $a_m = \max(a_1, a_2, \ldots, a_{n_\epsilon - 1})$ eine obere Schranke für $n < n_\epsilon$ finden. Die obere Schranke für $(a_n)$ ist dann $\max(a_m, |a| + 1)$.

Die untere Schranke lässt sich analog konstruieren, unter Verwendung von $\min$ und $|a| - \epsilon$.

\paragraph{Aufgabe 6.}

Angenommen die gegebene Folge $(a_n)$ konvergiert gegen ein $a$. Für alle $\epsilon > 0$ muss es also ein $n_\epsilon$ geben, mit $|a - a_n| < \epsilon$ für $n \geq n_\epsilon$. Sei $\epsilon = \frac{1}{2}$. Es gibt kein $a$ derart, dass für alle $n > n_\epsilon$ gilt
\begin{align*}
    |a - 1| < \frac{1}{2} \quad \text{und} \quad |a + 1| < \frac{1}{2} \\
    |a - 1| + |a + 1| < 1
\end{align*}

\paragraph{Aufgabe 7.}

Zu zeigen ist, dass $\lim_{n \to \infty} \frac{1}{\sqrt{n}} = 0$ gilt.

\begin{enumerate}
    \item Sei $\epsilon > 0$ und $n_\epsilon = \frac{2}{\epsilon^2}$. Dann gilt
    \begin{align*}
        |a - a_n| = \left|0 - \frac{1}{\sqrt{n}}\right| = \frac{1}{\sqrt{n}}
        = \frac{1}{\sqrt{\frac{2}{\epsilon^2}}}
        = \frac{1}{\frac{\sqrt{2}}{e}}
        = \frac{e}{\sqrt{2}} < e
    \end{align*}
\end{enumerate}

\paragraph{Aufgabe 8.}

Sei $(a_n)$ eine gegen $a$ konvergierende Folge. Für alle $\epsilon > 0$ gibt es also ein $n_\epsilon$ ab dem für alle $n \geq n_\epsilon$ 
\begin{align*}
    |a - a_n| &< \epsilon
\end{align*}
Dann gilt auch
\begin{align*}
    \lambda|a - a_n| &< \lambda\epsilon \\
    |\lambda a - \lambda a_n| &< \lambda\epsilon
\end{align*}
für alle $\epsilon > 0$ und ab jeweiligen $n_\epsilon$. Somit konvergiert $(\lambda a_n)$ gegen $\lambda a$ wenn $(a_n)$ gegen $a$ konvergiert.

\end{document}
