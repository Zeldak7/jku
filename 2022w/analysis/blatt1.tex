\documentclass{article}
\usepackage[utf8]{inputenc}
\usepackage[ngerman]{babel}

% Convenience improvements
\usepackage{csquotes}
\usepackage{enumitem}
\setlist[enumerate,1]{label={\alph*)}}
\usepackage{amsmath}
\usepackage{amssymb}
\usepackage{mathtools}
\usepackage{tabularx}

% Proper tables and centering for overfull ones
\usepackage{booktabs}
\usepackage{adjustbox}

% Change page/text dimensions, the package defaults work fine
\usepackage{geometry}

\usepackage{parskip}

% Drawings
\usepackage{tikz}
\usepackage{pgfplots}

% Adjust header and footer
\usepackage{fancyhdr}
\pagestyle{fancy}
\fancyhead[L]{Analysis --- \textbf{Übungsblatt 1}}
\fancyhead[R]{Laurenz Weixlbaumer (11804751)}
\fancyfoot[C]{}
\fancyfoot[R]{\thepage}
% Stop fancyhdr complaints
\setlength{\headheight}{12.5pt}

\newcommand{\Deltaop}{\, \Delta\, }
\newcommand{\xor}{\, \oplus\, }
\newcommand{\id}{\text{id}}

\begin{document}

\paragraph{Aufgabe 1.} Für $n = 0$ gilt
\begin{align*}
    \sum_{i = 0}^{0}r^i = r^0 = 1 = \frac{1 - r}{1 - r},
\end{align*}
Wir nehmen nun an, dass
\begin{align*}
    \sum_{i = 0}^n r^i = \frac{1 - r^{n + 1}}{1 - r},
    % \quad\Longrightarrow\quad \sum_{i = 0}^{n + 1} r^i = \frac{1 - r^{n + 2}}{1 - r}
\end{align*}
und zeigen
\begin{align*}
    \sum_{i = 0}^{n + 1} r^i = \frac{1 - r^{n + 2}}{1 - r}.
\end{align*}
Nachdem
\begin{align*}
    \sum_{i = 0}^{n + 1} r^i = r^{n + 1} + \sum_{i = 0}^{n} r^i
\end{align*}
muss weiters
\begin{align*}
    \frac{1 - r^{n + 2}}{1 - r} &= r^{n + 1} + \frac{1 - r^{n + 1}}{1 - r} \\
    &= \frac{r^{n + 1}(1 - r)}{1 - r} + \frac{1 - r^{n + 1}}{1 - r} \\
    &= \frac{r^{n + 1}(1 - r) + 1 - r^{n + 1}}{1 - r} \\
    &= \frac{r^{n + 1} - r^{n + 2} + 1 - r^{n + 1}}{1 - r} \\
    &= \frac{1 - r^{n + 2}}{1 - r}.
\end{align*}

\paragraph{Aufgabe 2.} \begin{enumerate}
    \item Für $n = 1$ haben wir $\sum_{k = 1}^{1} (2k - 1) = 2 - 1 = 1 = 1^2$. Unter Annahme von $\sum^n_{k = 1}(2k - 1) = n^2$ soll nun $\sum^{n + 1}_{k = 1}(2k - 1) = (n + 1)^2$ gelten. Wir haben
    \begin{align*}
        \sum^{n + 1}_{k = 1}(2k - 1) = 2(n + 1) - 1 + \sum^{n}_{k = 1}(2k - 1) = 2n + 1 + \sum^{n}_{k = 1}(2k - 1)
    \end{align*}
    und zeigen nun (siehe binomische Formel)
    \begin{align*}
        (n + 1)^2 &= 2n + 1 + n^2 \\
        n^2 + 2n + 1 &= 2n + 1 + n^2 \\
    \end{align*}

    \item Für $n = 1$ haben wir $\sum_{k = 1}^{1}k^3 = 1 = (\sum_{k = 1}^1k)^2$. Unter Annahme von $\sum_{k = 1}^n k^3 = (\sum^n_{k = 1}k)^2$ soll nun $\sum^{n + 1}_{k = 1}k^3 = (\sum^{n + 1}_{k = 1}k)^2$ gelten. 
    \begin{align*}
        \sum^{n + 1}_{k = 1}k^3 &= \left(\sum^{n + 1}_{k = 1}k\right)^2 \\
        (n + 1)^3 + \sum^{n}_{k = 1}k^3 &= \left(\sum^{n + 1}_{k = 1}k\right)^2 \\
        (n + 1)^3 + \left(\sum^n_{k = 1}k\right)^2 &= \left(\sum^{n + 1}_{k = 1}k\right)^2 \\
        (n + 1)^3 + \left(\frac{n(n + 1)}{2}\right)^2 &= \left(\frac{(n + 1)(n + 2)}{2}\right)^2 \\
        (n + 1)^3 + \frac{(n(n + 1))^2}{4} &= \frac{((n + 1)(n + 2))^2}{4} \\
        (n + 1)^3 + \frac{n^2 + 2 n^3 + n^4}{4} &= \frac{4 + 12 n + 13 n^2 + 6 n^3 + n^4}{4} \\
        1 + 3 n + 3 n^2 + n^3 + \frac{n^2 + 2 n^3 + n^4}{4} &= \frac{4 + 12 n + 13 n^2 + 6 n^3 + n^4}{4} \\
        4 + 12 n + 13 n^2 + 6 n^3 + n^4 &= 4 + 12 n + 13 n^2 + 6 n^3 + n^4
    \end{align*}
\end{enumerate}

\paragraph{Aufgabe 3.} Für $n = 1$ gilt $(1 + x)^1 = 1 + x \geq 1 + x$. Unter Annahme von $(1+x)^n \geq 1 + nx$ soll nun $(1+x)^{n + 1} \geq 1 + (n + 1)x$ gelten.
\begin{align*}
    (1+x)^{n + 1} &\geq 1 + (n + 1)x \\
    (1+x)(1+x)^{n} &\geq 1 + nx + x \\
    (1+x)(1 + nx) &\geq 1 + nx + x \quad\text{(Einsetzen eines $\leq$ Wertes)}\\
    1 + nx + x + n(x^2) &\geq 1 + nx + x \\
    n(x^2) &\geq 0
\end{align*}
Letzterer Ausdruck ist wahr nachdem $x^2$ sicher positiv oder Null ist und $n \geq 1$.

\paragraph{Aufgabe 4.}
\begin{enumerate}
    \item Seien $a$ und $b$ neutrale Elemente bezüglich der Addition in $K$ und $x$ und $y$ neutrale Elemente bezüglich der Multiplikation in $K$.
    \begin{align*}
        a + b = a \quad\text{und}\quad a + b = b, \quad\text{demzufolge $a = b$.} \\
        xy = x \quad\text{und}\quad xy = y, \quad\text{demzufolge $x = y$.}
    \end{align*}
    \item Seien $x$ und $y$ additive Inverse von $a$ in $K$. (Verwendet Assoziativität der Addition.)
    \begin{align*}
        x = x + 0 = x + (a + y) = (x + a) + y = 0 + y = y
    \end{align*}
    \item Seien $x$ und $y$ multiplikative Inverse von $a$ in $K$ mit $a \neq 0$. (Verwendet Assoziativität der Multiplikation.)
    \begin{align*}
        x = x \cdot 1 = x \cdot (a \cdot y) = (x \cdot a) \cdot y = 1 \cdot y = y
    \end{align*}
\end{enumerate}

\paragraph{Aufgabe 5.}
\begin{enumerate}
    \item Zu zeigen ist $c^{-1}d^{-1} = (cd)^{-1}$, anders formuliert
    \begin{align*}
        (c^{-1}d^{-1})(cd) = cc^{-1}dd^{-1} = 1 \cdot 1 = 1.
    \end{align*}
    (Verwendet Assoziativität und Kommutativität der Multiplikation in der ersten Umformung, und die Eindeutigkeit des multiplikativen Inverses für die Schlussfolgerung.)
    \item Zu zeigen ist
    \begin{align*}
        \frac{a}{c} + \frac{b}{d} = ac^{-1} + bd^{-1} &= ac^{-1}dd^{-1} + bd^{-1}cc^{-1} \\
        &= c^{-1}d^{-1}(ad + bc) \\
        &= (ad + bc)(cd)^{-1} = \frac{ad + bc}{cd}
    \end{align*}
    (Verwendet Neutralelement, \enquote{Herausheben} und 5a.
\end{enumerate}

\paragraph{Aufgabe 6.}
\begin{enumerate}[label=\arabic*)]
    \setcounter{enumi}{6}
    \item Es gibt ein $\frac{1}{1} \in \mathbb{Q}$ mit $\frac{1}{1} \cdot \frac{a}{b} = \frac{a}{b}$ für alle $\frac{a}{b} \in \mathbb{Q}$.
    \begin{align*}
        \frac{1}{1} \cdot \frac{a}{b} = \frac{1 \cdot a}{1 \cdot b} = \frac{a}{b}
    \end{align*}
    \item Für alle $\frac{a}{b} \neq 0 \in \mathbb{Q}$ gibt es ein $\frac{x}{y} \in \mathbb{Q}$ mit $\frac{a}{b} \cdot \frac{x}{y} = 1$. Sei $\frac{x}{y} = \frac{b}{a}$.
    \begin{align*}
        \frac{a}{b} \cdot \frac{b}{a} = \frac{ab}{ba} = ab(ba)^{-1} = aa^{-1}bb^{-1} = 1
    \end{align*}
    \item Für $\frac{p_1}{q_1}, \frac{p_2}{q_2}, \frac{p_3}{q_3} \in \mathbb{Q}$ gilt
    \begin{align*}
        \frac{p_1}{q_1} \cdot \left(\frac{p_2}{q_2} + \frac{p_3}{q_3}\right) = \frac{p_1}{q_1} \cdot \frac{p_2q_3 + p_3q_2}{q_2q_3} = \frac{p_1}{q_1} \cdot \left(\frac{p_2q_3}{q_2q_3} + \frac{p_3q_2}{q_2q_3}\right) = \frac{p_1}{q_1} \cdot \left(\frac{p_2}{q_2} + \frac{p_3}{q_3}\right)
    \end{align*}
\end{enumerate}

% \paragraph{Aufgabe 7.} Für die Assoziativität gilt es zu zeigen, dass
% \begin{align*}
%     \sum_{n = 0}^{\infty}a_nx^n \cdot \left(\sum_{n = 0}^{\infty}b_nx^n \cdot \sum_{n = 0}^{\infty}c_nx^n\right) &= \sum_{n = 0}^{\infty}a_nx^n \cdot \sum_{n = 0}^{\infty}\left(\sum_{k = 0}^{n} b_k c_{n - k}\right)x^n \\
%     %&= \sum_{n = 0}^{\infty}\left(\sum_{k = 0}^{n}a_\right)
%     %&= \left(\sum_{n = 0}^{\infty}a_nx^n \cdot \sum_{n = 0}^{\infty}b_nx^n\right) \cdot \sum_{n = 0}^{\infty}c_nx^n
% \end{align*}

\paragraph{Aufgabe 8.} Angenommen $\sqrt{3} \in \mathbb{Q}$, dann gibt es $x, y \in \mathbb{Z}$ mit $\sqrt{3} = \frac{x}{y}$ wobei angenommen werden kann, dass $\text{ggT}(x, y) = 1$. Aus der Annahme folgt
\begin{align*}
    3 &= \frac{x^2}{y^2} \\
    3y^2 &= x^2 \quad &\text{$x$ ist teilbar durch 3} \\
    3y^2 &= (3k)^2 \quad &\text{f\"ur ein bestimmtes $k \in \mathbb{Z}$} \\
    3y^2 &= 9k^2 \\
    y^2 &= 3k^2 \quad &\text{$y$ ist ebenfalls teilbar durch 3, Widerspruch} \\
\end{align*}

\end{document}
