\documentclass{article}
\usepackage[utf8]{inputenc}
\usepackage[ngerman]{babel}

% Convenience improvements
\usepackage{csquotes}
\usepackage{enumitem}
\setlist[enumerate,1]{label={\alph*)}}
\usepackage{amsmath}
\usepackage{amssymb}
\usepackage{mathtools}
\usepackage{tabularx}

% Proper tables and centering for overfull ones
\usepackage{booktabs}
\usepackage{adjustbox}

% Change page/text dimensions, the package defaults work fine
\usepackage{geometry}

\usepackage{parskip}

% Drawings
\usepackage{tikz}

% Adjust header and footer
\usepackage{fancyhdr}
\pagestyle{fancy}
\fancyhead[L]{Algebra --- \textbf{Exercise Sheet 1}}
\fancyhead[R]{Laurenz Weixlbaumer (11804751)}
\fancyfoot[C]{}
\fancyfoot[R]{\thepage}
% Stop fancyhdr complaints
\setlength{\headheight}{12.5pt}

\newcommand{\Deltaop}{\, \Delta\, }
\newcommand{\xor}{\, \oplus\, }

\begin{document}

\paragraph{Exercise 1.}

We have
\begin{align*}
    l &= \begin{pmatrix}
        1 & 2 & 3 \\
        2 & 3 & 1
    \end{pmatrix}&
    r &= \begin{pmatrix}
        1 & 2 & 3 \\
        3 & 1 & 2
    \end{pmatrix}&
    s &= \begin{pmatrix}
        1 & 2 & 3 \\
        1 & 3 & 2
    \end{pmatrix}\\
    e = s \circ s &= \begin{pmatrix}
        1 & 2 & 3 \\
        1 & 2 & 3
    \end{pmatrix}&
    t = l \circ s &= \begin{pmatrix}
        1 & 2 & 3 \\
        2 & 1 & 3
    \end{pmatrix}&
    u = s \circ l &= \begin{pmatrix}
        1 & 2 & 3 \\
        3 & 2 & 1
    \end{pmatrix}.
\end{align*}

\begin{enumerate}
    \item Cayley table.
    \begin{center}
        \begin{tabular}{c | c c c c c c}
            $\circ$ & $e$ & $l$ & $r$ & $s$ & $t$ & $u$ \\\midrule
            $e$ & $e$ & $l$ & $r$ & $s$ & $t$ & $u$ \\
            $l$ & $l$ & $r$ & $e$ & $t$ & $u$ & $s$ \\
            $r$ & $r$ & $e$ & $l$ & $u$ & $s$ & $t$ \\
            $s$ & $s$ & $u$ & $t$ & $e$ & $r$ & $l$ \\
            $t$ & $t$ & $s$ & $u$ & $l$ & $e$ & $r$ \\
            $u$ & $u$ & $t$ & $s$ & $r$ & $l$ & $e$ \\
        \end{tabular}
    \end{center}

    \item Not commutative, the table would have to be symmetrical alongside the main diagonal. For example $l \circ s \neq s \circ l$.
    
    There is a neutral element, the identity permutation $e$.

    Each element has an inverse since every row and every column of the table contains $e$. (Meaning that for each element, there exists another which \enquote{turns it} into $e$.)
    
    \item Semigroup because function composition is associative. (We have $a \circ (b \circ c) = (a \circ b) \circ c$, see discrete structures notes.)
    
    Monoid because there is a neutral element.

    Group because each element has an inverse.

    Not an abelian group, ring, etc. because it is not commutative.
\end{enumerate}

\pagebreak
\paragraph{Exercise 2.} To show that $(\mathcal{P}(X), \Delta)$ is an abelian group we show that it
\begin{enumerate}
    \item is associative by showing $x \Deltaop (y \Deltaop z) = (x \Deltaop y) \Deltaop z$.
    
    The expression $a \in x \Deltaop y$ tells us that $a$ is either in $x$ or $y$, but not in both of them. The expression $p = u \xor v$ tells us that $p$ is true if either $u$ or $v$ is true, but not both of them. The truth value of $a \in x \Deltaop y$ is thus equal to $a \in x \xor a \in y$.

    Assume that $a \in x \Deltaop (y \Deltaop z)$. We can transform this to
    \begin{align*}
        a \in x \Deltaop (y \Deltaop z) &\Longleftrightarrow a \in x \xor (a \in y \Deltaop z) \\
        &\Longleftrightarrow a \in x \xor (a \in y \xor a \in z) \Longleftrightarrow a \in x \xor a \in y \xor a \in z.
    \end{align*}
    
    Assume that $a \in (x \Deltaop y) \Deltaop z$. We can transform this to
    \begin{align*}
        a \in (x \Deltaop y) \Deltaop z &\Longleftrightarrow (a \in x \Deltaop y) \xor  a \in z \\
        &\Longleftrightarrow (a \in x \xor a \in y) \xor a \in z \Longleftrightarrow a \in x \xor a \in y \xor a \in z.
    \end{align*}

    The above transformations depends on $\xor$ being associative. This is true as shown by the following table
    \begin{center}
        \begin{tabular}{c c c | c c c c}
            $x$ & $y$ & $z$ & $y \xor z$ & $x \xor y$ & $x \xor (y \xor z)$ & $(x \xor y) \xor z$ \\\midrule
            $0$ & $0$ & $0$ & $0$ & $0$ & $0$ & $0$ \\
            $0$ & $0$ & $1$ & $1$ & $0$ & $1$ & $1$ \\
            $0$ & $1$ & $0$ & $1$ & $1$ & $1$ & $1$ \\
            $0$ & $1$ & $1$ & $0$ & $1$ & $0$ & $0$ \\
            $1$ & $0$ & $0$ & $0$ & $1$ & $1$ & $1$ \\
            $1$ & $0$ & $1$ & $1$ & $1$ & $0$ & $0$ \\
            $1$ & $1$ & $0$ & $1$ & $0$ & $0$ & $0$ \\
            $1$ & $1$ & $1$ & $0$ & $0$ & $1$ & $1$ \\
        \end{tabular}
    \end{center}

    We have thus shown that $a \in x \Deltaop (y \Deltaop z) \Leftrightarrow a \in (x \Deltaop y) \Deltaop z$ which is equivalent to $x \Deltaop (y \Deltaop z) = (x \Deltaop y) \Deltaop z$.

    \item contains a neutral element by showing that there exists an $e \in \mathcal{P}(X)$ such that $e \Deltaop x = x = x \Deltaop e$ for arbitrary $x \in \mathcal{P}(X)$.
    
    Consider that $x \Deltaop e = (x \backslash e) \cup (e \backslash x)$. Choose $e = \emptyset$. We now have
    \begin{align*}
        x \Deltaop e &= (x \backslash \emptyset) \cup (\emptyset \backslash x) = x \quad \text{and}\\
        e \Deltaop x &= (\emptyset \backslash x) \cup (x \backslash \emptyset) = x
    \end{align*}
    since $x \backslash \emptyset = x$, $\emptyset \backslash x = \emptyset$ and $x \cup \emptyset = \emptyset \cup x = x$.

    This assertion depends on $\cup$ being commutative. Consider two sets $X$ and $Y$ and an $a \in X \cup Y$. We know that $a \in X$ and/or $a \in Y$. Thus $a \in Y \cup X$. If $a \not\in X \cup Y$ then $a$ is neither in $X$ nor $Y$, it is thus also not in $Y \cup X$.

    \item contains an inverse for each element by showing that for each $x \in \mathcal{P}(X)$ there exists an $x^{-1} \in \mathcal{P}(X)$ such that $x \Deltaop x^{-1} = x^{-1} \Deltaop x = \emptyset$.
    
    Consider again that $x \Deltaop e = (x \backslash e) \cup (e \backslash x)$. Choose $x^{-1} = x$. We now have
    \begin{align*}
        x^{-1} \Deltaop x = x \Deltaop x^{-1} &= (x \backslash x) \cup (x \backslash x) = \emptyset
    \end{align*}
    since $x \backslash x = \emptyset$.

    \item is commutative by showing that $x \Deltaop y = y \Deltaop x$ holds for arbitrary $x, y \in \mathcal{P}(X)$. We have
    \begin{align*}
        x \Deltaop y &= y \Deltaop x \\
        (x \backslash y) \cup (y \backslash x) &= (y \backslash x) \cup (x \backslash y)
    \end{align*}
    which holds since $\cup$ is commutative.
\end{enumerate}

\pagebreak
\paragraph{Exercise 3}

\begin{enumerate}
    \item To show $x^{-1^{-1}} = x$ consider that, by definition, $x \circ x^{-1} = e$ and that $x^{-1} \circ x^{-1^{-1}} = e$. Thus we have $x \circ x^{-1} = x^{-1} \circ x^{-1^{-1}}$ and further $x^{-1^{-1}} = x$.
    
    \item To show that $(x \circ y)^{-1} = y^{-1} \circ x^{-1}$ consider that $x \circ x^{-1} = e$ and $y \circ y^{-1} = e$. We have that
    \begin{align*}
        e &= x \circ x^{-1} \\
        e &= x \circ e \circ x^{-1} \\
        e &= x \circ y \circ y^{-1} \circ x^{-1} \\
        (x \circ y)^{-1} &= y^{-1} \circ x^{-1}.
    \end{align*}
\end{enumerate}

\pagebreak
\paragraph{Exercise 4}

Assuming that $(\mathcal{P}(X), \cup, \cap)$ is a ring, $(\mathcal{P}(X), \cup)$ has to be an abelian group and therefore each element $x \in (\mathcal{P}(X)$ has to have an inverse $y$ such that $x \cup y = e$ where $e$ is the neutral element.

The neutral element of $(\mathcal{P}(X), \cup)$ is $\emptyset$ since $\emptyset \cup x = x \cup \emptyset = x$. Consider that, for arbitrary sets $X$ and $Y$, $|X| \leq |X \cup Y| \geq |Y|$. Since $|\emptyset| = 0$, the cardinality of $x \cup y$ has to be zero if $x \cup y = e$. Thus $(\mathcal{P}(X), \cup)$ only contains an inverse for every $x \in \mathcal{P}(X)$ if $X = \emptyset$.
P(S)
It now remains to be shown that $(\mathcal{P}(\emptyset), \cup, \cap)$ is a ring. We have $x, y, z \in \mathcal{P}(\emptyset)$, $x = y = z = \emptyset$ and will show that,
\begin{enumerate}
    \item $(\mathcal{P}(\emptyset), \cup)$ is associative by showing that we have
    \begin{align*}
        x \cup (y \cup z) &= (x \cup y) \cup z \\
        \emptyset \cup (\emptyset \cup \emptyset) &= (\emptyset \cup \emptyset) \cup \emptyset
    \end{align*}
    
    \item $(\mathcal{P}(\emptyset), \cup)$ contains a neutral element. Already shown for arbitrary $X$.
    
    \item $(\mathcal{P}(\emptyset), \cup)$ is commutative by
    \begin{align*}
        x \cup y &= y \cup x \\
        \emptyset \cup \emptyset &= \emptyset \cup \emptyset.
    \end{align*}
    
    \item $(\mathcal{P}(\emptyset), \cap)$ is associative by
    \begin{align*}
        x \cap (y \cap z) &= (x \cap y) \cap z \\
        \emptyset \cap (\emptyset \cap \emptyset) &= (\emptyset \cap \emptyset) \cap \emptyset.
    \end{align*}
    
    \item The distributive law,
    \begin{align*}
        x \cap (y \cup z) &= x \cap y \cup x \cap z \\
        \emptyset \cap (\emptyset \cup \emptyset) &= \emptyset \cap \emptyset \cup \emptyset \cap \emptyset
    \end{align*}
    holds.
\end{enumerate}

\pagebreak
\paragraph{Exercise 5}

To show that $(\mathcal{P}(X), \Delta, \cap)$ is a commutative ring we will show that
\begin{enumerate}
    \item $(\mathcal{P}(X), \Delta)$ is an abelian group. This was done as part of exercise 2.
    
    \item $(\mathcal{P}(X), \cap)$ is a monoid. We will assume associativity and thus only show that it contains a neutral element $e$ such that $e \cap x = x = x \cap e$.
    
    Assume $e = x$, we now have $e \cap x = x \cap x = x \cap e = x$.
    
    \item $(\mathcal{P}(X), \cap)$ is commutative by showing that $x \cap y = y \cap x$.
    
    Consider $a \in x \cap y$, we know $a \in x$ and $a \in y$ thus $a \in y \cap x$.
    
    Consider $a \not\in x \cap y$, we know $a \not\in x$ and $a \not\in y$ thus $a \not\in y \cap x$.
    
    \item the distributive law, $x \cap (y \Deltaop z) = x \cap y \Deltaop x \cap z$, holds.
    
    Consider $a \in x \cap (y \Deltaop z)$. We know that $a \in x$ and $a$ in either $z$ or $y$ but not both. Thus $a \in x \cap y$ or $a \in x \cap z$, but not both. Thus $a \in x \cap y \Deltaop x \cap z$.

    Consider $a \in x \cap y \Deltaop x \cap z$. We know that $a$ is either in $x$ and $y$ or in $x$ and $z$. It is thus definitely in $x$ and in either $y$ or $z$ but not in both. Thus $a \in x \cap (y \Deltaop z)$.
\end{enumerate}

\pagebreak
\paragraph{Exercise 6}

\begin{center}
    \begin{tabular}{c | c c c c}
        + & 0 & 1 & 2 & 3 \\\midrule
        0 & 0 & 1 & 2 & 3 \\
        1 & 1 & 2 & 3 & 0 \\
        2 & 2 & 3 & 0 & 1 \\
        3 & 3 & 0 & 1 & 2 \\
    \end{tabular}
    \qquad
    \begin{tabular}{c | c c c c}
        $\cdot$ & 0 & 1 & 2 & 3 \\\midrule
        0 & 0 & 0 & 0 & 0 \\
        1 & 0 & 1 & 2 & 3 \\
        2 & 0 & 2 & 0 & 2 \\
        3 & 0 & 3 & 2 & 1 \\
    \end{tabular}
\end{center}

We know that $(\mathbb{Z}_4, +, \cdot)$ is a ring. For it to be a commutative ring $(\mathbb{Z}_4, \cdot)$ has to be commutative. This is given since the respective table is symmetrical alongside the main diagonal.

According to axiom 1.20 it is not a field since 4 is not prime. (It states that $\mathbb{Z}_p$ is a field if and only if $p$ is prime.) Concretely, there doesn't exist an inverse for every element (0, 2).

\pagebreak
\paragraph{Exercise 7}

We have
\begin{align*}
    f &= \begin{pmatrix}
        1 & 2 & 3 & 4 & 5 \\
        1 & 5 & 4 & 3 & 2
    \end{pmatrix}&
    g &= \begin{pmatrix}
        1 & 2 & 3 & 4 & 5 \\
        3 & 5 & 1 & 2 & 4
    \end{pmatrix}&
    g^{-1} &= \begin{pmatrix}
        1 & 2 & 3 & 4 & 5 \\
        3 & 4 & 1 & 5 & 2
    \end{pmatrix}& \\
    h &= \begin{pmatrix}
        1 & 2 & 3 & 4 & 5 \\
        1 & 3 & 2 & 4 & 5
    \end{pmatrix}&
    x &= \begin{pmatrix}
        1 & 2 & 3 & 4 & 5 \\
        4 & 1 & 3 & 5 & 2
    \end{pmatrix}
\end{align*}

\pagebreak
\paragraph{Exercise 8}

\begin{align*}
    14x + 6^8 &= 10 \\
    14x + 18 &= 10 \\
    14x &= 15 \\
    x &= 14^{-1} \cdot 15 \\
\end{align*}

\end{document}
