\documentclass{article}
\usepackage[utf8]{inputenc}
\usepackage[ngerman]{babel}

% Convenience improvements
\usepackage{csquotes}
\usepackage{enumitem}
\setlist[enumerate,1]{label={\alph*)}}
\usepackage{amsmath}
\usepackage{amssymb}
\usepackage{mathtools}
\usepackage{tabularx}

% Proper tables and centering for overfull ones
\usepackage{booktabs}
\usepackage{adjustbox}

% Change page/text dimensions, the package defaults work fine
\usepackage{geometry}

\usepackage{parskip}

% Drawings
\usepackage{tikz}

% Adjust header and footer
\usepackage{fancyhdr}
\pagestyle{fancy}
\fancyhead[L]{Elektronik --- \textbf{Blatt 1}}
\fancyhead[R]{Laurenz Weixlbaumer (11804751)}
\fancyfoot[C]{}
\fancyfoot[R]{\thepage}
% Stop fancyhdr complaints
\setlength{\headheight}{12.5pt}

\begin{document}

\paragraph{Aufgabe 3}

\begin{enumerate}
    \item Es gilt
    \begin{align*}
        R &= \rho \frac{l}{A} \\
        100 \Omega &= 6 \Omega mm^2 m^{-1} \frac{l}{5000mm^2} \\
        l &= \frac{100 \Omega \cdot 5000 mm^2}{6 \Omega mm^2 m^{-1}} = 83333m
    \end{align*}

    Somit ist der gesuchte Punkt $\frac{83333m}{2} = 41666m$ vom Messpunkt entfernt.

    \item Es gilt
    \begin{align*}
        l &= 6cm \cdot 500 = 3000cm \\
        \rho_{Al} &= 2.65 \cdot 10^{-2} \Omega mm^2 m^{-1} \\
        A &= \frac{d}{2}^2 \pi = 0.1963mm^2
    \end{align*}
    und somit
    \begin{align*}
        R &= \rho \frac{l}{A} \\
        R &= 2.65 \cdot 10^{-2} \Omega mm^2 m^{-1} \frac{3000cm}{0.1963mm^2} \\
        R &= 40.5 \Omega.
    \end{align*}
\end{enumerate}

\paragraph{Aufgabe 4}

\begin{enumerate}
    \item Es gilt
    \begin{align*}
        A = \frac{d}{2}^2 \pi = 0.0314mm^2
    \end{align*}
    und somit
    \begin{align*}
        J &= \frac{I}{A} \\
        4 Amm^{-2} &= \frac{I}{0.0314mm^2} \\
        I &= 0.1257 A.
    \end{align*}

    \item
    \begin{align*}
        \frac{\rho \frac{l_1}{A_1}}{\rho \frac{l_2}{A_2}} = 
        \frac{\rho \frac{l_1}{0.25\pi}}{\rho \frac{4l_1}{4\pi}} =
        \frac{4 \pi l_1}{4 l_1 0.25 \pi} = \frac{1}{0.25} = 4
    \end{align*}
\end{enumerate}

\end{document}
