\documentclass{article}
\usepackage[utf8]{inputenc}
\usepackage[ngerman]{babel}

% Convenience improvements
\usepackage{csquotes}
\usepackage{enumitem}
\setlist[enumerate,1]{label={\alph*)}}
\usepackage{amsmath}
\usepackage{amssymb}
\usepackage{mathtools}
\usepackage{tabularx}

% Proper tables and centering for overfull ones
\usepackage{booktabs}
\usepackage{adjustbox}

% Change page/text dimensions, the package defaults work fine
\usepackage{geometry}

\usepackage{parskip}

% Drawings
\usepackage{tikz}

% Adjust header and footer
\usepackage{fancyhdr}
\pagestyle{fancy}
\fancyhead[L]{Elektronik --- \textbf{Blatt 1}}
\fancyhead[R]{Laurenz Weixlbaumer (11804751)}
\fancyfoot[C]{}
\fancyfoot[R]{\thepage}
% Stop fancyhdr complaints
\setlength{\headheight}{12.5pt}

\begin{document}

\paragraph{Aufgabe 3}

\begin{enumerate}
    \item Es gilt
    \begin{align*}
        R &= \rho \frac{l}{A} \\
        100 \Omega &= 6 \Omega mm^2 m^{-1} \frac{l}{5000mm^2} \\
        l &= \frac{100 \Omega \cdot 5000 mm^2}{6 \Omega mm^2 m^{-1}} = 83333m
    \end{align*}

    Somit ist der gesuchte Punkt $\frac{83333m}{2} = 41km$ vom Messpunkt entfernt.

    \item Es gilt
    \begin{align*}
        l &= 6cm \cdot 500 = 3000cm \\
        \rho_{Al} &= 2.65 \cdot 10^{-2} \Omega mm^2 m^{-1} \\
        A &= \left(\frac{d}{2}\right)^2 \pi = 0.1963mm^2
    \end{align*}
    und somit
    \begin{align*}
        R &= \rho \frac{l}{A} \\
        R &= 2.65 \cdot 10^{-2} \Omega mm^2 m^{-1} \frac{30000mm}{0.1963mm^2} \\
        R &= 4.05 \Omega.
    \end{align*}
\end{enumerate}

\paragraph{Aufgabe 4}

\begin{enumerate}
    \item Es gilt
    \begin{align*}
        A = \left(\frac{d}{2}\right)^2 \pi = 0.0314mm^2
    \end{align*}
    und somit
    \begin{align*}
        J &= \frac{I}{A} \\
        4 Amm^{-2} &= \frac{I}{0.0314mm^2} \\
        I &= 125.66mA.
    \end{align*}

    \item Es gilt
    \begin{align*}
        V_1 &= \pi \left(\frac{d_1}{2}\right)^2 l_1 & V_2 &= \pi \left(\frac{d_2}{2}\right)^2 l_2 \\
        A_1 &= \pi \left(\frac{d_1}{2}\right)^2 & A_2 &= \pi \left(\frac{d_2}{2}\right)^2
    \end{align*}
    Wir sind interessiert an $l_2$ und wissen, dass das Volumen konstant bleibt. Somit
    \begin{align*}
        \pi \left(\frac{d_1}{2}\right)^2 l_1 &= \pi \left(\frac{d_2}{2}\right)^2 l_2 \\
        \pi \left(\frac{4}{2}\right)^2 l_1 &= \pi \left(\frac{1}{2}\right)^2 l_2 \\
        4l_1 &= \frac{1}{4}l_2 \\
        16l_1 &= l_2.
    \end{align*}
    Damit kann nun das Verh\"altnis zwischen $R_D$ und $R_D'$ berechnet werden, wobei $R_D'$ den Widerstand des verl\"angerten Drahtes repr\"asentiert.
    \begin{align*}
        \frac{R_D'}{R_D} = 
        \frac{\rho \frac{l_2}{A_2}}{\rho \frac{l_1}{A_1}} = 
        \frac{16l_1 \pi \left(\frac{d_1}{2}\right)^2}{l_1\pi \left(\frac{d_2}{2}\right)^2} =
        \frac{16 \cdot \left(\frac{4}{2}\right)^2}{\left(\frac{1}{2}\right)^2} =
        \frac{64}{\frac{1}{4}} =
        256
    \end{align*}
    Es gilt also $R_D' = 256R_D$.
\end{enumerate}

\end{document}
