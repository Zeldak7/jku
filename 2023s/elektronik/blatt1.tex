\documentclass{article}
\usepackage[utf8]{inputenc}
\usepackage[ngerman]{babel}

% Convenience improvements
\usepackage{csquotes}
\usepackage{enumitem}
\setlist[enumerate,1]{label={\alph*)}}
\usepackage{amsmath}
\usepackage{amssymb}
\usepackage{mathtools}
\usepackage{tabularx}

% Proper tables and centering for overfull ones
\usepackage{booktabs}
\usepackage{adjustbox}

% Change page/text dimensions, the package defaults work fine
\usepackage{geometry}

\usepackage{parskip}

% Drawings
\usepackage{tikz}

% Units
\usepackage{siunitx}
\sisetup{locale=DE,round-mode=figures,round-precision=4,round-pad=false}


% Adjust header and footer
\usepackage{fancyhdr}
\pagestyle{fancy}
\fancyhead[L]{Elektronik --- \textbf{Blatt 1}}
\fancyhead[R]{Laurenz Weixlbaumer (11804751)}
\fancyfoot[C]{}
\fancyfoot[R]{\thepage}
% Stop fancyhdr complaints
\setlength{\headheight}{12.5pt}

\begin{document}

\paragraph{Aufgabe 3}

\begin{enumerate}
    \item \begin{align*}
        R &= \frac{\rho_s \cdot l}{A} \\
        l &= \frac{A \cdot R}{\rho_s} = \frac{\qty{4000}{\mm\squared} \cdot \qty{250}{\ohm}}{\qty{10}{\ohm\mm\squared\per\metre}} = \qty{100000}{\metre}
    \end{align*}
    Somit ist der Kurzschluss \qty{50000}{\metre} von der Messstelle entfernt.

    \item Die Querschnittsfläche des Drahtes ist $\pi r^2 = \pi \cdot 0.25^2 = \qty{0.19635}{\mm\squared}$. Wegen $J = I/A$ ist der maximal zulässige Strom also $\qty{1.5}{\ampere\per\mm\squared} \cdot \qty{0.19635}{\mm\squared} = \qty{0.294524}{\ampere}$
\end{enumerate}

\paragraph{Aufgabe 4}

\begin{enumerate}
    \item Der spezifische Widerstand von Eisen ist $\rho_e \approx \qty{.13}{\ohm\mm\squared\per\m}$. Der Draht ist \qty{75}{\m} lang (Spule hat 750 Windungen mit durchschnittlich 10cm Draht) und hat eine Querschnittsfläche von $\pi \cdot 0.2&\qty{0.12566370614359174}{\mm\squared}$. Somit
    \begin{align*}
        R = \frac{\rho_e \cdot l}{A} = \frac{\qty{.13}{\ohm\mm\squared\per\m} \cdot \qty{75}{\m}}{\qty{0.12566370614359174}{\mm\squared}} = \qty{77.58803475729897}{\ohm}
    \end{align*}

    \item
    \begin{align*}
        V_1 &= \pi \left(\frac{d_1}{2}\right)^2 l_1 & V_2 &= \pi \left(\frac{d_2}{2}\right)^2 l_2 \\
        A_1 &= \pi \left(\frac{d_1}{2}\right)^2 & A_2 &= \pi \left(\frac{d_2}{2}\right)^2
    \end{align*}
    Wir sind interessiert an $l_2$ und wissen, dass das Volumen konstant bleibt. Somit
    \begin{align*}
        \pi \left(\frac{d_1}{2}\right)^2 l_1 &= \pi \left(\frac{d_2}{2}\right)^2 l_2 \\
        \pi \left(\frac{4}{2}\right)^2 l_1 &= \pi \left(\frac{1}{2}\right)^2 l_2 \\
        4l_1 &= \frac{1}{4}l_2 \\
        16l_1 &= l_2.
    \end{align*}
    Es sei $R'_D$ der Widerstand des auseinandergezogenen Drahtes.
    \begin{align*}
        \frac{R_D'}{R_D} = 
        \frac{\rho \frac{l_2}{A_2}}{\rho \frac{l_1}{A_1}} = 
        \frac{16l_1 \pi \left(\frac{d_1}{2}\right)^2}{l_1\pi \left(\frac{d_2}{2}\right)^2} =
        \frac{16 \cdot \left(\frac{4}{2}\right)^2}{\left(\frac{1}{2}\right)^2} =
        \frac{64}{\frac{1}{4}} =
        256
    \end{align*}
    Es gilt also $R_D' = 256R_D$.
\end{enumerate}

\end{document}
