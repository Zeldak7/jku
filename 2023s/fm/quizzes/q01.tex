\documentclass{article}
\usepackage[utf8]{inputenc}
\usepackage[ngerman]{babel}

\usepackage{multicol}

% Convenience improvements
\usepackage{csquotes}
\usepackage{enumitem}
\setlist[enumerate,1]{label={\alph*)}}
\usepackage{amsmath}
\usepackage{amssymb}
\usepackage{mathtools}
\usepackage{tabularx}

% Proper tables and centering for overfull ones
\usepackage{booktabs}
\usepackage{adjustbox}

% Change page/text dimensions, the package defaults work fine
\usepackage{geometry}

\usepackage{parskip}

% Drawings
\usepackage{tikz}

% Units
\usepackage{siunitx}
\sisetup{locale=DE,round-mode=figures,round-precision=4,round-pad=false}


% Adjust header and footer
\usepackage{fancyhdr}
\pagestyle{fancy}
% Stop fancyhdr complaints
\setlength{\headheight}{12.5pt}

\begin{document}

% Determine whether a given automaton $A$ is deterministic or complete. Determine its 5-tuple and the language $L(A)$ it decides.

% What is a complement, power, (optimized) oracle, product or I/O automaton, when are they complete or deterministic and what are their accepted languages. How can they be constructed.

\section*{Finite Automata}

A finite \emph{automaton} $A = (S, I, \Sigma, T, F)$ consists of
\begin{itemize}
    \item a set of states $S$
    \item a set of initial states $I \subseteq S$
    \item an input alphabet $\Sigma$
    \item a transition relation $T \subseteq S \times \Sigma \times S$
    \item a set of final states $F \subseteq S$
\end{itemize}

The \emph{language} $L(A)$ of $A$ is the set of words accepted by $A$.

An automaton is called \emph{complete} iff
\begin{itemize}
    \item it has at least one initial state
    \item every state has at least one outgoing transition for every $e \in \Sigma$.
\end{itemize}

An automaton is called \emph{deterministic} iff
\begin{itemize}
    \item it has at most one initial state
    \item every state has at most one outgoing transition for every $e \in \Sigma$.
\end{itemize}

\paragraph{Power Automaton} $\mathbb{P}(A) = (\mathbb{P}(S), I_p, \ldots)$ is deterministic and complete with $L(A) = L(\mathbb{P}(A))$.

\paragraph{Complement Automaton} $C(A) = (\ldots, S \backslash F)$ is the result of flipping the final states of $A$. $L(C(A)) = \overline{L(A)}$ iff $A$ is complete and deterministic.

\paragraph{Oracle Automaton} $\text{Oracle}(A) = (S, I, \Sigma \times S, T_O, F)$ where transitions are now pairs $(e \in \Sigma, s \in S)$ with $e$ being the previous alphabet value and $s$ being the destination state. It is deterministic iff $|I| \leq 1$ and usually not complete.

\paragraph{Optimized Oracle Automaton} is a modification of $\text{Oracle}(A)$ which is deterministic \emph{and} complete iff $|I| \leq 1$. This is done by replacing destination states in the transition pairs with numbers and adding transitions where necessary.

\paragraph{Product Automaton} $A_1 \times A_2$ of $A_1$ and $A_2$ accepts $L(A) = L(A_1) \cap L(A_2)$

\end{document}
