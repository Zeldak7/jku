\documentclass[twocolumn]{article}

% Font
\usepackage[T2A]{fontenc}
% Language
\usepackage[ngerman]{babel}
% Encoding
\usepackage[utf8]{inputenc}

% Custom header/footer
\usepackage{fancyhdr}

% Make LaTeX better
\usepackage{enumitem}
\usepackage{booktabs}
\usepackage{csquotes}

% Sizing
\usepackage{geometry}
\geometry{
    a4paper,
    total={170mm,257mm},
    left=20mm,
    top=20mm,
}

% The \term command is used to introduce new terminology. It should only be used when the term is first introduced.
\newcommand{\term}[1]{\textbf{#1}}

% Visually separate content. Can be used to highlight a certain paragraph or to signify a context break.
\newcommand{\separator}{\vspace{0.5em}\noindent}

\begin{document}

\section{Urheberrecht}

Ein \term{Werk} ist eine \enquote{eigentümliche geistige Schöpfung auf den Gebieten der Literatur, der Tonkunst, der bildenden Künste und der Filmkunst.} Auch Teile eines Werks können geschützt sein, geistig meint hier von Menschen geschaffen.

\separator
\term{Eigentümlich} ist ein Werk dann, wenn es etwas \enquote{besonderes} ist, also von der Person des Schöpfers abhängt. Andere Personen hätten es anders gemacht, es muss aber nicht neu, schön, künstlerisch, \ldots sein.

\separator
Eine \term{Schöpfung} muss konkret und sinnlich wahrnehmbar sein (Buch, Musik, \ldots). Keine Schöpfungen sind etwa Ideen, Theorien, Methoden, Algorithmen.

\separator
Die \term{Schutzdauer} eines Werkes beträgt 70 Jahre nach dem Tod des letzten Mituhrhebers\footnote{Bei Filmwerken nur Regisseur, Drehbuchautor, Dialogautor, Komponist. Keine solche Sonderregelung für Computerprogramme.}, wenn der Urheber bekannt ist, nach dem Zeitpunkt der Schaffung.

\subsection{Dem Uhrheber vorbehaltenes}
\label{ssec:uhrheber-vorbehalten}

Verwertung durch Verfielfältigung, Verbreitung\footnote{Erschöpfung: Nach erstem Verkauf \enquote{frei}.}, das Vermieten und Verleihen, Vorführung, öffentlichem Zurverfügungstellen vorbehalten.

Künstler die Werke billig verkaufen welche später stark im Wert steigen werden an Weiterverkäufen beteiligt (Folgerecht). Recht auf \enquote{Sendung} (Senderecht; Kabel, Drahtlos, Satelit).

Die Urheberschaft ist geschützt, der Uhrheber kann sich als solcher reklamieren (inkl. Urheberbezeichnung). Schutz vor Entstellungen oder Bearbeitungen/Übersetzungen. Schutz vor Erstveröffentlichung/Inhaltsbekanntgabe.

\subsection{Freie Werknutzung}

Ausnahmen von den Exklusivrechten der Urheber. Man darf etwas, weil es so gesetzlich erlaubt ist (eng und im Sinne des Urhebers ausgelegt).

Ebenso für staatl. Zwecke (öffentl. Sicherheit, Verwaltungs/Gesetzgebungsverfahren, Gerichtsverfahren) ohne Entschädigung der Urheber. Nur als Beweismittel, nicht als Hilfsmittel.

\subsection{Begleitende Vervielfältigungen}

\ldots sind erlaubt, unter folgenden Bedingungen:
\begin{itemize}
    \item Flüchtige oder begleitende Vervielfältigung. Zeitlich kurz und automatisch wieder weg, etwa Browser-Cache (begleitend) oder Anzeige am Bildschirm (flüchtig).
    \item Integraler Teil eines techn. Verfahrens.
    \item Alleiniger Zweck ist eine rechtmäßige Nutzung. (Oder eine Übertragung in einem Netz zwischen Dritten durch einen Vermittler.)
    \item Keine eigenständige wirtschaftliche Bedeutung.
\end{itemize}
Etwa Kopien in einem Switch, Proxies (Weiterleitungen), \enquote{Weg zur Antenne}.

\subsection{Kopie für den eigenen Gebrauch}

Gilt für jeden (auch juristische Personen), darf aber nur auf \enquote{Papier oder einem ähnlichen Träger} erfolgen. Nicht erlaubt ist sie wenn die Vorlage offensichtlich rechtswidrig hergestellt wurde oder offensichtlich rechtswidrig öffentlich zugänglich gemacht wurde.

Darf nicht mit dem Ziel der Öffentlichmachung erstellt werden (und auch später nicht dazu verwendet werden).

\subsection{Privatkopie}

Ähnlich s. o. aber auch auf anderen Medien (z. B. digital) und nur für natürliche Personen.

Ein P2P-Download (und Video-Streaming ohne Abo) ist damit fast immer illegal, nachdem die Quelle/Vorlage offensichtlich rechtswidrig ist. (Selbst wenn die Quelle eine legale Privatkopie war, sie wurde der Öffentlichkeit zugänglich gemacht und ist somit rechtswidrig.) P2P-Upload ist immer illegal, Privatkopien sind nur für den eigenen Gebrauch.

\subsection{Zitate}

\ldots sind für alle Arten von Werken möglich, wenn das Ursprungswerk legal veröffentlicht wurde. Allgemein dürfen nur Teile verwendet werden, bei Bildern alles.

Erforderlich ist deutliche Quellenangabe (Originaltitel, Urheber, wo gefunden --- etwa Seitenanzahl). Zitate müssen der Erläuterung oder der inhaltlichen Auseinandersetzung dienen.

\subsection{Speichermedienvergütung}

Abgeltung für Privatkopie/Kopie für eigenen Gebrauch, wird bei Kauf des Mediums bezahlt. Theorie: Entsprechend der durchschnittlichen Nutzung, in der Praxis schwer feststellbar (bisherige/ausländische Vergütungssätze/Schaden für Urheber durch Vervielfältigung/Vorteil des Nutzers \ldots).

\subsection{Ausnahmen}

Als \enquote{unwesentliches Beiwerk} darf ein Werk wiederverwendet werden. (Zufällig, beiläufig und ohne Bezug zum eigentlichen Gegenstand der Verwertungshandlung.)

Menschen mit Behinderungen dürfen ein barrierefreies Werk herstellen wenn sie rechtmäßigen Zugang haben. (Sicherheitsmaßnahmen dürfen nicht umgangen werden.)



\end{document}